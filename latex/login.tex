
\section{用户帐号系统}
\subsection{新浪微博相关接口}

\subsubsection{网页登录接口:oauth2}


\begin{table}[H]
   \begin{center}
\begin{tabular}{|c|c|c|p{12cm}|}
\hline
POST & \multicolumn{3}{|c|}{/oauth2/sina\_callback} \\
\hline\hline
 \  参数  & 类型 & 必填 &  说明  \\
\hline
 code  & 字符串 & 必须 &  新浪返回的code\\
\hline
\end{tabular}
   \end{center}
\end{table}


例子:

\begin{figure}[H]
\begin{verbatim}
访问:/oauth2/sina_callback?code=...
该访问由新浪微博通过302转向发起。
返回值:

{
"id":1,
"password":"15c663b8ff620502",
"logo":"",
"name" : "name"
"gender" : 1
"wb_uid":"1884834632",
"expires_in":75693,
"token":"2.00aaZYDCMcnDPCb4dc439e06i2m_GC",
"expires_at":1338431259
}

其中,id和password是该用户在脸脸网的id和密码,在首次新浪授权时创建。该id加上"@dface.cn"就是openfire的jid。(目前还没有和jid关联上。)

后面几个字段都是新浪提供的,其中wb_uid是该用户的新浪微博的uid,token是授权的access_token,其它几个是授权有效期信息。

在首次登录获得用户的id和password以后,可以缓存到本地。以后用户每次登录,返回的id和password是不会改变的,除非服务器端重置密码(比如出现密码泄漏等安全情况)。

logo可以得知用户是否成功上传头像。对于第一次登录的用户,其logo为空,此时需要引导用户上传头像。

本接口不再需要,登录逻辑为:如果客户端安装了支持SSO的微博客户端,那么使用SSO的接口登录;否则使用xauth的接口登录。


\end{verbatim}
\end{figure}

\subsubsection{客户端登录接口:xauth}
\label{hash_algorithm}

\begin{table}[H]
   \begin{center}
\begin{tabular}{|c|c|c|p{12cm}|}
\hline
POST & \multicolumn{3}{|c|}{/oauth2/login} \\
\hline\hline
 \  参数  & 类型 & 必填 &  说明  \\
\hline
 name  & 字符串 & 必须 &  用户名\\
 \hline
 pass  & 字符串 & 必须 &  密码\\
  \hline
 mac  & 字符串 & 必须 &  网卡Mac地址\\
 \hline
 hash  & 字符串 & 必须 &  hash验证码\\
\hline
 bind  & 整数 & 可选 &  bind为0表示正常登录,1表示绑定微博帐户,2表示已经绑定过微博且已经用qq登录时,同时登录微博。默认为0\\
\hline
\end{tabular}
   \end{center}
\end{table}

本接口的输出和oauth2登录接口的输出相同。

\begin{verbatim}
hash的计算方式为:name+pass+mac+"dface"进行SHA1算法后取前32位。
比如用户名为name,密码为pa,mac地址为ss
那么待hash的字符串为"namepassdface",
其hash码为"11f4004a73a65117071bc4a7d3dfdf07"。
\end{verbatim}

新的登录方式不需要调用新浪的Xauth API,直接调用本API即可以登录。客户端应该不需要保存AppSecret。通过预先保存的AppKey和本调用输出中包含的token应该就可以访问新浪微博的API了。




\subsubsection{客户端SSO接口}
\label{hash_algorithm}

\begin{table}[H]
   \begin{center}
\begin{tabular}{|c|c|c|p{12cm}|}
\hline
POST & \multicolumn{3}{|c|}{/oauth2/sso} \\
\hline\hline
 \  参数  & 类型 & 必填 &  说明  \\
\hline
 remind\_in  & 字符串 & 必须 &  过期时间\\
 \hline
 expires\_in  & 字符串 & 必须 &  过期时间\\
  \hline
 uid  & 字符串 & 必须 &  微博的uid\\
  \hline
 access\_token  & 字符串 & 必须 &  微博的token\\
 \hline
 hash  & 字符串 & 必须 &  hash验证码\\
\hline
 bind  & 整数 & 可选 &  bind为0表示正常登录,1表示绑定微博帐户,2表示已经绑定过微博且已经用qq登录时,同时登录微博。默认为0\\
\hline
\end{tabular}
   \end{center}
\end{table}
当手机客户端安装有新浪官方微博客户端,使用微博SSO登录,调用本接口。
hash的计算方式为:uid+access\_token+"dface"进行SHA1算法后取前32位。

\subsubsection{解除新浪微博绑定}

\begin{table}[H]
   \begin{center}
\begin{tabular}{|c|c|c|p{12cm}|}
\hline
POST & \multicolumn{3}{|c|}{/oauth2/unbind\_sina} \\
\hline\hline
 \  参数  & 类型 & 必填 &  说明  \\
\hline
    uid  & 字符串 & 必须 &  当前用户的id\\
\hline
    wb\_uid  & 字符串 & 必须 &  新浪微博的uid\\    
\hline
\end{tabular}
   \end{center}
\end{table}
解除成功返回{unbind: true},否则返回error信息。



\subsection{QQ相关接口}
\subsubsection{QQ登录接口}
\label{hash_algorithm}

\begin{table}[H]
   \begin{center}
\begin{tabular}{|c|c|c|p{12cm}|}
\hline
POST & \multicolumn{3}{|c|}{/oauth2/qq\_client} \\
\hline\hline
 \  参数  & 类型 & 必填 &  说明  \\
 \hline
 expires\_in  & 字符串 & 必须 &  过期时间\\
  \hline
 openid  & 字符串 & 必须 &  QQ的openid\\
  \hline
 access\_token  & 字符串 & 必须 &  QQ授权的token\\
 \hline
 hash  & 字符串 & 必须 &  hash验证码\\
 \hline
 bind  & 整数 & 可选 &  bind为0表示正常登录,1表示绑定qq帐户,2表示已经绑定过qq且已经用微博登录时,同时登录qq。默认为0\\

\hline
\end{tabular}
   \end{center}
\end{table}
手机客户端使用QQ提供的SDK登录后,调用本接口。
hash的计算方式为:openid+access\_token+"dface"进行SHA1算法后取前32位。

\subsubsection{解除QQ绑定}

\begin{table}[H]
   \begin{center}
\begin{tabular}{|c|c|c|p{12cm}|}
\hline
POST & \multicolumn{3}{|c|}{/oauth2/unbind\_qq} \\
\hline\hline
 \  参数  & 类型 & 必填 &  说明  \\
\hline
    uid  & 字符串 & 必须 &  当前用户的id\\
\hline
    qq  & 字符串 & 必须 &  QQ分配的openid\\    
\hline
\end{tabular}
   \end{center}
\end{table}
解除成功返回{unbind: true},否则返回error信息。
新浪微博和QQ只有都处在有效登录状态时,才能解除其中的一个的绑定。



\subsection{手机注册接口}

\subsubsection{手机号码获取验证码}
\label{hash_algorithm}

\begin{table}[H]
   \begin{center}
\begin{tabular}{|c|c|c|p{12cm}|}
\hline
POST & \multicolumn{3}{|c|}{/phone/init} \\
\hline\hline
 \  参数  & 类型 & 必填 &  说明  \\
\hline
 phone  & 字符串 & 必须 &  用户的手机号码\\
\hline
 type  & 整数 & 可选 &  1代表注册,2代表绑定,3代表忘记密码\\
\hline
\end{tabular}
   \end{center}
\end{table}

调用成功返回{"code":HASH后的验证码},如果有错误,错误信息保存在error中。
同时验证码通过短信的方式发送到使用者的手机。

手机号码注册/忘记密码/绑定手机号码都要先调用本接口获取验证码。

[3013-08-11] 增加type参数,表示请求手机验证码的类型。同时取消客户端对验证码的HASH是否正确的判断

\subsubsection{手机号码注册}
\label{hash_algorithm}

\begin{table}[H]
   \begin{center}
\begin{tabular}{|c|c|c|p{12cm}|}
\hline
POST & \multicolumn{3}{|c|}{/phone/register} \\
\hline\hline
 \  参数  & 类型 & 必填 &  说明  \\
\hline
 phone  & 字符串 & 必须 &  用户的手机号码\\
\hline
 code  & 字符串 & 必须 &  验证码\\
\hline
 password  & 字符串 & 必须 &  用户设置的密码\\
\hline
\end{tabular}
   \end{center}
\end{table}

调用成功返回生成的用户id,如果有错误,错误信息保存在error中。


\subsubsection{忘记密码}
\label{hash_algorithm}

\begin{table}[H]
   \begin{center}
\begin{tabular}{|c|c|c|p{12cm}|}
\hline
POST & \multicolumn{3}{|c|}{/phone/forgot\_password} \\
\hline\hline
 \  参数  & 类型 & 必填 &  说明  \\
\hline
 phone  & 字符串 & 必须 &  用户的手机号码\\
\hline
 code  & 字符串 & 必须 &  验证码\\
\hline
 password  & 字符串 & 必须 &  用户设置的密码\\
\hline
\end{tabular}
   \end{center}
\end{table}

本接口和手机号码注册接口参数一样,要求相反。手机号码注册要求该手机号码不存在,忘记密码要求该手机号码已注册。


\subsubsection{修改密码}
\label{hash_algorithm}

\begin{table}[H]
   \begin{center}
\begin{tabular}{|c|c|c|p{12cm}|}
\hline
POST & \multicolumn{3}{|c|}{/phone/change\_password} \\
\hline\hline
 \  参数  & 类型 & 必填 &  说明  \\
\hline
 oldpass  & 字符串 & 必须 &  原来的密码\\
\hline
 password  & 字符串 & 必须 &  用户设置的密码\\
\hline
\end{tabular}
   \end{center}
\end{table}



\subsubsection{手机号码登录}
\label{hash_algorithm}

\begin{table}[H]
   \begin{center}
\begin{tabular}{|c|c|c|p{12cm}|}
\hline
POST & \multicolumn{3}{|c|}{/phone/login} \\
\hline\hline
 \  参数  & 类型 & 必填 &  说明  \\
\hline
 phone  & 字符串 & 必须 &  用户的手机号码\\
\hline
 password  & 字符串 & 必须 &  用户设置的密码\\
\hline
\end{tabular}
   \end{center}
\end{table}

调用成功返回用户的信息,格式和user\_info\/get\_self相同。如果有错误,错误信息保存在error中。



\subsubsection{绑定手机号码}
\label{hash_algorithm}

\begin{table}[H]
   \begin{center}
\begin{tabular}{|c|c|c|p{12cm}|}
\hline
POST & \multicolumn{3}{|c|}{/phone/bind} \\
\hline\hline
 \  参数  & 类型 & 必填 &  说明  \\
\hline
 phone  & 字符串 & 必须 &  用户的手机号码\\
\hline
 code  & 字符串 & 必须 &  验证码\\
\hline
\end{tabular}
   \end{center}
\end{table}

绑定手机号码要先调用phone\/init获取验证码。
调用成功返回{bind: true},如果有错误,错误信息保存在error中。

[2013-8-15] 取消参数password, 绑定手机号码时可以不用输入手机号码。


\subsubsection{绑定手机后设置密码}
\label{hash_algorithm}

\begin{table}[H]
   \begin{center}
\begin{tabular}{|c|c|c|p{12cm}|}
\hline
POST & \multicolumn{3}{|c|}{/phone/set\_password} \\
\hline\hline
 \  参数  & 类型 & 必填 &  说明  \\
\hline
 password  & 字符串 & 必须 &  用户设置的密码\\
 \hline
 phone  & 字符串 & 必须 &  用户的手机号码\\
\hline
\end{tabular}
   \end{center}
\end{table}
如果已经设置了密码后要修改密码,不能调用本接口。


\subsubsection{解除手机号码绑定}
\label{hash_algorithm}

\begin{table}[H]
   \begin{center}
\begin{tabular}{|c|c|c|p{12cm}|}
\hline
POST & \multicolumn{3}{|c|}{/phone/unbind} \\
\hline\hline
 \  参数  & 类型 & 必填 &  说明  \\
\hline
 phone  & 字符串 & 必须 &  用户的手机号码\\
\hline
\end{tabular}
   \end{center}
\end{table}
至少有一种以上的绑定关系才能解绑。


\subsubsection{上传手机通讯录}
\label{hash_algorithm}

\begin{table}[H]
   \begin{center}
\begin{tabular}{|c|c|c|p{12cm}|}
\hline
POST & \multicolumn{3}{|c|}{/phone/upload\_address\_list} \\
\hline\hline
 \  参数  & 类型 & 必填 &  说明  \\
\hline
 phone  & 字符串 & 必须 &  用户的手机号码\\
 \hline
 list  & json字符串 & 必须 &  通讯录\\
\hline
\end{tabular}
   \end{center}
\end{table}


\begin{verbatim}
list是一个json字符串表示的数组,其中每一个条目是一个hash, 格式如下:
[{"id":323,"number":"(0571)28996083","name":"***"},...]

返回值为“{imported: 导入的通讯录的条数}”。
\end{verbatim}




\subsection{退出}

\begin{table}[H]
   \begin{center}
\begin{tabular}{|c|c|c|p{12cm}|}
\hline
POST & \multicolumn{3}{|c|}{/oauth2/logout} \\
\hline\hline
 \  参数  & 类型 & 必填 &  说明  \\
\hline
    pushtoken  & 浮点数 & 可选 &  当前设备的push消息token\\
\hline
\end{tabular}
   \end{center}
\end{table}

当用户主动退出时,调用本接口并断开XMPP的连接。
[2012-12-30] 新增pushtoken参数。



\subsection{登录与绑定的状态判断}
目前,脸脸帐号登录相关方有四个:脸脸web服务器、脸脸xmpp服务器,新浪微博,QQ。

\begin{enumerate}

\item 当访问web接口时,如果返回的error信息是"not login",此时代表要登录脸脸web服务器。但是脸脸不提供独立的帐户体系,要登陆脸脸Web服务器,必须登陆新浪微博或者QQ或者手机号码中的一个。当脸脸登录时,如果客户端缓存有未过期的新浪微博或者QQ认证信息,要丢弃。Web服务器认证信息保存在\_session\_id中。\_session\_id默认没有过期时间。
\item 如果脸脸\_session\_id包含登录信息(没返回not login):
\begin{enumerate}

\item 如果新浪微博或者QQ中的一个处于有效登录状态(wb\_token/qq\_token处在有效期内),需要使用另外一个服务时:
\begin{enumerate}
\item 如果以前绑定过另外一个的帐号,使用bind=2参数登录。
\item 如果没有,此时要绑定帐号。要使用bind=1参数登录。此时会绑定帐号并登录。
\end{enumerate}

\item 如果新浪微博和QQ都没有登录,此时要返回到登录界面。
\item 绑定操作(bind=1)只需要执行一次,
\item 当用户的个人信息中有wb\_uid,代表绑定过微博;有qq\_openid,代表绑定过QQ。
\item 当用户的个人信息中有wb\_token,代表登录了微博;有qq\_token,代表登录了QQ。
\item 绑定操作可以取消。而对于登录操作,如果新浪微博或者QQ都没有登录,则不能取消。
\end{enumerate}

\item 关于\_session\_id:
\begin{enumerate}
\item 首次安装应用调用init接口时,生成\_session\_id。以后所有的调用包括后来的init调用也要带上\_session\_id。
\item 即使没有登录也分配\_session\_id。
\item 调用logout时会变更\_session\_id。
\item 任何调用,如果没带\_session\_id,会生成新的\_session\_id。这样用户原来的所有状态信息会丢失,这是不建议的。

\end{enumerate}

\item 总的来说就是:没有\_session\_id通过init调用获得\_session\_id;每次调用总是带上\_session\_id,logout时更新\_session\_id。


\item xmpp服务器是单独登录的。每次应用打开的时候登录,退出或切换到后台时退出。


\end{enumerate}


\subsection{关于首次登录的判断}
以前,脸脸判断用户是否首次登录是看TA是否有头像。如果没有头像,则走注册流程。

现在更改为:每种登录接口(sina xauth/sso, qq)都会在用户首次登录时返回字段:{newuser:1}。当有newuser时,走注册流程。
