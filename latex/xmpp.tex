\section{XMPP协议接口}


\subsection{与Openfire聊天服务器接口}
脸脸网的XMPP服务器采用的是Openfire。

\begin{enumerate}
\item 脸脸网的用户id加上"@dface.cn"就是openfire的jid,两个系统的密码一致。登录xmpp时的resource使用机器特定的字符串,以区分一个帐号在多台设备上登录。
\item 商家的id加上"@c.dface.cn"就是现场聊天室的jid。
\item 脸脸的用户名就是openfire的用户名,且也就是聊天时的用户名。Xmpp协议允许用户在加入聊天室时取一个名字,脸脸的聊天室应该忽略该名字。脸脸用户加入聊天室的时候,其jid格式为"roomid@c.dface.cn/uid"。
\item 脸脸中的关注和粉丝关系是单向的,和xmpp中的好友roster关系无关。
\item 只要用户登录,其xmpp协议中的presence就是在线,没有其它状态。
\item 脸脸中的隐身也和xmpp中的隐身状态无关。
\item xmpp中的任意两个用户可以发送消息,不管对方是否隐身,只要对方未设置黑名单即可。如果对方未登录,那么就是离线消息。
\end{enumerate}


\subsection{脸脸中的聊天用户的三种类型}

\begin{enumerate}
\item管理员,其jid固定为"502e6303421aa918ba000001@dface.cn";
\item商家,其jid为's+商家id@dface.cn';
\item个人,其jid为'个人id@dface.cn'。
\end{enumerate}
如果一个jid以字母s开头,那么它一定是商家。


\subsection{摇一摇及其消息格式}
目前,用户在聊天室摇一摇时,客户端发送“用户名 摇了摇手机和大家say hello~”给服务器。这导致要分析摇一摇数据很困难,所以要为摇一摇定义特殊的消息格式。

\begin{verbatim}
摇一摇的消息格式为:
 [摇一摇:$name]

其它客户端收到摇一摇的消息后,再将其转换为文字描述。
\end{verbatim}



\subsection{个人之间聊天的消息发送状态确认}
XMPP协议的消息发送状态确认参考规范“\href{http://xmpp.org/extensions/xep-0022.html}{XEP-0022: Message Events}”。其定义了四种消息事件:Offline、Delivered、Displayed、Composing。Dface只需要其中的两种。

\begin{verbatim}
当接收端收到消息时,发送delivered确认消息,例如:
<message id="Kk98S-16" to="s6@dface.cn/ylt">
<x xmlns="jabber:x:event"><delivered/><id>purplea5e6669c</id></x>
</message>

当接收端展示消息时,发送displayed确认消息,例如:
<message id="Kk98S-17" to="s6@dface.cn/ylt">
<x xmlns="jabber:x:event"><displayed/><id>purplea5e6669c</id></x>
</message>

发送端根据从接收端获得的状态通知更改单条消息的发送状态。

只有类型是message,且body不为空的消息需要确认状态。

当本地无法发送消息时(比如无网络、或者连接不上xmpp服务器),消息的状态为“发送失败”。发送失败的消息可以再次发送。

\end{verbatim}

\subsection{消息提醒}
\begin{enumerate}
\item 当不在现场的界面收到现场聊天室的消息时,在现场按钮旁边加红点提醒;
\item 当收到发送人是"sphoto@dface.cn"发过来的单聊消息时,表明是有照片提醒消息。这时在我的照片等地方加红点提醒。
\item 状态确认类消息不需要提醒。
\item 其它个人之间聊天的消息以数字计数的方式提醒。
\end{enumerate}




\section{在聊天室发图}
聊天发图主要流程是:客户端选择(拍摄)一张照片,通过http上传到服务器。上传完成后在xmpp里发送一个特定格式的消息。其它客户端收到消息后获取图片显示并发送回执消息。

\subsection{上传图片}

\begin{table}[H]
   \begin{center}
\begin{tabular}{|c|c|c|p{12cm}|}
\hline
POST & \multicolumn{3}{|c|}{/photos/create} \\
\hline\hline
 \  参数  & 类型 & 必填 &  说明  \\
\hline
 photo[img]  & file & 必须 &  头像文件\\
 \hline
 photo[room]  & 字符串 & 必须  &  聊天室的id\\
 \hline
 photo[weibo]  & 0/1 & 必须  &  是否同步发送到微博\\
  \hline
 photo[qq]  & 0/1 & 可选  &  是否同步发送到QQ空间\\
   \hline
 photo[wx]  & 0/1/2/3 & 可选  &  分享到微信: 1个人,2朋友圈, 3都分享了\\
   \hline
 photo[desc]  & 字符串 & 可选  &  图像的说明文字\\
 \hline
  photo[t]  & 整数 & 可选 &  图片类型:1拍照;2选自相册\\
 \hline
 wbtoken  & 字符串 & 可选  &  如果分享到微博,该用户的微博token\\
  \hline
 qqtoken  & 字符串 & 可选  &  如果分享到QQ空间,该用户的QQtoken\\
 \hline
\end{tabular}
   \end{center}
\end{table}

注意:

\begin{enumerate}
\item 该请求必须是POST请求,enctype="multipart/form-data"。
\item 所上传文件的type必须是image/jpeg', 'image/gif' 或者'image/png'。
\item 图片文件要在客户端先压缩到640*640。
\item 图片名为logo;缩略图分logo\_thumb2为200*200的png。
\item 分享到新浪微博和QQ空间由服务器执行,客户端只需要传递标志位即可;而分享到微信则由客户端实现,然后报告服务器分享的结果。
\end{enumerate}

例子:

\begin{figure}[H]
\begin{verbatim}
访问:curl -F "photo[img]=@firefox.png;type=image/png" 
/photos

{"_id":"509a1ffcbe4b1982a4000002",
"weibo":false,"room":"1",
"user_id":"502e61bfbe4b1921da000005",
"img_tmp":"20121107-1646-42114-4251/111.jpg",
"logo":"/uploads/tmp/20121107-1646-42114-4251/111.jpg",
"logo_thumb2":null,"id":"509a1ffcbe4b1982a4000002"}

当返回的响应中包含img_tmp字段,且logo_thumb2为null时,说明此时图片上传到了服务器,但是还未启动后台处理。后台异步处理图片,包括生成缩略图以及上传到阿里云。

\end{verbatim}
\end{figure}


\subsection{发送消息}
当图片发送完成后,服务器端发送一个xmpp消息,其中的body内容格式为:

\begin{verbatim}
[img:$id]$desc

比如上面的图片发送成功后,发送消息"[img:502fd10abe4b19ed48000002]"。其中$desc是可选的图片说明文字。
\end{verbatim}

\subsection{在聊天室收到消息后获取图片}

接收端收到消息后,判断body的内容是否符合图片消息的格式。如果符合就调用下面的接口获取图片。

\begin{table}[H]
   \begin{center}
\begin{tabular}{|c|c|c|p{12cm}|}
\hline
GET & \multicolumn{3}{|c|}{/photos/show} \\
\hline\hline
 \  参数  & 类型 & 必填 &  说明  \\
  \hline
 id  & 字符串 & 必须 & 图片id\\
\hline
 size  & 整数 & 可选 &  目前0代表原图;2代表thumb2缩略图,大小为200*200。默认为2。\\ 
\hline
\end{tabular}
   \end{center}
\end{table}

聊天室里发图不需要发送已读回执消息。
在现场收到图片时,如果图片id以“faq”开头,不添加到照片墙上。18:02:24


\subsection{在聊天室收点击图片获取图片详细信息}

\begin{table}[H]
   \begin{center}
\begin{tabular}{|c|c|c|p{12cm}|}
\hline
GET & \multicolumn{3}{|c|}{/photos/detail} \\
\hline\hline
 \  参数  & 类型 & 必填 &  说明  \\
  \hline
 id  & 字符串 & 必须 & 图片id\\
\hline
\end{tabular}
   \end{center}
\end{table}



\section{照片墙}

\subsection{删除聊天室图片}

\begin{table}[H]
   \begin{center}
\begin{tabular}{|c|c|c|p{12cm}|}
\hline
POST & \multicolumn{3}{|c|}{/photos/delete} \\
\hline\hline
 \  参数  & 类型 & 必填 &  说明  \\
  \hline
 id  & 字符串 & 必须 & 图片id\\
\hline
\end{tabular}
   \end{center}
\end{table}
个人只能删除自己发布的照片。



\subsection{对聊天室图片点赞}

\begin{table}[H]
   \begin{center}
\begin{tabular}{|c|c|c|p{12cm}|}
\hline
POST & \multicolumn{3}{|c|}{/photos/like} \\
\hline\hline
 \  参数  & 类型 & 必填 &  说明  \\
  \hline
 id  & 字符串 & 必须 & 图片id\\
\hline
\end{tabular}
   \end{center}
\end{table}

\subsection{对聊天室图片取消赞}

\begin{table}[H]
   \begin{center}
\begin{tabular}{|c|c|c|p{12cm}|}
\hline
POST & \multicolumn{3}{|c|}{/photos/dislike} \\
\hline\hline
 \  参数  & 类型 & 必填 &  说明  \\
  \hline
 id  & 字符串 & 必须 & 图片id\\
\hline
\end{tabular}
   \end{center}
\end{table}
个人只能取消自己给的赞。


\subsection{对聊天室图片点评}

\begin{table}[H]
   \begin{center}
\begin{tabular}{|c|c|c|p{12cm}|}
\hline
POST & \multicolumn{3}{|c|}{/photos/comment} \\
\hline\hline
 \  参数  & 类型 & 必填 &  说明  \\
  \hline
 id  & 字符串 & 必须 & 图片id\\
   \hline
 text  & 字符串 & 必须 & 评论的内容\\
  \hline
 sid  & 字符串 & 可选 & 当前用户所在现场的id\\ 
\hline
\end{tabular}
   \end{center}
\end{table}

[2013-9-24] 新增sid,表示发表评论的用户所在现场的id。同时,如果评论者的sid和图片的商家id一致,那么会在聊天室中发布一条动态消息,消息id以COMMENT开头。评论消息的格式和普通图片消息一致。

\begin{figure}[H]
\begin{verbatim}
返回的数据结构是评论的数组:
 id: 评论人的id
 name: 评论人的名字
 txt:评论的内容
 t:时间
 rid:被回复者的id
 rname: 被回复者的名字
\end{verbatim}
\end{figure}

\subsection{对聊天室图片点评进行回复}

\begin{table}[H]
   \begin{center}
\begin{tabular}{|c|c|c|p{12cm}|}
\hline
POST & \multicolumn{3}{|c|}{/photos/recomment} \\
\hline\hline
 \  参数  & 类型 & 必填 &  说明  \\
  \hline
 id  & 字符串 & 必须 & 图片id\\
 \hline
 rid  & 字符串 & 必须 & 被回复者的id\\
   \hline
 text  & 字符串 & 必须 & 评论的内容\\
  \hline
 sid  & 字符串 & 可选 & 当前用户所在现场的id\\  
\hline
\end{tabular}
   \end{center}
\end{table}
这里的rid是被回复者的id,而发布回复者的id是从session中自动获取的。

这里的回复其实是针对人的,如果一个人在一张图片下发布了多个评论,那么这里的回复不针对特定的那条评论。

[2013-9-24] 新增sid,表示发表评论的用户所在现场的id。


\subsection{对聊天室图片删除点评}

\begin{table}[H]
   \begin{center}
\begin{tabular}{|c|c|c|p{12cm}|}
\hline
POST & \multicolumn{3}{|c|}{/photos/delcomment} \\
\hline\hline
 \  参数  & 类型 & 必填 &  说明  \\
  \hline
 id  & 字符串 & 必须 & 图片id\\
  \hline
 text  & 字符串 & 必须 & 评论的内容\\
\hline
\end{tabular}
   \end{center}
\end{table}
个人只能删除自己发的点评。

评论的删除是区分两种情况:删除/隐藏,评论的发布人可以删除,图片的发布人可以隐藏。
对第三方来说都是删除。

如果一个评论是我发的,同时图也是我发的,那么此时对评论操作就是删除,不需要隐藏。


\subsection{对聊天室图片隐藏点评}

\begin{table}[H]
   \begin{center}
\begin{tabular}{|c|c|c|p{12cm}|}
\hline
POST & \multicolumn{3}{|c|}{/photos/hidecomment} \\
\hline\hline
 \  参数  & 类型 & 必填 &  说明  \\
  \hline
 id  & 字符串 & 必须 & 图片id\\
  \hline
 uid  & 字符串 & 必须 & 评论的发布者id\\
   \hline
 text  & 字符串 & 必须 & 评论的内容\\
\hline
\end{tabular}
   \end{center}
\end{table}
只有图片的发布者才能隐藏他人给自己图片发的点评。调用本接口的必须是图片的发布人。
点评隐藏后只有点评的发布者一人能够看到。


\subsection{我的照片墙}
用户的照片墙由其本人在各个现场照片墙发布的照片组成。

\begin{table}[H]
   \begin{center}
\begin{tabular}{|c|c|c|p{12cm}|}
\hline
GET & \multicolumn{3}{|c|}{/photos/me} \\
\hline\hline
 \  参数  & 类型 & 必填 &  说明  \\
\hline
 page  & 整数 & 可选 & 分页,缺省值为1\\ 
 \hline
 pcount  & 整数 & 可选 & 每页的数量,缺省为20\\ 
\hline
\end{tabular}
   \end{center}
\end{table}

输出格式参见:\ref{photowall} 。唯一不同点在于:
商家照片墙中包含user\_name,指明该照片是谁发的。
用户照片墙中包含shop\_name,指明该照片是在哪儿拍摄的。


\subsection{我的评论}
获得当前登录用户的评论的照片。

\begin{table}[H]
   \begin{center}
\begin{tabular}{|c|c|c|p{12cm}|}
\hline
GET & \multicolumn{3}{|c|}{/photos/my\_comments} \\
\hline\hline
 \  参数  & 类型 & 必填 &  说明  \\
\hline
 page  & 整数 & 可选 & 分页,缺省值为1\\ 
 \hline
 pcount  & 整数 & 可选 & 每页的数量,缺省为20\\ 
\hline
\end{tabular}
   \end{center}
\end{table}

输出格式同上。


\subsection{查看其它用户的照片墙}
其它用户的照片墙由其该在各个现场照片墙发布并分享到微博/qq的照片组成。没有分享过的图片看成只在该现场才能看到的图片。

\begin{table}[H]
   \begin{center}
\begin{tabular}{|c|c|c|p{12cm}|}
\hline
GET & \multicolumn{3}{|c|}{/photos/users} \\
\hline\hline
 \  参数  & 类型 & 必填 &  说明  \\
 \hline
 uid  & 整数 & 必须 & 被查看用户的id\\ 
\hline
 page  & 整数 & 可选 & 分页,缺省值为1\\ 
 \hline
 pcount  & 整数 & 可选 & 每页的数量,缺省为20\\ 
\hline
\end{tabular}
   \end{center}
\end{table}

输出格式同上。



\subsection{其它用户分享到微信}
看到别人发的照片/问答,分享到自己的微信朋友圈或者微信好友,在客户端完成分享后,调用本接口。

\begin{table}[H]
   \begin{center}
\begin{tabular}{|c|c|c|p{12cm}|}
\hline
GET & \multicolumn{3}{|c|}{/photos/re\_share} \\
\hline\hline
 \  参数  & 类型 & 必填 &  说明  \\
 \hline
 user\_id  & 字符串 & 必须 & 分享者的用户id\\ 
  \hline
 photo\_id  & 字符串 & 必须 & 被分享的图片id\\ 
 \hline
 weibo  & 0/1 & 可选  &  是否同步发送到微博\\
  \hline
 qq  & 0/1 & 可选  &  是否同步发送到QQ空间\\
  \hline
 wx  & 0/1/2/3 & 可选  &  分享到微信: 1个人,2朋友圈, 3都分享了\\
\hline
\end{tabular}
   \end{center}
\end{table}

\begin{verbatim}
本接口主要用于统计用户和照片的分享数量。分享到微信时,采用url类型,url地址为“http://www.dface.cn/web_photo/show?id=图片id&uid=用户id”。
\end{verbatim}



\section{个人聊天时发图}

\subsection{个人聊天时上传图片}

\begin{table}[H]
   \begin{center}
\begin{tabular}{|c|c|c|p{12cm}|}
\hline
POST & \multicolumn{3}{|c|}{/photo2s/create} \\
\hline\hline
 \  参数  & 类型 & 必填 &  说明  \\
\hline
 photo[img]  & file & 必须 &  头像文件\\
 \hline
 photo[to\_uid]  & 字符串 & 必须 &  接收人的id\\
\hline
  photo[t]  & 整数 & 可选 &  图片类型:1拍照;2选自相册\\
 \hline
\end{tabular}
   \end{center}
\end{table}


\subsection{发送消息}
当图片发送完成后,服务器端发送一个xmpp消息,其中的body内容格式为:

\begin{verbatim}
[img:$id]

\end{verbatim}

\subsection{在个人聊天时收到消息后获取图片}
\begin{table}[H]
   \begin{center}
\begin{tabular}{|c|c|c|p{12cm}|}
\hline
GET & \multicolumn{3}{|c|}{/photo2s/show} \\
\hline\hline
 \  参数  & 类型 & 必填 &  说明  \\
  \hline
 id  & 字符串 & 必须 & 图片id\\
\hline
 size  & 整数 & 可选 &  目前0代表原图;2代表thumb2缩略图,大小为200*200。默认为2。\\ 
\hline
\end{tabular}
   \end{center}
\end{table}

注意:聊天时发图,取消的100*100大小的缩略图。[2012-11-7]


\subsection{消息状态}
和普通的文字消息一样,接收者收到图片后发送状态更新消息(收到/已读)给发送者。发送者将状态显示在图片旁边。




\section{优惠券}

\subsection{商家推送优惠券}
当用户使用脸脸时,比如签到/摇一摇等,可以收到商家推送过来的优惠券。优惠券也是一个xmpp消息,其中的body内容格式为:

\begin{verbatim}
[优惠券:$name:$shop:$id:$time]

 其中,name是优惠券的名称、shop是商家的名称,id是该优惠券的id,time是该张优惠券的下发时间。
 
 同一张id的优惠券,如果time不同的话,分开显示和使用。
 
 所有优惠券都统一显示在会话中的“优惠券”文件夹中。优惠券的消息状态和其他消息同样处理。
\end{verbatim}

\subsection{优惠券展示}
优惠券以图片的方式展示,由服务器端生成该图片。当用户进入优惠券文件夹查看优惠券时,调用下面的接口获得优惠券的图片。

\begin{table}[H]
   \begin{center}
\begin{tabular}{|c|c|c|p{12cm}|}
\hline
GET & \multicolumn{3}{|c|}{/coupons/img} \\
\hline\hline
 \  参数  & 类型 & 必填 &  说明  \\
\hline
 \  id  & 字符串 & 必填 &  该优惠券的id  \\
\hline
 size  & 整数 & 可选 &  目前0代表原图(580, 224);1代表缩略图,大小为(290,112)。默认为1。\\ 
\hline
\end{tabular}
   \end{center}
\end{table}



\subsection{优惠券使用}
目前,所有优惠券都只支持单次使用,使用后即过期。当用户确认使用优惠券时,调用如下的接口:
\begin{table}[H]
   \begin{center}
\begin{tabular}{|c|c|c|p{12cm}|}
\hline
POST & \multicolumn{3}{|c|}{/coupons/use} \\
\hline\hline
 \  参数  & 类型 & 必填 &  说明  \\
\hline
 \  id  & 字符串 & 必填 &  该优惠券的id  \\
\hline
\end{tabular}
   \end{center}
\end{table}

优惠券的使用要保证只能点击一次,使用后就灰掉,不需要等待服务器端的返回结果。如果网络不通或者服务器端失败,要以一定的策略重试。





\section{XMPP服务器的Web接口}




\subsection{获得聊天室的消息历史记录}

\begin{table}[H]
   \begin{center}
\begin{tabular}{|c|c|c|p{12cm}|}
\hline
GET & \multicolumn{3}{|c|}{http://xmpp\_ip:5280/api/gchat} \\
\hline\hline
 \  参数  & 类型 & 必填 &  说明  \\
\hline
 \  room  & 字符串 & 必填 &  聊天室的房间id  \\
\hline
\end{tabular}
   \end{center}
\end{table}
返回值的数据结构:
[发送者ID,消息内容,以秒表示的时间,消息ID]


\subsection{获得聊天室的消息历史记录2}

\begin{table}[H]
   \begin{center}
\begin{tabular}{|c|c|c|p{12cm}|}
\hline
GET & \multicolumn{3}{|c|}{http://xmpp\_ip:5280/api/gchat2} \\
\hline\hline
 \  参数  & 类型 & 必填 &  说明  \\
\hline
 \  room  & 字符串 & 必填 &  聊天室的房间id  \\
\hline
 \  skip  & 数字 & 必填 &  跳过多少条  \\
\hline
 \  count  & 数字 & 必填 &  取多少条  \\  
\hline
\end{tabular}
   \end{center}
\end{table}
注意:skip=0时,消息历史用ejabberd的内存中获取;skip大于0时,从数据库取。

返回值的数据结构:
[发送者ID,消息内容,以秒表示的时间,消息ID]




\subsection{获得个人聊天的历史记录}

\begin{table}[H]
   \begin{center}
\begin{tabular}{|c|c|c|p{12cm}|}
\hline
GET & \multicolumn{3}{|c|}{http://xmpp\_ip:5280/api/chat} \\
\hline\hline
 \  参数  & 类型 & 必填 &  说明  \\
\hline
 \  uid  & 字符串 & 必填 &  用户的id  \\
\hline
\end{tabular}
   \end{center}
\end{table}
返回值的数据结构:
[接受者ID,消息内容,以秒表示的时间,消息ID]


\subsection{获得两人之间的聊天历史记录}

\begin{table}[H]
   \begin{center}
\begin{tabular}{|c|c|c|p{12cm}|}
\hline
GET & \multicolumn{3}{|c|}{http://xmpp\_ip:5280/api/chat2} \\
\hline\hline
 \  参数  & 类型 & 必填 &  说明  \\
\hline
 \  uid1  & 字符串 & 必填 &  用户1的id  \\
\hline
 \  uid2  & 字符串 & 必填 &  用户2的id  \\
\hline
\end{tabular}
   \end{center}
\end{table}

返回值的数据结构:
[发送者ID, 接受者ID,消息内容,以秒表示的时间,消息ID]


\subsection{其它内部接口}
仅用于服务器端的后台内部调用。

\subsubsection{以给定用户身份给指定的聊天室发送消息}

\begin{table}[H]
   \begin{center}
\begin{tabular}{|c|c|c|p{12cm}|}
\hline
POST & \multicolumn{3}{|c|}{http://xmpp\_ip:5280/api/room} \\
\hline\hline
 \  参数  & 类型 & 必填 &  说明  \\
\hline
 \  roomid  & 字符串 & 必填 &  聊天室的房间id  \\
\hline
 \  message  & 字符串 & 必填 &  需要发送的消息  \\
\hline
 \  uid  & 字符串 & 必填 &  发送此消息的用户id  \\
\hline
\end{tabular}
   \end{center}
\end{table}
这条消息会发送给聊天室的所有人,比如用户进入聊天室时的打招呼。
如果是给聊天室的某个人发消息,比如优惠券中奖/公告消息,则通过rest接口发送groupchat类型的消息。

\subsubsection{用户屏蔽通讯}

\begin{table}[H]
   \begin{center}
\begin{tabular}{|c|c|c|p{12cm}|}
\hline
POST & \multicolumn{3}{|c|}{http://xmpp\_ip:5280/api/block} \\
\hline\hline
 \  参数  & 类型 & 必填 &  说明  \\
\hline
 \  uid  & 字符串 & 必填 &  发起屏蔽请求的用户的id  \\
\hline
 \  bid  & 字符串 & 必填 &  被屏蔽的用户的id  \\
\hline
\end{tabular}
   \end{center}
\end{table}

\subsubsection{用户解除屏蔽}

\begin{table}[H]
   \begin{center}
\begin{tabular}{|c|c|c|p{12cm}|}
\hline
POST & \multicolumn{3}{|c|}{http://xmpp\_ip:5280/api/unblock} \\
\hline\hline
 \  参数  & 类型 & 必填 &  说明  \\
\hline
 \  uid  & 字符串 & 必填 &  发起解除屏蔽请求的用户的id  \\
\hline
 \  bid  & 字符串 & 必填 &  被屏蔽的用户的id  \\
\hline
\end{tabular}
   \end{center}
\end{table}


\subsubsection{获取用户屏蔽列表}

\begin{table}[H]
   \begin{center}
\begin{tabular}{|c|c|c|p{12cm}|}
\hline
POST & \multicolumn{3}{|c|}{http://xmpp\_ip:5280/api/blocklist} \\
\hline\hline
 \  参数  & 类型 & 必填 &  说明  \\
\hline
 \  uid  & 字符串 & 必填 &  发起解除屏蔽请求的用户的id  \\
\hline
\end{tabular}
   \end{center}
\end{table}


\subsubsection{杀掉用户会话}

\begin{table}[H]
   \begin{center}
\begin{tabular}{|c|c|c|p{12cm}|}
\hline
POST & \multicolumn{3}{|c|}{http://xmpp\_ip:5280/api/kill} \\
\hline\hline
 \  参数  & 类型 & 必填 &  说明  \\
\hline
 \  user  & 字符串 & 必填 &  被封杀的用户的id  \\
\hline
\end{tabular}
   \end{center}
\end{table}


\subsubsection{禁止用户在聊天室发言}

\begin{table}[H]
   \begin{center}
\begin{tabular}{|c|c|c|p{12cm}|}
\hline
POST & \multicolumn{3}{|c|}{http://xmpp\_ip:5280/api/room\_ban} \\
\hline\hline
 \  参数  & 类型 & 必填 &  说明  \\
\hline
 \  roomid  & 字符串 & 必填 & 聊天室id \\
 \hline
 \  uid  & 字符串 & 必填 &  被封杀的用户的id  \\
\hline
\end{tabular}
   \end{center}
\end{table}


\subsubsection{解除禁止用户在聊天室发言}

\begin{table}[H]
   \begin{center}
\begin{tabular}{|c|c|c|p{12cm}|}
\hline
POST & \multicolumn{3}{|c|}{http://xmpp\_ip:5280/api/room\_unban} \\
\hline\hline
 \  参数  & 类型 & 必填 &  说明  \\
\hline
 \  roomid  & 字符串 & 必填 & 聊天室id \\
 \hline
 \  uid  & 字符串 & 必填 &  被封杀的用户的id  \\
\hline
\end{tabular}
   \end{center}
\end{table}



\subsubsection{发送任意XMPP消息}

\begin{table}[H]
   \begin{center}
\begin{tabular}{|c|c|c|p{12cm}|}
\hline
POST & \multicolumn{3}{|c|}{http://xmpp\_ip:5280/rest} \\
\hline\hline
 \  参数  & 类型 & 必填 &  说明  \\
\hline
\end{tabular}
   \end{center}
\end{table}

通过POST提交任意类型的XMPP文本消息。



\section{关于Push消息}

\subsection{登录Xmpp服务器时,报告设备的Push消息token}
登录Xmpp服务器,发送的密码格式变更为:

\begin{verbatim}
“用户密码”+《状态码》+token

状态码为:1代表ios开发设备,2代表实际的发布设备。
token是一个64位的字符串。

比如某用户的密码是“c53b2f16a24c61d9”,
token是“ea6e4f05b48f4a057816956e567c6feacf14ee28cbbbab67993e263c3dfa2c27”,
目前是进行开发测试,那么登录Xmpp服务器时的密码为:
“c53b2f16a24c61d91ea6e4f05b48f4a057816956e567c6feacf14ee28cbbbab67993e263c3dfa2c27”。
\end{verbatim}

服务器端解析出状态码和token并保存。


\subsection{退出时,报告设备的Push消息token}

用户退出的API接口新增pushtoken参数,以取消Push消息token绑定。

\subsection{Push消息的发送}
当有离线消息产生时,xmpp服务器获得用户的状态码以决定给那个Apple Push服务器发消息,以及获得和该用户绑定的token。一个用户只能绑定一个token。
