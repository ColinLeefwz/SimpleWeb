

\section{用户基本信息}


\subsection{获得用户基本信息}

\begin{table}[H]
   \begin{center}
\begin{tabular}{|c|c|c|p{12cm}|}
\hline
GET & \multicolumn{3}{|c|}{/user\_info/basic} \\
\hline\hline
 \  参数  & 类型 & 必填 &  说明  \\
\hline
 id  & 整数 & 必须 &  用户id\\
\hline
\end{tabular}
   \end{center}
\end{table}



例子:

\begin{figure}[H]
\begin{verbatim}
访问:/user_info/basic?id=50bec2c1c90d8bd12f000086
返回值:

{"name":"amanda林","signature":"","wb_v":false,"wb_vs":"","gender":2,"birthday":"1986-2-15","job":"","jobtype":5,"pcount":5,"wb_uid":"1735512691","id":"50bec2c1c90d8bd12f000086","logo":"http://oss.aliyuncs.com/logo/50fe370ec90d8b173a00023c/0.jpg","logo_thumb":"http://oss.aliyuncs.com/logo/50fe370ec90d8b173a00023c/t1_0.jpg","logo_thumb2":"http://oss.aliyuncs.com/logo/50fe370ec90d8b173a00023c/t2_0.jpg"}

\end{verbatim}
\end{figure}



\subsection{获得用户和当前登录用户的关系}

\begin{table}[H]
   \begin{center}
\begin{tabular}{|c|c|c|p{12cm}|}
\hline
GET & \multicolumn{3}{|c|}{/user\_info/relation} \\
\hline\hline
 \  参数  & 类型 & 必填 &  说明  \\
\hline
 id  & 整数 & 必须 &  用户id\\
\hline
\end{tabular}
   \end{center}
\end{table}


获得信息里有两个关系字段:friend和follower。这里的关系都是针对我的,这里的我就是当前登录用户。如果我关注了该用户,那么该用户就是我的friend; 如果该用户关注了我,那么他就是我的follower。

例子:

\begin{figure}[H]
\begin{verbatim}
访问:/user_info/friend?id=502e6303421aa918ba000001
返回值:
{"id":"502e6303421aa918ba000001","friend":true,"follower":true}
\end{verbatim}
\end{figure}



\subsection{获得用户最后出现的位置信息}

\begin{table}[H]
   \begin{center}
\begin{tabular}{|c|c|c|p{12cm}|}
\hline
GET & \multicolumn{3}{|c|}{/user\_info/last\_loc} \\
\hline\hline
 \  参数  & 类型 & 必填 &  说明  \\
\hline
 id  & 整数 & 必须 &  用户id\\
\hline
\end{tabular}
   \end{center}
\end{table}

例子:

\begin{figure}[H]
\begin{verbatim}
访问:/user_info/last_loc?id=1
返回值:

{
"time":1天以前,
"last": "1天以前 顺旺基(益乐路店)" 
"id":1
}

last字段,表示用户最后在脸脸上的时间和地点,一个以空格分割的字符串。
\end{verbatim}
\end{figure}


\subsection{获得用户的地主地点列表}

\begin{table}[H]
   \begin{center}
\begin{tabular}{|c|c|c|p{12cm}|}
\hline
GET & \multicolumn{3}{|c|}{/user\_info/lords} \\
\hline\hline
 \  参数  & 类型 & 必填 &  说明  \\
\hline
 id  & 整数 & 必须 &  用户id\\
\hline
\end{tabular}
   \end{center}
\end{table}

例子:

\begin{figure}[H]
\begin{verbatim}
访问:/user_info/lords?id=502e6303421aa918ba000001
返回值:

[{"name":"千粥汇华星路店","lo":[30.281387,120.11969],"t":4,"lat":30.281387,"lng":120.11969,
"address":"","phone":"","id":21829233,"user":0,"male":0,"female":0},
{"name":"浦东国际人才城","lo":[31.189001,121.586937],"t":"10","lat":31.189001,"lng":121.586937,
"address":"","phone":"","id":21830801,"user":0,"male":0,"female":0}]

\end{verbatim}
\end{figure}


\subsection{获得当前登录用户基本信息}

\begin{table}[H]
   \begin{center}
\begin{tabular}{|c|c|c|p{12cm}|}
\hline
GET & \multicolumn{3}{|c|}{/user\_info/get\_self} \\
\hline\hline
 \  参数  & 类型 & 必填 &  说明  \\
\hline
\end{tabular}
   \end{center}
\end{table}

例子:

\begin{figure}[H]
\begin{verbatim}
访问:/user_info/get_self
返回值:

{
"name":null,
wb_uid:1884834632
"gender":null,
"birthday":null,
"invisible":0,
"logo":"/system/imgs/1/original/clojure.png?1339398140",
"logo_thumb":"/system/imgs/1/thumb/clojure.png?1339398140"
"password":"c84dad462d5b7282",
"id":1
}

[2013-03-12]  新增
微博登陆的话:增加wb_token,wb_expire
qq登陆的话:增加qq_token,qq_expire

[2013-06-28]  新增wb_hidden,  1代表对他人隐藏自己的微博,2代表解除微博绑定.

\end{verbatim}
\end{figure}




\subsection{设置当前登录用户基本信息}

\begin{table}[H]
   \begin{center}
\begin{tabular}{|c|c|c|p{12cm}|}
\hline
POST & \multicolumn{3}{|c|}{/user\_info/set} \\
\hline\hline
 \  参数  & 类型 & 必填 &  说明  \\
\hline
 name  & 字符串 & 选填 &  用户的名字. 长度小于64\\
\hline
 gender  & 数字 & 选填 &  用户的性别,未设置0男1女2\\
\hline
 birthday  & 字符串 & 选填 &  用户的生日,格式如2012-06-01\\
 \hline
 signature  & 字符串 & 选填 &  签名档. 长度小于255\\
 \hline
 job  & 字符串 & 选填 &  职业\\
 \hline
 jobtype  & 整数 & 选填 &  职业类别\\
 \hline
 hobby  & 字符串 & 选填 &  爱好. 长度小于255\\
 \hline
 invisible  & 数字 & 选填 &  0:不隐身,1:对黑名单隐身,2:陌生人隐身, 3:全部隐身\\
 \hline
 wb\_hidden  & 数字 & 选填 &  0:不隐藏微博, 1代表对他人隐藏自己的微博\\ 
\hline
 no\_push  & 数字 & 选填 &  0:默认可以push, 1代表不接收push消息提醒。只有android客户端需要设置这个参数\\ 
\hline
\end{tabular}
   \end{center}
\end{table}

注意:该请求必须是POST请求。

用户初次登录时获取新浪微博的名称/性别/出生日期等信息并提交到服务端。以后更新时可以只更新其中的某个字段。

隐身的效果是:
1、在商家的用户列表中不在出现该用户
2、不更新自己的位置信息
3、用户进入商家时,不发打招呼的信息。

例子:

\begin{figure}[H]
\begin{verbatim}
访问:curl -b "_session_id=ead9ac4f6291c55bb467ad4138eca2ed" 
        -F "name=newname" /user_info/set

返回值:

{
"name":newname,
"gender":null,
"birthday":null,
"logo":"/system/imgs/1/original/clojure.png?1339398140",
"logo_thumb":"/system/imgs/1/thumb/clojure.png?1339398140"
"id":1
}

\end{verbatim}
\end{figure}


\subsection{设置其它用户的备注名}

\begin{table}[H]
   \begin{center}
\begin{tabular}{|c|c|c|p{12cm}|}
\hline
POST & \multicolumn{3}{|c|}{/user\_info/set\_comment\_name} \\
\hline\hline
 \  参数  & 类型 & 必填 &  说明  \\
\hline
 id  & 字符串 & 必填 &  被设置备注名的用户的id\\
\hline
 name  & 字符串 & 必填 &  用户的名字. 长度小于64\\
\hline
\end{tabular}
   \end{center}
\end{table}
当前登录用户给其它用户设置备注名,方便记忆。

\begin{figure}[H]
\begin{verbatim}
返回值:{"success" : 1}
\end{verbatim}
\end{figure}


\subsection{获得当前登录用户设置的所有备注名}

\begin{table}[H]
   \begin{center}
\begin{tabular}{|c|c|c|p{12cm}|}
\hline
GET & \multicolumn{3}{|c|}{/user\_info/get\_comment\_names} \\
\hline\hline
 \  参数  & 类型 & 必填 &  说明  \\
\hline
 user\_id  & 字符串 & 必填 &  当前登录用户的id\\
\hline
\end{tabular}
   \end{center}
\end{table}
当前登录用户给其它用户设置备注名,方便记忆。

\begin{figure}[H]
\begin{verbatim}
访问:/user_info/get_comment_names?user_id=502e6303421aa918ba000005
返回值:
[{"id":"502e6303421aa918ba000001","s":"yxy"}]
id是被设置备注名的用户的id,s是备注名。
\end{verbatim}
\end{figure}



\section{用户相册与头像}
用户相册中的第一张图片就是头像。

\subsection{当前登录用户上传图片}

\begin{table}[H]
   \begin{center}
\begin{tabular}{|c|c|c|p{12cm}|}
\hline
POST & \multicolumn{3}{|c|}{/user\_logos/create} \\
\hline\hline
 \  参数  & 类型 & 必填 &  说明  \\
\hline
 user\_logo[img]  & file & 必须 &  头像文件\\
 \hline
 user\_logo[t]  & 整数 & 可选 &  图片类型:1拍照;2选自相册\\
\hline
\end{tabular}
   \end{center}
\end{table}

注意:

\begin{enumerate}
\item 该请求必须是POST请求,enctype="multipart/form-data"。
\item 所上传文件的type必须是image/jpeg', 'image/gif' 或者'image/png'。
\item 图片文件要在客户端先压缩到640*640。
\item 图片名为logo;缩略图分为两种大小,logo\_thumb为100*100的png,logo\_thumb2为200*200的png。
\end{enumerate}

例子:

\begin{figure}[H]
\begin{verbatim}
访问:curl -F "user_logo[img]=@firefox.png;type=image/png" 
/user_logos


{"logo_thumb2":null, "logo_thumb1":null
,"id":6,
"logo":"",
"user_id":3,
"img_tmp":"/system/imgs/6/thumb/csky2.png?1340779488"}

\end{verbatim}

用户上传图片更改为异步操作。图片上传完成后返回图片的id,但是此时图片还没有实际的阿里云链接。客户端上传完成后,可以用下面的“根据图片id获得用户相册里的图片”接口尝试获得实际的上传图片和缩略图。

\end{figure}


\subsection{根据图片id获得用户相册里的图片}
\begin{table}[H]
   \begin{center}
\begin{tabular}{|c|c|c|p{12cm}|}
\hline
GET & \multicolumn{3}{|c|}{/user\_logos/show} \\
\hline\hline
 \  参数  & 类型 & 必填 &  说明  \\
  \hline
 id  & 字符串 & 必须 & 图片id\\
\hline
 size  & 整数 & 可选 &  目前0代表原图;1代表thumb1缩略图,大小为100*100;2代表thumb2缩略图,大小为200*200。默认为1。\\ 
\hline
\end{tabular}
   \end{center}
\end{table}



\subsection{获得用户的头像}

\begin{table}[H]
   \begin{center}
\begin{tabular}{|c|c|c|p{12cm}|}
\hline
GET & \multicolumn{3}{|c|}{/user\_info/logo} \\
\hline\hline
 \  参数  & 类型 & 必填 &  说明  \\
\hline
 id  & 整数 & 必须 &  用户id\\
\hline
 size  & 整数 & 可选 &  目前0代表原图;1代表thumb缩略图,大小为100*100;2代表thumb2缩略图,大小为200*200。默认为1。\\ 
\hline
\end{tabular}
   \end{center}
\end{table}

该接口应该只有一种情况下被使用:在聊天室里有一个陌生的用户发了一条消息。此时根据他在聊天室的resource获得其用户id,然后调用本接口拿头像。


输出的Response Header中包含“Img\_url”头,这个头就是查看用户信息时显示的用户头像的路径。

例如:“Img\_url:/system/imgs/1/thumb2/1.png?1339649979”

注意:采用阿里云存储后,会输出302重定向,而不是直接给二进制流。


\subsection{获得用户的图片列表}

\begin{table}[H]
   \begin{center}
\begin{tabular}{|c|c|c|p{12cm}|}
\hline
GET & \multicolumn{3}{|c|}{/user\_info/photos} \\
\hline\hline
 \  参数  & 类型 & 必填 &  说明  \\
\hline
 id  & 整数 & 必须 &  用户id\\
\hline
\end{tabular}
   \end{center}
\end{table}


例子:

\begin{figure}[H]
\begin{verbatim}
访问:/user_info/photos?id=1
返回值:

[
{"logo_thumb2":"/system/imgs/2/thumb2/2.png.png?1340616951","updated_at":"2012-06-25T09:35:51Z",
"id":2,"logo":"/system/imgs/2/original/2.png.png?1340616951","img_file_size":81881,"user_id":1,"logo_thumb":"/system/imgs/2/thumb/2.png.png?1340616951"},
{"logo_thumb2":"/system/imgs/4/thumb2/2.png.png?1340617039","updated_at":"2012-06-25T09:37:20Z",
"id":4,"logo":"/system/imgs/4/original/2.png.png?1340617039","img_file_size":81881,"user_id":1,"logo_thumb":"/system/imgs/4/thumb/2.png.png?1340617039"}
]

图片列表中的第一张图片就是用户的头像。


\end{verbatim}
\end{figure}



\subsection{当前登录用户批量调整所有图片的位置}

\begin{table}[H]
   \begin{center}
\begin{tabular}{|c|c|c|p{12cm}|}
\hline
POST & \multicolumn{3}{|c|}{/user\_logos/change\_all\_position} \\
\hline\hline
 \  参数  & 类型 & 必填 &  说明  \\
\hline
 ids  & 逗号分割的整数 & 必须 &  所有图片的id按新的循序排列\\
\hline
\end{tabular}
   \end{center}
\end{table}

说明:

比如原有图片的循序是2468,新的循序是8642,此时需要调用本接口,传递ids=8,6,4,2。此时服务器会更新所有图片的排序,而客户端只需要发送一个请求。

例子:

\begin{figure}[H]
\begin{verbatim}
访问:curl -F "ids=3,1,4" 
/user_logos/position


\end{verbatim}
\end{figure}



\subsection{当前登录用户删除图片}

\begin{table}[H]
   \begin{center}
\begin{tabular}{|c|c|c|p{12cm}|}
\hline
POST & \multicolumn{3}{|c|}{/user\_logos/delete} \\
\hline\hline
 \  参数  & 类型 & 必填 &  说明  \\
\hline
 id  & 整数 & 必须 &  图片的id\\
\hline
\end{tabular}
   \end{center}
\end{table}

例子:

\begin{figure}[H]
\begin{verbatim}
访问:curl -F "id=2" 
/user_logos/delete

输出:
{"deleted":"2"}

\end{verbatim}
\end{figure}


\section{关注和粉丝}
\subsection{添加我关注的人}

\begin{table}[H]
   \begin{center}
\begin{tabular}{|c|c|c|p{12cm}|}
\hline
POST & \multicolumn{3}{|c|}{/follows/create} \\
\hline\hline
 \  参数  & 类型 & 必填 &  说明  \\
\hline
 user\_id  & 整数 & 必须 &  当前登录用户的id\\
\hline
 follow\_id  & 整数 & 必须 &  关注用户的id\\
\hline
\end{tabular}
   \end{center}
\end{table}

注意:该请求必须是POST请求。

例子:

\begin{figure}[H]
\begin{verbatim}
访问:curl -b "_session_id=f03bef9371c119b6fcecbeefdaaac1b2"
       -d "user_id=2&follow_id=1" /follows

返回值:

{
"id":1,
"user_id":2,
"follow_id":1
}

\end{verbatim}
\end{figure}



\subsection{删除我的关注的人}

\begin{table}[H]
   \begin{center}
\begin{tabular}{|c|c|c|p{12cm}|}
\hline
POST & \multicolumn{3}{|c|}{/follows/delete} \\
\hline\hline
 \  参数  & 类型 & 必填 &  说明  \\
\hline
 user\_id  & 整数 & 必须 &  当前登录用户的id\\
\hline
 follow\_id  & 整数 & 必须 &  关注用户的id\\
\hline
\end{tabular}
   \end{center}
\end{table}

注意:该请求必须是POST请求。

例子:

\begin{figure}[H]
\begin{verbatim}
访问:curl -b "_session_id=f03bef9371c119b6fcecbeefdaaac1b2"
       -d "user_id=2&follow_id=3" /follows/delete

返回值:

{
"deleted":3
}

\end{verbatim}
\end{figure}



\subsection{粉丝列表}

\begin{table}[H]
   \begin{center}
\begin{tabular}{|c|c|c|p{12cm}|}
\hline
GET & \multicolumn{3}{|c|}{/follow\_info/followers} \\
\hline\hline
 \  参数  & 类型 & 必填 &  说明  \\
\hline
 id  & 整数 & 必须 & 用户的id\\
   \hline
 page  & 整数 & 可选 & 分页,缺省值为1\\ 
 \hline
 pcount  & 整数 & 可选 & 每页的数量,缺省为20\\ 
    \hline
 name  & 字符串 & 可选 & 用户的名字\\ 
    \hline    
 hash  & 整数 & 可选 & 是否返回hash,缺省值为0\\ 
\hline

\end{tabular}
   \end{center}
\end{table}

例子:

\begin{figure}[H]
\begin{verbatim}
访问:/follow_info/followers?id=2

返回值:

[
{"count":1}
{"data":
{"id":1,
"name":"name23",
"wb_uid":1884834632,
"gender":0,
"friend":true,
"follower":true,
"birthday":null,
"last":"1 hour 浙江科技产业大厦"}
}
]

其中count是符合条件的粉丝的总数。

获得用户基本信息里有两个关系字段:friend和follower。这里的关系都是针对我的,这里的我就是给定ID的用户。如果我关注了该用户,那么该用户就是我的friend; 如果该用户关注了我,那么他就是我的follower。

这里返回的所有用户,其"follower"值一定为true。

[2013-2-20] 新增last,表示用户最后一次出现的时间和地点。


\end{verbatim}
\end{figure}


\subsection{关注列表}
本接口以后将不再支持。
\begin{table}[H]
   \begin{center}
\begin{tabular}{|c|c|c|p{12cm}|}
\hline
GET & \multicolumn{3}{|c|}{/follow\_info/friends} \\
\hline\hline
 \  参数  & 类型 & 必填 &  说明  \\
\hline
 id  & 整数 & 必须 &  用户的id\\
   \hline
 page  & 整数 & 可选 & 分页,缺省值为1\\ 
 \hline
 pcount  & 整数 & 可选 & 每页的数量,缺省为20\\ 
     \hline
 name  & 字符串 & 可选 & 用户的名字\\ 
     \hline
 hash  & 整数 & 可选 & 是否返回hash,缺省值为0\\ 
\hline

\end{tabular}
   \end{center}
\end{table}

返回的内容格式同粉丝列表一样

在1.5版本以前,本接口返回所有我关注的用户。
在1.5版本以后,本接口返回所有我关注的且对方没有关注我的用户。双向关注的则由新增的good\_friends接口提供。


\subsection{双向好友列表}
本接口以后将不再支持。
\begin{table}[H]
   \begin{center}
\begin{tabular}{|c|c|c|p{12cm}|}
\hline
GET & \multicolumn{3}{|c|}{/follow\_info/good\_friends} \\
\hline\hline
 \  参数  & 类型 & 必填 &  说明  \\
\hline
 id  & 整数 & 必须 &  用户的id\\
   \hline
 page  & 整数 & 可选 & 分页,缺省值为1\\ 
 \hline
 pcount  & 整数 & 可选 & 每页的数量,缺省为20\\ 
     \hline
 name  & 字符串 & 可选 & 用户的名字\\ 
     \hline
 hash  & 整数 & 可选 & 是否返回hash,缺省值为0\\ 
\hline

\end{tabular}
   \end{center}
\end{table}

返回的内容格式同粉丝列表一样

这里返回的所有用户,都和当前登录用户互相关注。


\subsection{批量获取所有我关注用户的信息}

\begin{table}[H]
   \begin{center}
\begin{tabular}{|c|c|c|p{12cm}|}
\hline
GET & \multicolumn{3}{|c|}{/follow\_info/friend\_infos} \\
\hline\hline
 \  参数  & 类型 & 必填 &  说明  \\
\hline
 id  & 整数 & 必须 &  用户的id\\
\hline
\end{tabular}
   \end{center}
\end{table}

返回的用户基本信息的一个数组。



\subsection{关注列表ID}

\begin{table}[H]
   \begin{center}
\begin{tabular}{|c|c|c|p{12cm}|}
\hline
GET & \multicolumn{3}{|c|}{/follow\_info/friend\_ids} \\
\hline\hline
 \  参数  & 类型 & 必填 &  说明  \\
\hline
 id  & 整数 & 必须 &  用户的id\\
\hline

\end{tabular}
   \end{center}
\end{table}

\subsection{双向好友列表ID}

\begin{table}[H]
   \begin{center}
\begin{tabular}{|c|c|c|p{12cm}|}
\hline
GET & \multicolumn{3}{|c|}{/follow\_info/good\_friend\_ids} \\
\hline\hline
 \  参数  & 类型 & 必填 &  说明  \\
\hline
 id  & 整数 & 必须 &  用户的id\\
\hline

\end{tabular}
   \end{center}
\end{table}

\subsection{批量获取关注者的位置信息}
\begin{table}[H]
   \begin{center}
\begin{tabular}{|c|c|c|p{12cm}|}
\hline
GET & \multicolumn{3}{|c|}{/follow\_info/friend\_locs} \\
\hline\hline
 \  参数  & 类型 & 必填 &  说明  \\
\hline
 \  ids  & 字符串 & 必填 &  要获取的用户的id,多个id以英文","分割  \\
 \hline
\end{tabular}
   \end{center}
\end{table}


例子:

\begin{figure}[H]
\begin{verbatim}
访问:/follow_info/friend_ids?ids=502e6303421aa918ba000001,502e6303421aa918ba000003
返回值:

[{"id":"502e6303421aa918ba000001","last":"","time":""},
{"id":"502e6303421aa918ba000003","last":"","time":""}]


\end{verbatim}
\end{figure}

\subsection{批量获取好友位置信息}
\begin{table}[H]
   \begin{center}
\begin{tabular}{|c|c|c|p{12cm}|}
\hline
GET & \multicolumn{3}{|c|}{/follow\_info/good\_friend\_locs} \\
\hline\hline
 \  参数  & 类型 & 必填 &  说明  \\
\hline
 \  ids  & 字符串 & 必填 &  要获取的用户的id,多个id以英文","分割  \\
 \hline
\end{tabular}
   \end{center}
\end{table}

输出同上


\subsection{批量获取粉丝位置信息}
\begin{table}[H]
   \begin{center}
\begin{tabular}{|c|c|c|p{12cm}|}
\hline
GET & \multicolumn{3}{|c|}{/follow\_info/fan\_locs} \\
\hline\hline
 \  参数  & 类型 & 必填 &  说明  \\
\hline
 \  ids  & 字符串 & 必填 &  要获取的用户的id,多个id以英文","分割  \\
 \hline
\end{tabular}
   \end{center}
\end{table}

输出同上


\subsection{关于关注/好友接口的说明}
\begin{enumerate}
\item 关注信息要在客户端单独建表,本地完整保存。好友信息是关注信息的子集合。
\item 当关注信息为空时,调用friend\_infos接口初始化关注信息,然后调用good\_friend\_ids初始化好友信息。
\item 当用户手动刷新关注/好友列表时,调用friend\_ids/good\_friend\_ids获取最新的关注/好友列表。对于新增的id,调用user\_info/basic接口获得用户信息,对于减少的id,取消关注/好友标志。
\item 用户最后出现的地点信息,单独一个接口,每次按需获取。
\item 原有的friends和good\_friends接口只是为了兼容性而存在,以后不要再调用。
\item 粉丝接口followers接口不变,每次只缓存最新的20条,手动刷新和加载更多。
\end{enumerate}



\section{黑名单}
\subsection{黑名单列表}

\begin{table}[H]
   \begin{center}
\begin{tabular}{|c|c|c|p{12cm}|}
\hline
GET & \multicolumn{3}{|c|}{/blacklists} \\
\hline\hline
 \  参数  & 类型 & 必填 &  说明  \\
\hline
 id  & 整数 & 必须 &  用户的id\\
   \hline
 page  & 整数 & 可选 & 分页,缺省值为1\\ 
 \hline
 pcount  & 整数 & 可选 & 每页的数量,缺省为20\\ 
     \hline
 name  & 字符串 & 可选 & 用户的名字\\ 
     \hline
 hash  & 整数 & 可选 & 是否返回hash,缺省值为0\\ 
\hline
\end{tabular}
   \end{center}
\end{table}

返回的内容格式和好友列表一样。


\subsection{添加黑名单}

\begin{table}[H]
   \begin{center}
\begin{tabular}{|c|c|c|p{12cm}|}
\hline
POST & \multicolumn{3}{|c|}{/blacklists/create} \\
\hline\hline
 \  参数  & 类型 & 必填 &  说明  \\
\hline
 user\_id  & 整数 & 必须 &  当前登录用户的id\\
\hline
 block\_id  & 整数 & 必须 &  加黑阻止的用户的id\\
 \hline
 report  & 整数 & 可选 &  是否同时举报被加黑的用户,0不举报1举报\\
\hline
\end{tabular}
   \end{center}
\end{table}

\begin{figure}[H]
\begin{verbatim}

返回值:

{
"id":1,
"user_id":2,
"report":false,
"block_id":1
}

\end{verbatim}
\end{figure}



\subsection{删除黑名单}

\begin{table}[H]
   \begin{center}
\begin{tabular}{|c|c|c|p{12cm}|}
\hline
POST & \multicolumn{3}{|c|}{/blacklists/delete} \\
\hline\hline
 \  参数  & 类型 & 必填 &  说明  \\
\hline
 user\_id  & 整数 & 必须 &  当前登录用户的id\\
\hline
 block\_id  & 整数 & 必须 &  加黑用户的id\\
\hline
\end{tabular}
   \end{center}
\end{table}

例子:

\begin{figure}[H]
\begin{verbatim}
访问:curl -b "_session_id=f03bef9371c119b6fcecbeefdaaac1b2"
       -d "user_id=2&block_id=3" /blacklists/delete

返回值:

{
"deleted":3
}

\end{verbatim}
\end{figure}


