
\section{个人聊天时发语音}

\subsection{获取七牛云存储的文件上传token}

\begin{table}[H]
   \begin{center}
\begin{tabular}{|c|c|c|p{12cm}|}
\hline
POST & \multicolumn{3}{|c|}{/sound2s/uptoken} \\
\hline\hline
 \  参数  & 类型 & 必填 &  说明  \\
\hline
 user\_id  & 字符串 & 必须 &  当前登录用户的id\\
 \hline
\end{tabular}
   \end{center}
\end{table}


\subsection{上传文件}

调用七牛的API上传文件,其中包含的参数有:

\begin{table}[H]
   \begin{center}
\begin{tabular}{|c|c|c|p{12cm}|}
\hline
POST & \multicolumn{3}{|c|}{七牛的上传模式2:高级上传(带回调)} \\
\hline\hline
 \  参数  & 类型 & 必填 &  说明  \\
\hline
 uptoken  & 字符串 & 必须 &  上一步获得的token\\
 \hline
 file  & 文件 & 必须 &  要上传的文件\\
 \hline
 bucket  & 字符串 & 必须 & 必须是"sound"\\
 \hline
 key  & 字符串 & 必须 &  UUID的主键\\
  \hline
 callback\_params  & 字符串 & 必须 &  回调参数\\
 \hline    
\end{tabular}
   \end{center}
\end{table}

\begin{verbatim}
回调参数的格式:"sec=&from=&to=&id=&key=$(etag)&size=$(fsize)"
其中sec是声音的长度(秒),
from是发送者的uid,
to是接受者的uid,
id是客户端生成的UUID主键key,
"key=$(etag)&size=$(fsize)"这一段则不做改变。


\end{verbatim}

      
\subsection{上传成功后的回调}
文件上传成功后,七牛服务器会回调脸脸服务器的接口,然后把响应转发给客户端。
\begin{table}[H]
   \begin{center}
\begin{tabular}{|c|c|c|p{12cm}|}
\hline
POST & \multicolumn{3}{|c|}{/sound2s/callback} \\
\hline\hline
 \  参数  & 类型 & 必填 &  说明  \\
 \hline
  sec  & 整数 & 可选 &  语音长度:秒\\
  \hline
 from  & 字符串 & 必须 &  发送者uid\\
 \hline
 to  & 字符串 & 必须 &  接受者uid\\
 \hline
 id  & 字符串 & 必须 &  语音文件的id,也就是客户端生成的key\\
 \hline
\end{tabular}
   \end{center}
\end{table}

注意:本接口不需要客户端调用。

\begin{verbatim}

比如上一步中callback_params 的输入:
"sec=3&from=fid&to=tid&id=1376548998&key=$(etag)&size=$(fsize)"

本接口的输出如下:
{"from"=>"fid", "sec"=>"3", "size"=>"691", "to"=>"tid"
"id"=>"1376548710", "key"=>"FgaFYWcGt7w4f60v4NG2DB8zKSNV" } 
\end{verbatim}


\subsection{发送消息}
语音上传完成后,服务器端发送一个xmpp消息,其中的body内容格式为:

\begin{verbatim}
[sound:$id]$sec

\end{verbatim}


\subsection{在个人聊天时收到消息后获取语音}
\begin{table}[H]
   \begin{center}
\begin{tabular}{|c|c|c|p{12cm}|}
\hline
GET & \multicolumn{3}{|c|}{/sound2s/show} \\
\hline\hline
 \  参数  & 类型 & 必填 &  说明  \\
  \hline
 id  & 字符串 & 必须 & 语音id\\
\hline
\end{tabular}
   \end{center}
\end{table}


