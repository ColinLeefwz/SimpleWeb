\documentclass{article}
\usepackage{CJKutf8}

\usepackage{hyperref}
\hypersetup{colorlinks,%
                 citecolor=black,%
                 filecolor=black,%
                 linkcolor=black,%
                 urlcolor=blue,%
                 pdftex}

\author{YXY}
\title{脸脸应用设计方案}
  
\begin{document}
\begin{CJK}{UTF8}{gbsn}

\maketitle
\tableofcontents

\section{脸脸网站}
脸脸的官方网站是\href{http://www.dface.cn}{http://www.dface.cn}。官方网站的用途是提供脸脸手机应用的介绍、下载和反馈。个人用户无法利用网站登录和使用脸脸,商家可以通过网站登录到商家管理后台。


脸脸网站的设计可以参考微信官方网站,主要提供“首页 ,下载 , 最新消息,常见问题,商家加盟,联系我们”等几个栏目。总体上来说比较简单。


注意:目前官方网站的phone目录下是手机应用的原型设计,但是网站上线后,phone目录会禁止访问。phone目录下的文件也禁止引用网站其它部分的文件。


\section{手机端应用}
手机端应用的设计主要参照现有的原型设计。这里只是补充原型设计里无法描述的其它信息。


\section{脸脸网商家后台管理功能}
\subsection{商家帐户认证}
\subsection{管理滚动广告}
\subsection{活动管理}





\section{脸脸网服务器与手机端的接口}
\subsection{用户注册}
脸脸应用直接使用新浪微博的用户系统,所以从用户的角度看不需要主动在脸脸中注册。但是为了统计当前登录的用户信息,管理用户的聊天信息和好友关系,必须建立自己的用户系统。

用户注册的逻辑是:当手机用户经过新浪认证后,给脸脸网服务器发送登录成功的消息。服务器端判断该用户是否是首次使用脸脸,如果是首次使用,那么自动创建用户系统。

\subsection{用户登录}
手机用户经过新浪认证登录后,要在回调函数中通知脸脸网该用户登录了。
http://www.dface.cn/user/login
id 还是name? 调研新浪微博的回调接口

返回给用户的token


\subsection{用户退出}
手机用户退出时要先调用新浪的退出,然后通知脸脸网该用户退出了。
http://www.dface.cn/user/logout
id 还是name?
token


3、上传个人头像
参考http://open.weibo.com/wiki/2/statuses/upload


4、设置个人信息
设置生日(星座、年龄)信息。


5、获得登录用户的好友信息
5、获得登录用户的会话信息
6、添加好友
7、根据经纬度获得附近的商家
8、根据经纬度获得附近的用户
10、向给定用户发送信息
11、获取给定商家的信息


\newpage

\end{CJK}

\end{document}