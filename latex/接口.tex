\documentclass[cs4size]{ctexartutf8} 
\usepackage[unicode={true}]{hyperref}
\hypersetup{colorlinks,%
                 citecolor=black,%
                 filecolor=black,%
                 linkcolor=black,%
                 urlcolor=blue,%
                 pdftex}

\usepackage{graphicx}
\usepackage{float}

				
\author{YXY}
\title{脸脸网服务器与手机端的开发接口API}

\begin{document} 

\maketitle
\tableofcontents

\newpage

\textbf{注意事项}:
\begin{enumerate}
\item 所有的链接,如果要获得json格式的数据,最好带上".json"的后缀。因为同一个链接以后可能返回多种格式的数据,比如xml/html/json等。
\item API2(获得附近的商家)的链接有变更,从http://www.dface.cn/mshop/aroundme变更为http://www.dface.cn/aroundme/shops
\end{enumerate}

\newpage

\section{新浪微博登录后的回调页}

\begin{table}[H]
   \begin{center}
\begin{tabular}{|c|p{12cm}|}
\hline
\multicolumn{2}{|c|}{http://www.dface.cn/oauth2/sina\_callback} \\
\hline\hline
 \  参数  &  说明  \\
\hline
 code  &  新浪返回的code\\
\hline
\end{tabular}
   \end{center}
\end{table}


例子:

\begin{figure}[H]
\begin{verbatim}
访问:http://www.dface.cn/oauth2/sina_callback?code=...
该访问由新浪微博通过302转向发起。
返回值:

{
"id":1,
"password":"15c663b8ff620502",
"sina_uid":"1884834632",
"expires_in":75693,
"token":"2.00aaZYDCMcnDPCb4dc439e06i2m_GC",
"expires_at":1338431259
}

其中,id和password是该用户在脸脸网的id和密码,在首次新浪授权时创建。该id加上"@dface.cn"就是openfire的jid。(目前还没有和jid关联上。)

后面几个字段都是新浪提供的,其中sina_uid是该用户的新浪的uid,token是授权的access_token,其它几个是授权有效期信息。

\end{verbatim}
\end{figure}


\section{根据经纬度获得附近的商家}

\begin{table}[H]
   \begin{center}
\begin{tabular}{|c|p{12cm}|}
\hline
\multicolumn{2}{|c|}{http://www.dface.cn/aroundme/shops} \\
\hline\hline
 \  参数  &  说明  \\
\hline
 lat  &  经度\\
\hline
 lng  &  纬度\\ 
\hline
 accuracy  &  精确度\\ 
\hline
\end{tabular}
   \end{center}
\end{table}


例子:

\begin{figure}[H]
\begin{verbatim}
访问:/aroundme/shops.json?lat=30.2829754&lng=120.1337336
返回值:

[
{"mshop":
	{"name":"新紫轩花店",
	"address":"文一路266号",
	"linkman":"",
	"dp_id":5097101,
	"fullname":"",
	"lng":120.12763,
	"kb_id":"",
	"id":40056,
	"phone":"",
	"lat":30.28691}
}
]

\end{verbatim}
\end{figure}
     

TODO: 是否同时返回商家在线(男女)人数




\section{根据经纬度获得附近的用户}

\begin{table}[H]
   \begin{center}
\begin{tabular}{|c|p{12cm}|}
\hline
\multicolumn{2}{|c|}{http://www.dface.cn/aroundme/users} \\
\hline\hline
 \  参数  &  说明  \\
\hline
 lat  &  经度\\
\hline
 lng  &  纬度\\ 
\hline
 accuracy  &  精确度\\ 
\hline
\end{tabular}
   \end{center}
\end{table}


例子:

\begin{figure}[H]
\begin{verbatim}
访问:/aroundme/users.json?lat=30.2829754&lng=120.1337336&accuracy=100
返回值:

[
{"id":2,
"logo":"/phone2/images/namei2.gif",
"sina_uid":1743108103}
]

\end{verbatim}
\end{figure}


\section{进入现场签到}

\begin{table}[H]
   \begin{center}
\begin{tabular}{|c|p{12cm}|}
\hline
\multicolumn{2}{|c|}{http://www.dface.cn/checkins} \\
\hline\hline
 \  参数  &  说明  \\
\hline
 lat  &  经度\\
\hline
 lng  &  纬度\\ 
\hline
 accuracy  &  精确度\\ 
\hline
 mshop\_id  &  现场商家id\\ 
\hline
 user\_id  &  签到用户id\\ 
\hline
\end{tabular}
   \end{center}
\end{table}

注意:该请求必须是POST请求。如果是GET请求,获得的是签到列表。



\newpage


\end{document}
