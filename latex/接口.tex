\documentclass[cs4size]{ctexartutf8} 
\usepackage[unicode={true}]{hyperref}
\hypersetup{colorlinks,%
                 citecolor=black,%
                 filecolor=black,%
                 linkcolor=black,%
                 urlcolor=blue,%
                 pdftex}

\usepackage{graphicx}
\usepackage{float}

				
\author{YXY}
\title{脸脸网服务器与手机端的开发接口API}

\begin{document} 
\maketitle
\tableofcontents

\newpage

\textbf{注意事项}:
\begin{enumerate}
\item 所有的链接,如果要获得json格式的数据,最好带上".json"的后缀。因为同一个链接以后可能返回多种格式的数据,比如xml/html/json等。
\item 当调用出错时,会在返回的json中包含"error"信息。
\item 只有开发调试时调用http的接口,发布时全部使用https接口。
\end{enumerate}

\newpage

\section{初始化与升级}
\subsection{初始化}

手机应用启动后,初次访问时调用此API。
\begin{table}[H]
   \begin{center}
\begin{tabular}{|c|c|c|p{12cm}|}
\hline
GET & \multicolumn{3}{|c|}{https://42.121.79.210/init/init} \\
\hline\hline
 \  参数  & 类型 & 必填 &  说明  \\
 \hline
 model  & 字符串 & 必须 &  硬件型号\\
\hline
 os  & 字符串 & 必须 &  手机操作系统版本\\
 \hline
 mac  & 字符串 & 必须 &  无线网卡MAC地址的MD5\\
 \hline
 hash  & 字符串 & 必须 &  hash验证码\\
  \hline
 ver  & 字符串 & 可选 &  软件版本\\
\hline
\end{tabular}
   \end{center}
\end{table}

注意:本API直接通过IP地址“42.121.79.210”访问,不通过DNS。而后续所有API调用的IP地址根据本调用的返回的IP地址确定。在目前只有单台服务器的情况下,返回的ip地址也是“42.121.79.210”,但是以后会根据离用户的距离返回就近的IP地址。


hash的计算方式为:model+os+mac+"init"进行SHA1算法后取前32位,和登录时的hash算法[\ref{hash_algorithm}]类似。

\begin{verbatim}
返回值:
{
"ip": "60.191.119.190",
"xmpp": "60.191.119.190",
"ver":1.0
}

后续的http访问使用返回的ip地址,xmpp协议的访问使用xmpp地址。

[2012-12-27] 新增ver,表示脸脸当前线上的最新版本。客户端根据这个版本号可以判断是否需要升级。如果需要升级,提醒用户一次,不要每次启动重复提醒。比如1.1版本时提醒一次,如果用户忽略了,那么只有当版本升级到了1.2时再提醒一次。

\end{verbatim}

本API的作用一个是给客户端推荐IP地址,一个是统计应用的开机情况和操作系统分布。

TODO: android

\subsection{获得最新的升级信息}

如果客户端判断出版本不是最新的,调用本接口获得最新的升级信息。
\begin{table}[H]
   \begin{center}
\begin{tabular}{|c|c|c|p{12cm}|}
\hline
GET & \multicolumn{3}{|c|}{https://42.121.79.210/init/upgrade} \\
\hline\hline
 \  参数  & 类型 & 必填 &  说明  \\
 \hline
 os  & 字符串 & 必须 &  手机操作系统版本\\
 \hline
 ver  & 字符串 & 必须 &  客户端软件版本\\
\hline
\end{tabular}
   \end{center}
\end{table}

\begin{verbatim}
返回值:
["2.0.1","聊天室发图增加了分享到微信朋友圈和微信好友功能\n界面美化,更美观更清新",true]

分别代表:最新的软件版本、功能介绍、是否强制升级。

如果不强制升级,那么弹出升级对话框时有两个选择:立即升级/暂不升级。

如果强制升级,那么弹出升级对话框时只有一个选择:立即升级。强制升级用于发生不兼容改动时。强制升级也只应该提示一次,防止升级失败导致死循环。
\end{verbatim}




\section{用户帐号系统}
\subsection{新浪微博相关接口}

\subsubsection{网页登录接口:oauth2}


\begin{table}[H]
   \begin{center}
\begin{tabular}{|c|c|c|p{12cm}|}
\hline
POST & \multicolumn{3}{|c|}{/oauth2/sina\_callback} \\
\hline\hline
 \  参数  & 类型 & 必填 &  说明  \\
\hline
 code  & 字符串 & 必须 &  新浪返回的code\\
\hline
\end{tabular}
   \end{center}
\end{table}


例子:

\begin{figure}[H]
\begin{verbatim}
访问:/oauth2/sina_callback?code=...
该访问由新浪微博通过302转向发起。
返回值:

{
"id":1,
"password":"15c663b8ff620502",
"logo":"",
"name" : "name"
"gender" : 1
"wb_uid":"1884834632",
"expires_in":75693,
"token":"2.00aaZYDCMcnDPCb4dc439e06i2m_GC",
"expires_at":1338431259
}

其中,id和password是该用户在脸脸网的id和密码,在首次新浪授权时创建。该id加上"@dface.cn"就是openfire的jid。(目前还没有和jid关联上。)

后面几个字段都是新浪提供的,其中wb_uid是该用户的新浪微博的uid,token是授权的access_token,其它几个是授权有效期信息。

在首次登录获得用户的id和password以后,可以缓存到本地。以后用户每次登录,返回的id和password是不会改变的,除非服务器端重置密码(比如出现密码泄漏等安全情况)。

logo可以得知用户是否成功上传头像。对于第一次登录的用户,其logo为空,此时需要引导用户上传头像。

本接口不再需要,登录逻辑为:如果客户端安装了支持SSO的微博客户端,那么使用SSO的接口登录;否则使用xauth的接口登录。


\end{verbatim}
\end{figure}

\subsubsection{客户端登录接口:xauth}
\label{hash_algorithm}

\begin{table}[H]
   \begin{center}
\begin{tabular}{|c|c|c|p{12cm}|}
\hline
POST & \multicolumn{3}{|c|}{/oauth2/login} \\
\hline\hline
 \  参数  & 类型 & 必填 &  说明  \\
\hline
 name  & 字符串 & 必须 &  用户名\\
 \hline
 pass  & 字符串 & 必须 &  密码\\
  \hline
 mac  & 字符串 & 必须 &  网卡Mac地址\\
 \hline
 hash  & 字符串 & 必须 &  hash验证码\\
\hline
 bind  & 整数 & 可选 &  bind为0表示正常登录,1表示绑定微博帐户,2表示已经绑定过微博且已经用qq登录时,同时登录微博。默认为0\\
\hline
\end{tabular}
   \end{center}
\end{table}

本接口的输出和oauth2登录接口的输出相同。

\begin{verbatim}
hash的计算方式为:name+pass+mac+"dface"进行SHA1算法后取前32位。
比如用户名为name,密码为pa,mac地址为ss
那么待hash的字符串为"namepassdface",
其hash码为"11f4004a73a65117071bc4a7d3dfdf07"。
\end{verbatim}

新的登录方式不需要调用新浪的Xauth API,直接调用本API即可以登录。客户端应该不需要保存AppSecret。通过预先保存的AppKey和本调用输出中包含的token应该就可以访问新浪微博的API了。




\subsubsection{客户端SSO接口}
\label{hash_algorithm}

\begin{table}[H]
   \begin{center}
\begin{tabular}{|c|c|c|p{12cm}|}
\hline
POST & \multicolumn{3}{|c|}{/oauth2/sso} \\
\hline\hline
 \  参数  & 类型 & 必填 &  说明  \\
\hline
 remind\_in  & 字符串 & 必须 &  过期时间\\
 \hline
 expires\_in  & 字符串 & 必须 &  过期时间\\
  \hline
 uid  & 字符串 & 必须 &  微博的uid\\
  \hline
 access\_token  & 字符串 & 必须 &  微博的token\\
 \hline
 hash  & 字符串 & 必须 &  hash验证码\\
\hline
 bind  & 整数 & 可选 &  bind为0表示正常登录,1表示绑定微博帐户,2表示已经绑定过微博且已经用qq登录时,同时登录微博。默认为0\\
\hline
\end{tabular}
   \end{center}
\end{table}
当手机客户端安装有新浪官方微博客户端,使用微博SSO登录,调用本接口。
hash的计算方式为:uid+access\_token+"dface"进行SHA1算法后取前32位。

\subsubsection{解除新浪微博绑定}

\begin{table}[H]
   \begin{center}
\begin{tabular}{|c|c|c|p{12cm}|}
\hline
POST & \multicolumn{3}{|c|}{/oauth2/unbind\_sina} \\
\hline\hline
 \  参数  & 类型 & 必填 &  说明  \\
\hline
    uid  & 字符串 & 必须 &  当前用户的id\\
\hline
    wb\_uid  & 字符串 & 必须 &  新浪微博的uid\\    
\hline
\end{tabular}
   \end{center}
\end{table}
解除成功返回{unbind: true},否则返回error信息。



\subsection{QQ相关接口}
\subsubsection{QQ登录接口}
\label{hash_algorithm}

\begin{table}[H]
   \begin{center}
\begin{tabular}{|c|c|c|p{12cm}|}
\hline
POST & \multicolumn{3}{|c|}{/oauth2/qq\_client} \\
\hline\hline
 \  参数  & 类型 & 必填 &  说明  \\
 \hline
 expires\_in  & 字符串 & 必须 &  过期时间\\
  \hline
 openid  & 字符串 & 必须 &  QQ的openid\\
  \hline
 access\_token  & 字符串 & 必须 &  QQ授权的token\\
 \hline
 hash  & 字符串 & 必须 &  hash验证码\\
 \hline
 bind  & 整数 & 可选 &  bind为0表示正常登录,1表示绑定qq帐户,2表示已经绑定过qq且已经用微博登录时,同时登录qq。默认为0\\

\hline
\end{tabular}
   \end{center}
\end{table}
手机客户端使用QQ提供的SDK登录后,调用本接口。
hash的计算方式为:openid+access\_token+"dface"进行SHA1算法后取前32位。

\subsubsection{解除QQ绑定}

\begin{table}[H]
   \begin{center}
\begin{tabular}{|c|c|c|p{12cm}|}
\hline
POST & \multicolumn{3}{|c|}{/oauth2/unbind\_qq} \\
\hline\hline
 \  参数  & 类型 & 必填 &  说明  \\
\hline
    uid  & 字符串 & 必须 &  当前用户的id\\
\hline
    qq  & 字符串 & 必须 &  QQ分配的openid\\    
\hline
\end{tabular}
   \end{center}
\end{table}
解除成功返回{unbind: true},否则返回error信息。
新浪微博和QQ只有都处在有效登录状态时,才能解除其中的一个的绑定。



\subsection{手机注册接口}

\subsubsection{手机号码获取验证码}
\label{hash_algorithm}

\begin{table}[H]
   \begin{center}
\begin{tabular}{|c|c|c|p{12cm}|}
\hline
POST & \multicolumn{3}{|c|}{/phone/init} \\
\hline\hline
 \  参数  & 类型 & 必填 &  说明  \\
\hline
 phone  & 字符串 & 必须 &  用户的手机号码\\
\hline
 type  & 整数 & 可选 &  1代表注册,2代表绑定,3代表忘记密码\\
\hline
\end{tabular}
   \end{center}
\end{table}

调用成功返回{"code":HASH后的验证码},如果有错误,错误信息保存在error中。
同时验证码通过短信的方式发送到使用者的手机。

手机号码注册/忘记密码/绑定手机号码都要先调用本接口获取验证码。

[3013-08-11] 增加type参数,表示请求手机验证码的类型。同时取消客户端对验证码的HASH是否正确的判断

\subsubsection{手机号码注册}
\label{hash_algorithm}

\begin{table}[H]
   \begin{center}
\begin{tabular}{|c|c|c|p{12cm}|}
\hline
POST & \multicolumn{3}{|c|}{/phone/register} \\
\hline\hline
 \  参数  & 类型 & 必填 &  说明  \\
\hline
 phone  & 字符串 & 必须 &  用户的手机号码\\
\hline
 code  & 字符串 & 必须 &  验证码\\
\hline
 password  & 字符串 & 选填 &  用户设置的密码\\
\hline
\end{tabular}
   \end{center}
\end{table}

调用成功返回生成的用户id,如果有错误,错误信息保存在error中。

[3014-4-2] 手机号码注册时,密码变为选填。

\subsubsection{忘记密码}
\label{hash_algorithm}

\begin{table}[H]
   \begin{center}
\begin{tabular}{|c|c|c|p{12cm}|}
\hline
POST & \multicolumn{3}{|c|}{/phone/forgot\_password} \\
\hline\hline
 \  参数  & 类型 & 必填 &  说明  \\
\hline
 phone  & 字符串 & 必须 &  用户的手机号码\\
\hline
 code  & 字符串 & 必须 &  验证码\\
\hline
 password  & 字符串 & 必须 &  用户设置的密码\\
\hline
\end{tabular}
   \end{center}
\end{table}

本接口和手机号码注册接口参数一样,要求相反。手机号码注册要求该手机号码不存在,忘记密码要求该手机号码已注册。


\subsubsection{修改密码}
\label{hash_algorithm}

\begin{table}[H]
   \begin{center}
\begin{tabular}{|c|c|c|p{12cm}|}
\hline
POST & \multicolumn{3}{|c|}{/phone/change\_password} \\
\hline\hline
 \  参数  & 类型 & 必填 &  说明  \\
\hline
 oldpass  & 字符串 & 必须 &  原来的密码\\
\hline
 password  & 字符串 & 必须 &  用户设置的密码\\
\hline
\end{tabular}
   \end{center}
\end{table}



\subsubsection{手机号码登录}
\label{hash_algorithm}

\begin{table}[H]
   \begin{center}
\begin{tabular}{|c|c|c|p{12cm}|}
\hline
POST & \multicolumn{3}{|c|}{/phone/login} \\
\hline\hline
 \  参数  & 类型 & 必填 &  说明  \\
\hline
 phone  & 字符串 & 必须 &  用户的手机号码\\
\hline
 password  & 字符串 & 必须 &  用户设置的密码\\
\hline
\end{tabular}
   \end{center}
\end{table}

调用成功返回用户的信息,格式和user\_info\/get\_self相同。如果有错误,错误信息保存在error中。



\subsubsection{绑定手机号码}
\label{hash_algorithm}

\begin{table}[H]
   \begin{center}
\begin{tabular}{|c|c|c|p{12cm}|}
\hline
POST & \multicolumn{3}{|c|}{/phone/bind} \\
\hline\hline
 \  参数  & 类型 & 必填 &  说明  \\
\hline
 phone  & 字符串 & 必须 &  用户的手机号码\\
\hline
 code  & 字符串 & 必须 &  验证码\\
\hline
\end{tabular}
   \end{center}
\end{table}

绑定手机号码要先调用phone\/init获取验证码。
调用成功返回{bind: true},如果有错误,错误信息保存在error中。

[2013-8-15] 取消参数password, 绑定手机号码时可以不用输入手机号码。


\subsubsection{绑定手机后设置密码}
\label{hash_algorithm}

\begin{table}[H]
   \begin{center}
\begin{tabular}{|c|c|c|p{12cm}|}
\hline
POST & \multicolumn{3}{|c|}{/phone/set\_password} \\
\hline\hline
 \  参数  & 类型 & 必填 &  说明  \\
\hline
 password  & 字符串 & 必须 &  用户设置的密码\\
 \hline
 phone  & 字符串 & 必须 &  用户的手机号码\\
\hline
\end{tabular}
   \end{center}
\end{table}
如果已经设置了密码后要修改密码,不能调用本接口。


\subsubsection{解除手机号码绑定}
\label{hash_algorithm}

\begin{table}[H]
   \begin{center}
\begin{tabular}{|c|c|c|p{12cm}|}
\hline
POST & \multicolumn{3}{|c|}{/phone/unbind} \\
\hline\hline
 \  参数  & 类型 & 必填 &  说明  \\
\hline
 phone  & 字符串 & 必须 &  用户的手机号码\\
\hline
\end{tabular}
   \end{center}
\end{table}
至少有一种以上的绑定关系才能解绑。


\subsubsection{上传手机通讯录}
\label{hash_algorithm}

\begin{table}[H]
   \begin{center}
\begin{tabular}{|c|c|c|p{12cm}|}
\hline
POST & \multicolumn{3}{|c|}{/phone/upload\_address\_list} \\
\hline\hline
 \  参数  & 类型 & 必填 &  说明  \\
\hline
 phone  & 字符串 & 必须 &  用户的手机号码\\
 \hline
 list  & json字符串 & 必须 &  通讯录\\
\hline
\end{tabular}
   \end{center}
\end{table}


\begin{verbatim}
list是一个json字符串表示的数组,其中每一个条目是一个hash, 格式如下:
[{"id":323,"number":"(0571)28996083","name":"***"},...]

返回值为“{imported: 导入的通讯录的条数}”。
\end{verbatim}


\subsection{短信上传注册接口}


\subsubsection{获取短信发送号码和内容}

\begin{table}[H]
   \begin{center}
\begin{tabular}{|c|c|c|p{12cm}|}
\hline
GET & \multicolumn{3}{|c|}{/phone/sms\_up} \\
\hline\hline
 \  参数  & 类型 & 必填 &  说明  \\
\hline
 phone  & 字符串 & 必须 &  用户的手机号码\\
\hline
\end{tabular}
   \end{center}
\end{table}
注意:本接口只支持自己手机号码的注册,也就是发送短信的号码必须是接口中填写的号码。


\begin{verbatim}
返回值为“{phone: 发送到的手机号码, txt:要发送的内容}”。

然后客户端调用本机器的短信发送程序,把目标号码和短信内容设置好。


\end{verbatim}


\subsubsection{用户是否短信注册通过}
当用户点击本机的短信发送按钮后,调用本接口。

\begin{table}[H]
   \begin{center}
\begin{tabular}{|c|c|c|p{12cm}|}
\hline
GET & \multicolumn{3}{|c|}{/phone/do\_register} \\
\hline\hline
 \  参数  & 类型 & 必填 &  说明  \\
\hline
 phone  & 字符串 & 必须 &  用户的手机号码\\
\hline
\end{tabular}
   \end{center}
\end{table}

注册成功时,本接口的返回值和register接口一致。短信未收到则返回error信息。



\subsection{退出}

\begin{table}[H]
   \begin{center}
\begin{tabular}{|c|c|c|p{12cm}|}
\hline
POST & \multicolumn{3}{|c|}{/oauth2/logout} \\
\hline\hline
 \  参数  & 类型 & 必填 &  说明  \\
\hline
    pushtoken  & 浮点数 & 可选 &  当前设备的push消息token\\
\hline
\end{tabular}
   \end{center}
\end{table}

当用户主动退出时,调用本接口并断开XMPP的连接。
[2012-12-30] 新增pushtoken参数。



\subsection{登录与绑定的状态判断}
目前,脸脸帐号登录相关方有四个:脸脸web服务器、脸脸xmpp服务器,新浪微博,QQ。

\begin{enumerate}

\item 当访问web接口时,如果返回的error信息是"not login",此时代表要登录脸脸web服务器。但是脸脸不提供独立的帐户体系,要登陆脸脸Web服务器,必须登陆新浪微博或者QQ或者手机号码中的一个。当脸脸登录时,如果客户端缓存有未过期的新浪微博或者QQ认证信息,要丢弃。Web服务器认证信息保存在\_session\_id中。\_session\_id默认没有过期时间。
\item 如果脸脸\_session\_id包含登录信息(没返回not login):
\begin{enumerate}

\item 如果新浪微博或者QQ中的一个处于有效登录状态(wb\_token/qq\_token处在有效期内),需要使用另外一个服务时:
\begin{enumerate}
\item 如果以前绑定过另外一个的帐号,使用bind=2参数登录。
\item 如果没有,此时要绑定帐号。要使用bind=1参数登录。此时会绑定帐号并登录。
\end{enumerate}

\item 如果新浪微博和QQ都没有登录,此时要返回到登录界面。
\item 绑定操作(bind=1)只需要执行一次,
\item 当用户的个人信息中有wb\_uid,代表绑定过微博;有qq\_openid,代表绑定过QQ。
\item 当用户的个人信息中有wb\_token,代表登录了微博;有qq\_token,代表登录了QQ。
\item 绑定操作可以取消。而对于登录操作,如果新浪微博或者QQ都没有登录,则不能取消。
\end{enumerate}

\item 关于\_session\_id:
\begin{enumerate}
\item 首次安装应用调用init接口时,生成\_session\_id。以后所有的调用包括后来的init调用也要带上\_session\_id。
\item 即使没有登录也分配\_session\_id。
\item 调用logout时会变更\_session\_id。
\item 任何调用,如果没带\_session\_id,会生成新的\_session\_id。这样用户原来的所有状态信息会丢失,这是不建议的。

\end{enumerate}

\item 总的来说就是:没有\_session\_id通过init调用获得\_session\_id;每次调用总是带上\_session\_id,logout时更新\_session\_id。


\item xmpp服务器是单独登录的。每次应用打开的时候登录,退出或切换到后台时退出。


\end{enumerate}


\subsection{关于首次登录的判断}
以前,脸脸判断用户是否首次登录是看TA是否有头像。如果没有头像,则走注册流程。

现在更改为:每种登录接口(sina xauth/sso, qq)都会在用户首次登录时返回字段:{newuser:1}。当有newuser时,走注册流程。




\section{QQ登录用户看他人微博}
由于看他人微博需要微博登陆,所以使用QQ登陆而没有绑定微博的用户无法直接查看他人的微博。获取他人微博的接口是:
\begin{verbatim}
https://api.weibo.com/2/statuses/user_timeline.json
\end{verbatim}

把上面的域名api.weibo.com替换为www.dface.cn,且不传递access\_token 参数,就可以通过脸脸查看他人的微博了。比如:
\begin{verbatim}
http://www.dface.cn/2/statuses/user_timeline.json?uid=1884834632
\end{verbatim}
其中uid指的是新浪微博的wb\_uid,其他参数和新浪的接口完全一致。
参见:http://open.weibo.com/wiki/2/statuses/user\_timeline



\section{首次使用脸脸时发送分享信息}

\begin{table}[H]
   \begin{center}
\begin{tabular}{|c|c|c|p{12cm}|}
\hline
POST & \multicolumn{3}{|c|}{/oauth2/share} \\
\hline\hline
 \  参数  & 类型 & 必填 &  说明  \\
\hline
   ver  & 浮点数 & 可选 &  当前安装的版本\\
\hline
   t  & 整数 & 可选 &  分享类型:0代表微博,1代表qq空间,默认为0\\
\hline
\end{tabular}
   \end{center}
\end{table}

首次使用脸脸/或者以后重大版本更新时,以登陆用户个人的名义发送使用脸脸的分享微博,微博由服务器端发送。


\section{现场与附近}

\subsection{根据经纬度获得可能的现场商家}

\begin{table}[H]
   \begin{center}
\begin{tabular}{|c|c|c|p{12cm}|}
\hline
GET & \multicolumn{3}{|c|}{/aroundme/shops} \\
\hline\hline
 \  参数  & 类型 & 必填 &  说明  \\
\hline
 lat  & 浮点数 & 必须 & 经度\\
\hline
 lng  &  浮点数 & 必须 & 纬度\\ 
\hline
 accuracy  & 整数 & 必须 & 经纬度的精确度\\ 
\hline
 speed  & 浮点数 & 可选 & 速度 m/s\\   
\hline
 bssid  & 字符串 & 可选 & wifi上网时的bssid,如果不是wifi上网则忽略\\  
\hline
 baidu  & 整数 & 可选 & 如果是百度坐标,则baidu=1,否则忽略\\  
\hline
\end{tabular}
   \end{center}
\end{table}

注意:一定要在accuracy小于200m或者调用获取GPS的API超过三次后才能调用本接口。

Gps获取位置是一个逐步精确的过程。很多时候首次获取的误差超过1000米。此时显然无法准确获得商家列表。此时建议等待0.5秒再次获取gps。要保证gps的精确度小于200m,或则定位三次以上确实不能更准确了。宁可等待的时间久一些,也要保证定位的准确性。

最多返回50条数据,也有可能没有数据(比如西部沙漠地区)。

例子:

\begin{figure}[H]
\begin{verbatim}
访问:/aroundme/shops.json
返回值:

[
	{
	"name":"新紫轩花店",
	"address":"文一路266号",
	"lng":120.12763,
	"id":40056,
	"phone":"",
	"lat":30.28691,
	"lo":[30.297,120.1288],
	"t":1,
	"user":0,
	"male",0,
	"female",0
	}
]

除了返回商家的id、名称、电话、经纬度以后,增加了在该商家的用户总数“user”、男“male”、女“female”数量。

[2012-8-7] 新增lo字段。这是商家的实际经纬度,以前的lat/lng是地图上显示的经纬度,两者有几百米的误差。计算商家和自己的距离时,要采用lo来计算。

[2012-8-8] 新增t字段,表示商家的类型。

[2013-5-6] 新增coupon字段,表示商家是否由优惠券。

\end{verbatim}
\end{figure}



\subsection{根据经纬度获得附近的热点商家}

\begin{table}[H]
   \begin{center}
\begin{tabular}{|c|c|c|p{12cm}|}
\hline
GET & \multicolumn{3}{|c|}{/shop/nearby} \\
\hline\hline
 \  参数  & 类型 & 必填 &  说明  \\
\hline
 lat  & 浮点数 & 必须 & 经度\\
\hline
 lng  &  浮点数 & 必须 & 纬度\\ 
\hline
 accuracy  & 整数 & 必须 & 经纬度的精确度\\ 
 \hline
 page  & 整数 & 可选 & 分页,缺省值为1\\ 
 \hline
 pcount  & 整数 & 可选 & 每页的数量,缺省为20\\ 
  \hline
 name  & 字符串 & 可选 & 商家的名称\\ 
  \hline
 type  & 整数 & 可选 & 商家类型\\  
\hline
 baidu  & 整数 & 可选 & 如果是百度坐标,则baidu=1,否则忽略\\  
\hline
\end{tabular}
   \end{center}
\end{table}

商家类型:
\begin{enumerate}
\item 酒吧• 活动
\item 咖啡• 茶馆   
\item 餐饮• 酒店
\item 休闲• 娱乐
\item 购物• 广场
\item 楼宇• 社区
\end{enumerate}


用户当前所在的现场只有一个,但是由于定位有误差,所以服务器返回最有可能的几个商家让用户选择;而附近的商家有很多,按距离由近到远排序,且支持查询、分类筛选和分页。


\subsection{根据经纬度获得附近的热点用户}

\begin{table}[H]
   \begin{center}
\begin{tabular}{|c|c|c|p{12cm}|}
\hline
GET & \multicolumn{3}{|c|}{/aroundme/hot\_users} \\
\hline\hline
 \  参数  & 类型 & 必填 &  说明  \\
\hline
 lat  & 浮点数 & 必须 & 经度\\
\hline
 lng  &  浮点数 & 必须 & 纬度\\ 
  \hline
 page  & 整数 & 可选 & 分页,缺省值为1\\ 
 \hline
 pcount  & 整数 & 可选 & 每页的数量,缺省为20\\ 
\hline
 baidu  & 整数 & 可选 & 如果是百度坐标,则baidu=1,否则忽略\\  
\hline
\end{tabular}
   \end{center}
\end{table}

输出格式参考“根据商家ID获得商家用户”的接口。唯一的不同点是:“根据商家ID获得商家用户”返回该用户在商家出现的time, 而本接口返回用户当前的location。




\subsection{获得附近的用户}

\begin{table}[H]
   \begin{center}
\begin{tabular}{|c|c|c|p{12cm}|}
\hline
GET & \multicolumn{3}{|c|}{/aroundme/users} \\
\hline\hline
 \  参数  & 类型 & 必填 &  说明  \\
 \hline
 gender  & 数字 & 必填 &  用户的性别,未设置0男1女2\\
\hline
\end{tabular}
   \end{center}
\end{table}

输出格式和“根据商家ID获得商家用户”的接口一样。

本接口用于“应用安装的时候,用户上传头像的页面”。其中有8张头像,是本地存储4张/网络获取(通过本接口)4张。其中本地的4张立刻显示。


\subsection{地点报错}

\begin{table}[H]
   \begin{center}
\begin{tabular}{|c|c|c|p{12cm}|}
\hline
GET & \multicolumn{3}{|c|}{/aroundme/shop\_report} \\
\hline\hline
 \  参数  & 类型 & 必填 &  说明  \\
 \hline
 uid  & 字符串 & 必须 & 登录用户的uid\\
\hline
 lat  & 浮点数 & 必须 & 经度\\
\hline
 lng  &  浮点数 & 必须 & 纬度\\ 
\hline
 accuracy  & 整数 & 必须 & 经纬度的精确度\\ 
\hline
 bssid  & 字符串 & 可选 & wifi上网时的bssid,如果不是wifi上网则忽略\\  
\hline
 baidu  & 整数 & 可选 & 如果是百度坐标,则baidu=1,否则忽略\\  
\hline
\end{tabular}
   \end{center}
\end{table}

本接口需要通过浏览器以URL的方式直接调用。



\section{根据商家ID获得商家的信息}
\subsection{根据商家ID获得商家的基本信息}

\begin{table}[H]
   \begin{center}
\begin{tabular}{|c|c|c|p{12cm}|}
\hline
GET & \multicolumn{3}{|c|}{/shop/info} \\
\hline\hline
 \  参数  & 类型 & 必填 &  说明  \\
\hline
 id  & 整数 & 必须 & 商家的ID\\
\hline
\end{tabular}
   \end{center}
\end{table}

例子:

\begin{figure}[H]
\begin{verbatim}

{
    "name": "浙江科技产业大厦",
    "t": 6,
    "lat": 30.280254,
    "lng": 120.121824,
    "address": "古翠路80",
    "id": 4928288,
    "staffs": [
        "502e6303421aa918ba000001"
    ],
    "notice":"",
    "text":"",    
    "photos": []
}

staffs是该商家的员工数组。在商家聊天室,商家和商家的员工发言时都加V并加黄色背景。只有该商家在商家聊天室才加V,其它商家不做特殊处理。

notice是该商家发布的公告,已取消。

新增photos字段,表示该商家最热的图片,最多4张。

[2013-04-09] 新增字段text,表示用户在分享图片时,默认带上的文字。

\end{verbatim}
\end{figure}



\subsection{根据商家ID获得商家的图片墙}
商家的图片来自个人在聊天室的发图。\label{photowall}

\begin{table}[H]
   \begin{center}
\begin{tabular}{|c|c|c|p{12cm}|}
\hline
GET & \multicolumn{3}{|c|}{/shop/photos} \\
\hline\hline
 \  参数  & 类型 & 必填 &  说明  \\
\hline
 id  & 整数 & 必须 & 商家的ID\\
   \hline
 page  & 整数 & 可选 & 分页,缺省值为1\\ 
 \hline
 pcount  & 整数 & 可选 & 每页的数量,缺省为20\\ 
\hline
\end{tabular}
   \end{center}
\end{table}

例子:

\begin{figure}[H]
\begin{verbatim}
[{"id":"509a061fbe4b197297000002",
"room":"1",
"desc":"图片说明",  
"updated_at":"2013-01-13T10:11:42Z",
"user_id":"502e61bfbe4b1921da000005",
"weibo":false,
"logo":"http://oss.aliyuncs.com/dface_test/509a061fbe4b197297000002/0.jpg",
"logo_thumb2":"http://oss.aliyuncs.com/dface_test/509a061fbe4b197297000002/t2_0.jpg",
"id":"509a061fbe4b197297000002"}]

\end{verbatim}
\end{figure}


\subsection{根据商家ID获得当前在该商家的所有用户}

\begin{table}[H]
   \begin{center}
\begin{tabular}{|c|c|c|p{12cm}|}
\hline
GET & \multicolumn{3}{|c|}{/shop/users} \\
\hline\hline
 \  参数  & 类型 & 必填 &  说明  \\
\hline
 id  & 整数 & 必须 & 商家的ID\\
  \hline
 page  & 整数 & 可选 & 分页,缺省值为1\\ 
 \hline
 pcount  & 整数 & 可选 & 每页的数量,缺省为20\\ 
\hline
\end{tabular}
   \end{center}
\end{table}


例子:

\begin{figure}[H]
\begin{verbatim}

[
{"id":1,
"logo":"/phone2/images/namei2.gif",
"name":"name23",
"wb_uid":1884834632,
"gender":0,
"friend":true,
"follower":false,
"birthday":null}
]

"friend"表示该用户是否是当前用户的朋友;"follower"代表该用户是否关注了当前用户。

[2012-8-8] 新增time字段,表示用户在该商家的最后活跃时间。
[2012-10-30] 新增分页参数。


\end{verbatim}
\end{figure}

在商家的用户的统计数据根据签到来统计,而签到则是当用户进入商家聊天室时自动后台发送。


\subsection{根据商家ID获得商家的聊天室历史}
商家的图片来自个人在聊天室的发图。\label{photowall}

\begin{table}[H]
   \begin{center}
\begin{tabular}{|c|c|c|p{12cm}|}
\hline
GET & \multicolumn{3}{|c|}{/shop/history} \\
\hline\hline
 \  参数  & 类型 & 必填 &  说明  \\
\hline
 id  & 整数 & 必须 & 商家的ID\\
   \hline
 skip  & 整数 & 必须 & 跳过多少条消息\\ 
 \hline
 pcount  & 整数 & 可选 & 每页的数量,缺省为10\\ 
\hline
\end{tabular}
   \end{center}
\end{table}

skip的大小应该是聊天室当前最新显示的消息的总数。这样传递的skip获得的历史消息才不会和聊天室当前的消息重复。

例子:

\begin{figure}[H]
\begin{verbatim}
每次进聊天室,总是用下面的Xmpp接口获得最新的10条消息
  <x xmlns='http://jabber.org/protocol/muc'>
    <history maxstanzas='10'/>
  </x>
  当要加载超过10条的消息时,才调用本接口。历史消息接口的数据从数据库中加在,而通过xmpp的history stanza加载的最新10条消息是从内存中加载的。
  
[["5160f00fc90d8be23000007c", "茉莉", 1367575501, "q9d8G-8"], 
["5160f00fc90d8be23000007c", "哦你家", 1367575499, "q9d8G-7"]]
返回的是历史消息的数据,其中每个字段的含义是:
[发送者、消息内容、发送时间、消息ID ]。
\end{verbatim}
\end{figure}



\section{签到与足迹}

\subsection{进入现场签到}

\begin{table}[H]
   \begin{center}
\begin{tabular}{|c|c|c|p{12cm}|}
\hline
POST & \multicolumn{3}{|c|}{/checkins} \\
\hline\hline
 \  参数  & 类型 & 必填 &  说明  \\
\hline
 lat  & 浮点数 & 必须 & 经度\\
\hline
 lng  &  浮点数 & 必须 & 纬度\\ 
\hline
 accuracy  & 整数 & 必须 & 经纬度的精确度\\ 
\hline
 shop\_id  & 整数 & 必须 &  现场商家id\\ 
\hline
 user\_id  & 整数 & 必须 &  签到用户id\\ 
\hline
 od  & 整数 & 必须 &  实际签到的商家的排序\\  
\hline
 altitude  &  浮点数 & 可选 & 海拔高度\\ 
\hline
 altacc  & 整数 & 可选 & 海拔高度的精确度\\  
\hline
 speed  & 浮点数 & 可选 & 速度 m/s\\   
\hline
 bssid  & 字符串 & 可选 & wifi上网时的bssid,如果不是wifi上网则忽略\\  
\hline
 ssid  & 字符串 & 可选 & wifi上网时的ssid,如果不是wifi上网则忽略\\  
\hline
 baidu  & 整数 & 可选 & 如果是百度坐标,则baidu=1,否则忽略\\  
\hline
\end{tabular}
   \end{center}
\end{table}

注意:该请求必须是POST请求。如果是GET请求,获得的是签到列表。

[2012-8-10] 新增od字段,表示用户实际签到的商家在现场商家列表中的顺序位置,从1开始,也就是从/aroundme/shops中获得的商家列表中,那个被用户选中了。



\subsection{输入商家的全称创建商家并签到}

\begin{table}[H]
   \begin{center}
\begin{tabular}{|c|c|c|p{12cm}|}
\hline
POST & \multicolumn{3}{|c|}{/checkins/new\_shop} \\
\hline\hline
 \  参数  & 类型 & 必填 &  说明  \\
\hline
 lat  & 浮点数 & 必须 & 经度\\
\hline
 lng  &  浮点数 & 必须 & 纬度\\ 
\hline
 accuracy  & 整数 & 必须 & 经纬度的精确度\\ 
\hline
 user\_id  & 整数 & 必须 &  签到用户id\\ 
\hline
 sname  & 字符串 & 必须 &  商家的全称\\  
\hline
 altitude  &  浮点数 & 可选 & 海拔高度\\ 
\hline
 altacc  & 整数 & 可选 & 海拔高度的精确度\\  
 \hline
 speed  & 浮点数 & 可选 & 速度 m/s\\   
\hline
 bssid  & 字符串 & 可选 & wifi上网时的bssid,如果不是wifi上网则忽略\\  
\hline
\end{tabular}
   \end{center}
\end{table}

注意:该请求必须是POST请求。

定位时,当用户当前的商家不存在时,可以通过输入商家的全称进入该商家的现场并签到。

本API接口和签到接口类似,不同之处在于:

1)定位时找到了商家,客户端签到(报告商家id/商家的排名od),客户端同时进入该商家的聊天室。
2)定位时找不到商家,用户输入商家的全称(sname),客户端调用本接口并获得新建立的商家的id,然后进入该商家的聊天室。


例子:

\begin{figure}[H]
\begin{verbatim}
返回值:

{"name":"asdf","lo":[1,2],"lat":1,"lng":2,"address":null,"id":92264}


\end{verbatim}
\end{figure}



\subsection{签到时输入地点名称查找地点}

\begin{table}[H]
   \begin{center}
\begin{tabular}{|c|c|c|p{12cm}|}
\hline
GET & \multicolumn{3}{|c|}{/shop/add\_search} \\
\hline\hline
 \  参数  & 类型 & 必填 &  说明  \\
\hline
 lat  & 浮点数 & 必须 & 经度\\
\hline
 lng  &  浮点数 & 必须 & 纬度\\ 
\hline
 accuracy  & 整数 & 必须 & 经纬度的精确度\\ 
\hline
 sname  & 字符串 & 必须 &  商家的全称\\  
\hline
\end{tabular}
   \end{center}
\end{table}

定位时,如果在滚轮中找不到商家,在输入商家名称时查询本接口给用户提供实时匹配的可选择地点。


例子:

\begin{figure}[H]
\begin{verbatim}

curl /shop/add_search?lat=30.28&lng=120.108&accuracy=100&sname=新村

返回值:

[{"id":10443603,"name":"高教新村"},{"id":10428968,"name":"古荡新村"}]

\end{verbatim}
\end{figure}


\subsection{获得当前登录用户的足迹}

\begin{table}[H]
   \begin{center}
\begin{tabular}{|c|c|c|p{12cm}|}
\hline
GET & \multicolumn{3}{|c|}{/user\_info/trace} \\
\hline\hline
 \  参数  & 类型 & 必填 &  说明  \\
   \hline
 page  & 整数 & 可选 & 分页,缺省值为1\\ 
 \hline
 pcount  & 整数 & 可选 & 每页的数量,缺省为20\\ 
  \hline
 hash  & 整数 & 可选 & 是否返回hash,缺省值为0\\
 \hline
\end{tabular}
   \end{center}
\end{table}

例子:

\begin{figure}[H]
\begin{verbatim}
访问:/user_info/trace
返回值:
{"pcount":20,"data":
[{"id":"50b71b73c90d8b3c73000019","time":["09.22","14:24"],
"shop":"(下沙服务区)书报百货","shop_id":1,
"photos":[{"logo":"http://oss.aliyuncs.com/dface_test/5044301dbe4b192a30000002/0.jpg",
"logo_thumb2":"http://oss.aliyuncs.com/dface_test/5044301dbe4b192a30000002/t2_0.jpg",
"id":"5044301dbe4b192a30000002","desc":null}]},
{"time":["07.21","00:02"],"shop":"海洋二所","shop_id":60775,"photos":[]}]
}

其中pcount代表数据库中查询到的足迹的数量,如果该pcount等于客户端请求的数量,那么代表还会有更多的足迹。否则代表该用户没有足迹了。
而data中保存的实际返回的足迹,其数量可能少于数据库中查询到的足迹。比如一个人在一天内在同一个地方的签到会被合并。

data中的字段说明:
{
id:签到id
time: [签到日期,签到时间]
shop:商家
shop_id:商家id
photos :[{desc:图片描述,id:图片id,logo:原图URL,logo_thumb2:缩略图}]
}

\end{verbatim}
\end{figure}



\subsection{删除当前登录用户的一个签到足迹}

\begin{table}[H]
   \begin{center}
\begin{tabular}{|c|c|c|p{12cm}|}
\hline
POST & \multicolumn{3}{|c|}{/checkins/delete} \\
\hline\hline
 \  参数  & 类型 & 必填 &  说明  \\
   \hline
 id  & 字符串 & 必须 & 签到的id\\ 
 \hline
\end{tabular}
   \end{center}
\end{table}

注意:该请求必须是POST请求。




\section{用户基本信息}


\subsection{获得用户基本信息}

\begin{table}[H]
   \begin{center}
\begin{tabular}{|c|c|c|p{12cm}|}
\hline
GET & \multicolumn{3}{|c|}{/user\_info/basic} \\
\hline\hline
 \  参数  & 类型 & 必填 &  说明  \\
\hline
 id  & 整数 & 必须 &  用户id\\
\hline
\end{tabular}
   \end{center}
\end{table}



例子:

\begin{figure}[H]
\begin{verbatim}
访问:/user_info/basic?id=50bec2c1c90d8bd12f000086
返回值:

{"name":"amanda林","signature":"","wb_v":false,"wb_vs":"","gender":2,"birthday":"1986-2-15","job":"","jobtype":5,"pcount":5,"wb_uid":"1735512691","id":"50bec2c1c90d8bd12f000086","logo":"http://oss.aliyuncs.com/logo/50fe370ec90d8b173a00023c/0.jpg","logo_thumb":"http://oss.aliyuncs.com/logo/50fe370ec90d8b173a00023c/t1_0.jpg","logo_thumb2":"http://oss.aliyuncs.com/logo/50fe370ec90d8b173a00023c/t2_0.jpg"}

如果是微博认证用户,有如下信息
   :wb_v,  #是否是微博认证用户
   :wb_vs # 微博认证说明
  
\end{verbatim}
\end{figure}



\subsection{获得用户和当前登录用户的关系}

\begin{table}[H]
   \begin{center}
\begin{tabular}{|c|c|c|p{12cm}|}
\hline
GET & \multicolumn{3}{|c|}{/user\_info/relation} \\
\hline\hline
 \  参数  & 类型 & 必填 &  说明  \\
\hline
 id  & 整数 & 必须 &  用户id\\
\hline
\end{tabular}
   \end{center}
\end{table}


获得信息里有两个关系字段:friend和follower。这里的关系都是针对我的,这里的我就是当前登录用户。如果我关注了该用户,那么该用户就是我的friend; 如果该用户关注了我,那么他就是我的follower。

例子:

\begin{figure}[H]
\begin{verbatim}
访问:/user_info/friend?id=502e6303421aa918ba000001
返回值:
{"id":"502e6303421aa918ba000001","friend":true,"follower":true}
\end{verbatim}
\end{figure}



\subsection{获得用户最后出现的位置信息}

\begin{table}[H]
   \begin{center}
\begin{tabular}{|c|c|c|p{12cm}|}
\hline
GET & \multicolumn{3}{|c|}{/user\_info/last\_loc} \\
\hline\hline
 \  参数  & 类型 & 必填 &  说明  \\
\hline
 id  & 整数 & 必须 &  用户id\\
\hline
\end{tabular}
   \end{center}
\end{table}

例子:

\begin{figure}[H]
\begin{verbatim}
访问:/user_info/last_loc?id=1
返回值:

{
"time":1天以前,
"last": "1天以前 顺旺基(益乐路店)" 
"id":1
}

last字段,表示用户最后在脸脸上的时间和地点,一个以空格分割的字符串。
\end{verbatim}
\end{figure}


\subsection{获得用户的地主地点列表}

\begin{table}[H]
   \begin{center}
\begin{tabular}{|c|c|c|p{12cm}|}
\hline
GET & \multicolumn{3}{|c|}{/user\_info/lords} \\
\hline\hline
 \  参数  & 类型 & 必填 &  说明  \\
\hline
 id  & 整数 & 必须 &  用户id\\
\hline
\end{tabular}
   \end{center}
\end{table}

例子:

\begin{figure}[H]
\begin{verbatim}
访问:/user_info/lords?id=502e6303421aa918ba000001
返回值:

[{"name":"千粥汇华星路店","lo":[30.281387,120.11969],"t":4,"lat":30.281387,"lng":120.11969,
"address":"","phone":"","id":21829233,"user":0,"male":0,"female":0},
{"name":"浦东国际人才城","lo":[31.189001,121.586937],"t":"10","lat":31.189001,"lng":121.586937,
"address":"","phone":"","id":21830801,"user":0,"male":0,"female":0}]

\end{verbatim}
\end{figure}


\subsection{获得当前登录用户基本信息}

\begin{table}[H]
   \begin{center}
\begin{tabular}{|c|c|c|p{12cm}|}
\hline
GET & \multicolumn{3}{|c|}{/user\_info/get\_self} \\
\hline\hline
 \  参数  & 类型 & 必填 &  说明  \\
\hline
\end{tabular}
   \end{center}
\end{table}

例子:

\begin{figure}[H]
\begin{verbatim}
访问:/user_info/get_self
返回值:

{
"name":null,
wb_uid:1884834632
"gender":null,
"birthday":null,
"invisible":0,
"logo":"/system/imgs/1/original/clojure.png?1339398140",
"logo_thumb":"/system/imgs/1/thumb/clojure.png?1339398140"
"password":"c84dad462d5b7282",
"id":1
}

[2013-03-12]  新增
微博登陆的话:增加wb_token,wb_expire
qq登陆的话:增加qq_token,qq_expire

[2013-06-28]  新增wb_hidden,  1代表对他人隐藏自己的微博,2代表解除微博绑定.

[2013-08-26]  新增psd字段,代表设置过密码;新增pmatch字段代表启用了通讯录匹配.

\end{verbatim}
\end{figure}




\subsection{设置当前登录用户基本信息}

\begin{table}[H]
   \begin{center}
\begin{tabular}{|c|c|c|p{12cm}|}
\hline
POST & \multicolumn{3}{|c|}{/user\_info/set} \\
\hline\hline
 \  参数  & 类型 & 必填 &  说明  \\
\hline
 name  & 字符串 & 选填 &  用户的名字. 长度小于64\\
\hline
 gender  & 数字 & 选填 &  用户的性别,未设置0男1女2\\
\hline
 birthday  & 字符串 & 选填 &  用户的生日,格式如2012-06-01\\
 \hline
 signature  & 字符串 & 选填 &  签名档. 长度小于255\\
 \hline
 job  & 字符串 & 选填 &  职业\\
 \hline
 jobtype  & 整数 & 选填 &  职业类别\\
 \hline
 hobby  & 字符串 & 选填 &  爱好. 长度小于255\\
 \hline
 invisible  & 数字 & 选填 &  0:不隐身,1:对黑名单隐身,2:陌生人隐身, 3:全部隐身\\
 \hline
 wb\_hidden  & 数字 & 选填 &  0:不隐藏微博, 1代表对他人隐藏自己的微博\\ 
\hline
 no\_push  & 数字 & 选填 &  0:默认可以push, 1代表不接收push消息提醒。只有android客户端需要设置这个参数\\ 
\hline
\end{tabular}
   \end{center}
\end{table}

注意:该请求必须是POST请求。

用户初次登录时获取新浪微博的名称/性别/出生日期等信息并提交到服务端。以后更新时可以只更新其中的某个字段。

隐身的效果是:
1、在商家的用户列表中不在出现该用户
2、不更新自己的位置信息
3、用户进入商家时,不发打招呼的信息。

例子:

\begin{figure}[H]
\begin{verbatim}
访问:curl -b "_session_id=ead9ac4f6291c55bb467ad4138eca2ed" 
        -F "name=newname" /user_info/set

返回值:

{
"name":newname,
"gender":null,
"birthday":null,
"logo":"/system/imgs/1/original/clojure.png?1339398140",
"logo_thumb":"/system/imgs/1/thumb/clojure.png?1339398140"
"id":1
}

\end{verbatim}
\end{figure}


\subsection{设置其它用户的备注名}

\begin{table}[H]
   \begin{center}
\begin{tabular}{|c|c|c|p{12cm}|}
\hline
POST & \multicolumn{3}{|c|}{/user\_info/set\_comment\_name} \\
\hline\hline
 \  参数  & 类型 & 必填 &  说明  \\
\hline
 id  & 字符串 & 必填 &  被设置备注名的用户的id\\
\hline
 name  & 字符串 & 必填 &  用户的名字. 长度小于64\\
\hline
\end{tabular}
   \end{center}
\end{table}
当前登录用户给其它用户设置备注名,方便记忆。

\begin{figure}[H]
\begin{verbatim}
返回值:{"success" : 1}
\end{verbatim}
\end{figure}


\subsection{获得当前登录用户设置的所有备注名}

\begin{table}[H]
   \begin{center}
\begin{tabular}{|c|c|c|p{12cm}|}
\hline
GET & \multicolumn{3}{|c|}{/user\_info/get\_comment\_names} \\
\hline\hline
 \  参数  & 类型 & 必填 &  说明  \\
\hline
 user\_id  & 字符串 & 必填 &  当前登录用户的id\\
\hline
\end{tabular}
   \end{center}
\end{table}
当前登录用户给其它用户设置备注名,方便记忆。

\begin{figure}[H]
\begin{verbatim}
访问:/user_info/get_comment_names?user_id=502e6303421aa918ba000005
返回值:
[{"id":"502e6303421aa918ba000001","s":"yxy"}]
id是被设置备注名的用户的id,s是备注名。
\end{verbatim}
\end{figure}


\subsection{根据用户名查找脸脸用户}

\begin{table}[H]
   \begin{center}
\begin{tabular}{|c|c|c|p{12cm}|}
\hline
GET & \multicolumn{3}{|c|}{/user\_info/search} \\
\hline\hline
 \  参数  & 类型 & 必填 &  说明  \\
  \hline
 name  & 字符串 & 必须 & 用户的脸脸名字、微博/qq昵称、手机号码等\\ 
\hline
 page  & 整数 & 可选 & 分页,缺省值为1\\ 
 \hline
 pcount  & 整数 & 可选 & 每页的数量,缺省为20\\ 
\hline
\end{tabular}
   \end{center}
\end{table}

根据用户的脸脸名字、微博/qq昵称、手机号码等查找脸脸用户。其中手机号码是精确查询,其它是模糊查询。


\begin{figure}[H]
\begin{verbatim}
返回用户的基本信息,该用户和当前用户的关系信息,距离等。如果是黑名单用户,则有“black:1”。

[{"name":"袁乐天","signature":"","gender":1,"birthday":"","jobtype":null,"pcount":0,"wb_uid":"a1","id":"502e6303421aa918ba00007c","logo":"","logo_thumb":"","logo_thumb2":"","wb_name":null,"qq_name":null,
"friend":false,"follower":false,"distance":99999999}]
\end{verbatim}
\end{figure}



\section{用户相册与头像}
用户相册中的第一张图片就是头像。

\subsection{当前登录用户上传图片}

\begin{table}[H]
   \begin{center}
\begin{tabular}{|c|c|c|p{12cm}|}
\hline
POST & \multicolumn{3}{|c|}{/user\_logos/create} \\
\hline\hline
 \  参数  & 类型 & 必填 &  说明  \\
\hline
 user\_logo[img]  & file & 必须 &  头像文件\\
 \hline
 user\_logo[t]  & 整数 & 可选 &  图片类型:1拍照;2选自相册\\
\hline
\end{tabular}
   \end{center}
\end{table}

注意:

\begin{enumerate}
\item 该请求必须是POST请求,enctype="multipart/form-data"。
\item 所上传文件的type必须是image/jpeg', 'image/gif' 或者'image/png'。
\item 图片文件要在客户端先压缩到640*640。
\item 图片名为logo;缩略图分为两种大小,logo\_thumb为100*100的png,logo\_thumb2为200*200的png。
\end{enumerate}

例子:

\begin{figure}[H]
\begin{verbatim}
访问:curl -F "user_logo[img]=@firefox.png;type=image/png" 
/user_logos


{"logo_thumb2":null, "logo_thumb1":null
,"id":6,
"logo":"",
"user_id":3,
"img_tmp":"/system/imgs/6/thumb/csky2.png?1340779488"}

\end{verbatim}

用户上传图片更改为异步操作。图片上传完成后返回图片的id,但是此时图片还没有实际的阿里云链接。客户端上传完成后,可以用下面的“根据图片id获得用户相册里的图片”接口尝试获得实际的上传图片和缩略图。

\end{figure}



\subsection{获得用户的图片列表}

\begin{table}[H]
   \begin{center}
\begin{tabular}{|c|c|c|p{12cm}|}
\hline
GET & \multicolumn{3}{|c|}{/user\_info/photos} \\
\hline\hline
 \  参数  & 类型 & 必填 &  说明  \\
\hline
 id  & 整数 & 必须 &  用户id\\
\hline
\end{tabular}
   \end{center}
\end{table}


例子:

\begin{figure}[H]
\begin{verbatim}
访问:/user_info/photos?id=1
返回值:

[
{"logo_thumb2":"/system/imgs/2/thumb2/2.png.png?1340616951","updated_at":"2012-06-25T09:35:51Z",
"id":2,"logo":"/system/imgs/2/original/2.png.png?1340616951","img_file_size":81881,"user_id":1,"logo_thumb":"/system/imgs/2/thumb/2.png.png?1340616951"},
{"logo_thumb2":"/system/imgs/4/thumb2/2.png.png?1340617039","updated_at":"2012-06-25T09:37:20Z",
"id":4,"logo":"/system/imgs/4/original/2.png.png?1340617039","img_file_size":81881,"user_id":1,"logo_thumb":"/system/imgs/4/thumb/2.png.png?1340617039"}
]

图片列表中的第一张图片就是用户的头像。


\end{verbatim}
\end{figure}



\subsection{当前登录用户批量调整所有图片的位置}

\begin{table}[H]
   \begin{center}
\begin{tabular}{|c|c|c|p{12cm}|}
\hline
POST & \multicolumn{3}{|c|}{/user\_logos/change\_all\_position} \\
\hline\hline
 \  参数  & 类型 & 必填 &  说明  \\
\hline
 ids  & 逗号分割的整数 & 必须 &  所有图片的id按新的循序排列\\
\hline
\end{tabular}
   \end{center}
\end{table}

说明:

比如原有图片的循序是2468,新的循序是8642,此时需要调用本接口,传递ids=8,6,4,2。此时服务器会更新所有图片的排序,而客户端只需要发送一个请求。

例子:

\begin{figure}[H]
\begin{verbatim}
访问:curl -F "ids=3,1,4" 
/user_logos/position


\end{verbatim}
\end{figure}



\subsection{当前登录用户删除图片}

\begin{table}[H]
   \begin{center}
\begin{tabular}{|c|c|c|p{12cm}|}
\hline
POST & \multicolumn{3}{|c|}{/user\_logos/delete} \\
\hline\hline
 \  参数  & 类型 & 必填 &  说明  \\
\hline
 id  & 整数 & 必须 &  图片的id\\
\hline
\end{tabular}
   \end{center}
\end{table}

例子:

\begin{figure}[H]
\begin{verbatim}
访问:curl -F "id=2" 
/user_logos/delete

输出:
{"deleted":"2"}

\end{verbatim}
\end{figure}


\section{关注和粉丝}
\subsection{添加我关注的人}

\begin{table}[H]
   \begin{center}
\begin{tabular}{|c|c|c|p{12cm}|}
\hline
POST & \multicolumn{3}{|c|}{/follows/create} \\
\hline\hline
 \  参数  & 类型 & 必填 &  说明  \\
\hline
 user\_id  & 整数 & 必须 &  当前登录用户的id\\
\hline
 follow\_id  & 整数 & 必须 &  关注用户的id\\
\hline
\end{tabular}
   \end{center}
\end{table}

注意:该请求必须是POST请求。

例子:

\begin{figure}[H]
\begin{verbatim}
访问:curl -b "_session_id=f03bef9371c119b6fcecbeefdaaac1b2"
       -d "user_id=2&follow_id=1" /follows

返回值:

{
"id":1,
"user_id":2,
"follow_id":1
}

\end{verbatim}
\end{figure}



\subsection{删除我的关注的人}

\begin{table}[H]
   \begin{center}
\begin{tabular}{|c|c|c|p{12cm}|}
\hline
POST & \multicolumn{3}{|c|}{/follows/delete} \\
\hline\hline
 \  参数  & 类型 & 必填 &  说明  \\
\hline
 user\_id  & 整数 & 必须 &  当前登录用户的id\\
\hline
 follow\_id  & 整数 & 必须 &  关注用户的id\\
\hline
\end{tabular}
   \end{center}
\end{table}

注意:该请求必须是POST请求。

例子:

\begin{figure}[H]
\begin{verbatim}
访问:curl -b "_session_id=f03bef9371c119b6fcecbeefdaaac1b2"
       -d "user_id=2&follow_id=3" /follows/delete

返回值:

{
"deleted":3
}

\end{verbatim}
\end{figure}



\subsection{粉丝列表}

\begin{table}[H]
   \begin{center}
\begin{tabular}{|c|c|c|p{12cm}|}
\hline
GET & \multicolumn{3}{|c|}{/follow\_info/followers} \\
\hline\hline
 \  参数  & 类型 & 必填 &  说明  \\
\hline
 id  & 整数 & 必须 & 用户的id\\
   \hline
 page  & 整数 & 可选 & 分页,缺省值为1\\ 
 \hline
 pcount  & 整数 & 可选 & 每页的数量,缺省为20\\ 
    \hline
 name  & 字符串 & 可选 & 用户的名字\\ 
    \hline    
 hash  & 整数 & 可选 & 是否返回hash,缺省值为0\\ 
\hline

\end{tabular}
   \end{center}
\end{table}

例子:

\begin{figure}[H]
\begin{verbatim}
访问:/follow_info/followers?id=2

返回值:

[
{"count":1}
{"data":
{"id":1,
"name":"name23",
"wb_uid":1884834632,
"gender":0,
"friend":true,
"follower":true,
"birthday":null,
"last":"1 hour 浙江科技产业大厦"}
}
]

其中count是符合条件的粉丝的总数。

获得用户基本信息里有两个关系字段:friend和follower。这里的关系都是针对我的,这里的我就是给定ID的用户。如果我关注了该用户,那么该用户就是我的friend; 如果该用户关注了我,那么他就是我的follower。

这里返回的所有用户,其"follower"值一定为true。

[2013-2-20] 新增last,表示用户最后一次出现的时间和地点。


\end{verbatim}
\end{figure}


\subsection{关注列表}
本接口以后将不再支持。
\begin{table}[H]
   \begin{center}
\begin{tabular}{|c|c|c|p{12cm}|}
\hline
GET & \multicolumn{3}{|c|}{/follow\_info/friends} \\
\hline\hline
 \  参数  & 类型 & 必填 &  说明  \\
\hline
 id  & 整数 & 必须 &  用户的id\\
   \hline
 page  & 整数 & 可选 & 分页,缺省值为1\\ 
 \hline
 pcount  & 整数 & 可选 & 每页的数量,缺省为20\\ 
     \hline
 name  & 字符串 & 可选 & 用户的名字\\ 
     \hline
 hash  & 整数 & 可选 & 是否返回hash,缺省值为0\\ 
\hline

\end{tabular}
   \end{center}
\end{table}

返回的内容格式同粉丝列表一样

在1.5版本以前,本接口返回所有我关注的用户。
在1.5版本以后,本接口返回所有我关注的且对方没有关注我的用户。双向关注的则由新增的good\_friends接口提供。


\subsection{双向好友列表}
本接口以后将不再支持。
\begin{table}[H]
   \begin{center}
\begin{tabular}{|c|c|c|p{12cm}|}
\hline
GET & \multicolumn{3}{|c|}{/follow\_info/good\_friends} \\
\hline\hline
 \  参数  & 类型 & 必填 &  说明  \\
\hline
 id  & 整数 & 必须 &  用户的id\\
   \hline
 page  & 整数 & 可选 & 分页,缺省值为1\\ 
 \hline
 pcount  & 整数 & 可选 & 每页的数量,缺省为20\\ 
     \hline
 name  & 字符串 & 可选 & 用户的名字\\ 
     \hline
 hash  & 整数 & 可选 & 是否返回hash,缺省值为0\\ 
\hline

\end{tabular}
   \end{center}
\end{table}

返回的内容格式同粉丝列表一样

这里返回的所有用户,都和当前登录用户互相关注。


\subsection{批量获取所有我关注用户的信息}

\begin{table}[H]
   \begin{center}
\begin{tabular}{|c|c|c|p{12cm}|}
\hline
GET & \multicolumn{3}{|c|}{/follow\_info/friend\_infos} \\
\hline\hline
 \  参数  & 类型 & 必填 &  说明  \\
\hline
 id  & 整数 & 必须 &  用户的id\\
\hline
\end{tabular}
   \end{center}
\end{table}

返回的用户基本信息的一个数组。



\subsection{关注列表ID}

\begin{table}[H]
   \begin{center}
\begin{tabular}{|c|c|c|p{12cm}|}
\hline
GET & \multicolumn{3}{|c|}{/follow\_info/friend\_ids} \\
\hline\hline
 \  参数  & 类型 & 必填 &  说明  \\
\hline
 id  & 整数 & 必须 &  用户的id\\
\hline

\end{tabular}
   \end{center}
\end{table}

\subsection{双向好友列表ID}

\begin{table}[H]
   \begin{center}
\begin{tabular}{|c|c|c|p{12cm}|}
\hline
GET & \multicolumn{3}{|c|}{/follow\_info/good\_friend\_ids} \\
\hline\hline
 \  参数  & 类型 & 必填 &  说明  \\
\hline
 id  & 整数 & 必须 &  用户的id\\
\hline

\end{tabular}
   \end{center}
\end{table}

\subsection{批量获取关注者的位置信息}
\begin{table}[H]
   \begin{center}
\begin{tabular}{|c|c|c|p{12cm}|}
\hline
GET & \multicolumn{3}{|c|}{/follow\_info/friend\_locs} \\
\hline\hline
 \  参数  & 类型 & 必填 &  说明  \\
\hline
 \  ids  & 字符串 & 必填 &  要获取的用户的id,多个id以英文","分割  \\
 \hline
\end{tabular}
   \end{center}
\end{table}


例子:

\begin{figure}[H]
\begin{verbatim}
访问:/follow_info/friend_ids?ids=502e6303421aa918ba000001,502e6303421aa918ba000003
返回值:

[{"id":"502e6303421aa918ba000001","last":"","time":""},
{"id":"502e6303421aa918ba000003","last":"","time":""}]


\end{verbatim}
\end{figure}

\subsection{批量获取好友位置信息}
\begin{table}[H]
   \begin{center}
\begin{tabular}{|c|c|c|p{12cm}|}
\hline
GET & \multicolumn{3}{|c|}{/follow\_info/good\_friend\_locs} \\
\hline\hline
 \  参数  & 类型 & 必填 &  说明  \\
\hline
 \  ids  & 字符串 & 必填 &  要获取的用户的id,多个id以英文","分割  \\
 \hline
\end{tabular}
   \end{center}
\end{table}

输出同上


\subsection{批量获取粉丝位置信息}
\begin{table}[H]
   \begin{center}
\begin{tabular}{|c|c|c|p{12cm}|}
\hline
GET & \multicolumn{3}{|c|}{/follow\_info/fan\_locs} \\
\hline\hline
 \  参数  & 类型 & 必填 &  说明  \\
\hline
 \  ids  & 字符串 & 必填 &  要获取的用户的id,多个id以英文","分割  \\
 \hline
\end{tabular}
   \end{center}
\end{table}

输出同上


\subsection{关于关注/好友接口的说明}
\begin{enumerate}
\item 关注信息要在客户端单独建表,本地完整保存。好友信息是关注信息的子集合。
\item 当关注信息为空时,调用friend\_infos接口初始化关注信息,然后调用good\_friend\_ids初始化好友信息。
\item 当用户手动刷新关注/好友列表时,调用friend\_ids/good\_friend\_ids获取最新的关注/好友列表。对于新增的id,调用user\_info/basic接口获得用户信息,对于减少的id,取消关注/好友标志。
\item 用户最后出现的地点信息,单独一个接口,每次按需获取。
\item 原有的friends和good\_friends接口只是为了兼容性而存在,以后不要再调用。
\item 粉丝接口followers接口不变,每次只缓存最新的20条,手动刷新和加载更多。
\end{enumerate}



\section{黑名单}
\subsection{黑名单列表}

\begin{table}[H]
   \begin{center}
\begin{tabular}{|c|c|c|p{12cm}|}
\hline
GET & \multicolumn{3}{|c|}{/blacklists} \\
\hline\hline
 \  参数  & 类型 & 必填 &  说明  \\
\hline
 id  & 整数 & 必须 &  用户的id\\
   \hline
 page  & 整数 & 可选 & 分页,缺省值为1\\ 
 \hline
 pcount  & 整数 & 可选 & 每页的数量,缺省为20\\ 
     \hline
 name  & 字符串 & 可选 & 用户的名字\\ 
     \hline
 hash  & 整数 & 可选 & 是否返回hash,缺省值为0\\ 
\hline
\end{tabular}
   \end{center}
\end{table}

返回的内容格式和好友列表一样。


\subsection{添加黑名单}

\begin{table}[H]
   \begin{center}
\begin{tabular}{|c|c|c|p{12cm}|}
\hline
POST & \multicolumn{3}{|c|}{/blacklists/create} \\
\hline\hline
 \  参数  & 类型 & 必填 &  说明  \\
\hline
 user\_id  & 整数 & 必须 &  当前登录用户的id\\
\hline
 block\_id  & 整数 & 必须 &  加黑阻止的用户的id\\
 \hline
 report  & 整数 & 可选 &  是否同时举报被加黑的用户,0不举报1举报\\
\hline
\end{tabular}
   \end{center}
\end{table}

\begin{figure}[H]
\begin{verbatim}

返回值:

{
"id":1,
"user_id":2,
"report":false,
"block_id":1
}

\end{verbatim}
\end{figure}



\subsection{删除黑名单}

\begin{table}[H]
   \begin{center}
\begin{tabular}{|c|c|c|p{12cm}|}
\hline
POST & \multicolumn{3}{|c|}{/blacklists/delete} \\
\hline\hline
 \  参数  & 类型 & 必填 &  说明  \\
\hline
 user\_id  & 整数 & 必须 &  当前登录用户的id\\
\hline
 block\_id  & 整数 & 必须 &  加黑用户的id\\
\hline
\end{tabular}
   \end{center}
\end{table}

例子:

\begin{figure}[H]
\begin{verbatim}
访问:curl -b "_session_id=f03bef9371c119b6fcecbeefdaaac1b2"
       -d "user_id=2&block_id=3" /blacklists/delete

返回值:

{
"deleted":3
}

\end{verbatim}
\end{figure}




\section{XMPP协议接口}


\subsection{与Xmpp协议的不同}

\begin{enumerate}
\item 脸脸中的关注和粉丝关系是单向的,和xmpp中的好友roster关系无关。
\item 只要用户登录,其xmpp协议中的presence就是在线,没有其它状态。
\item 脸脸中的隐身也和xmpp中的隐身状态无关。
\item 脸脸中的任意两个用户可以发送消息,不管对方是否隐身,只要对方未设置黑名单即可。如果对方未登录,那么就是离线消息。
\end{enumerate}

\subsection{脸脸Xmpp的JID}
\begin{enumerate}
\item 脸脸网的用户id加上"@dface.cn"就是Xmpp的jid。
登录时的resource格式为: '操作系统脸脸版本-月日',其中操作系统为i或者a。
\item 商家的id加上"@c.dface.cn"就是现场聊天室的jid。Xmpp协议允许用户在加入聊天室时取一个名字,脸脸的聊天室应该忽略该名字。脸脸用户加入聊天室的时候,其jid格式为"roomid@c.dface.cn/uid"。
\end{enumerate}
例如:个人登录Xmpp服务器后的完整JID如“502e6303421aa918ba000005@dface.cn/i2.3.4-0927-09:34”,进入聊天室的完整JID如"1234@c.dface.cn/502e6303421aa918ba000005"。


\subsection{脸脸中的聊天用户的三种类型}

\begin{enumerate}
\item管理员,其jid固定为"502e6303421aa918ba000001@dface.cn";
\item商家,其jid为's+商家id@dface.cn';
\item个人,其jid为'个人id@dface.cn'。
\end{enumerate}
如果一个jid以字母s开头,那么它一定是商家。




\subsection{消息ID}
所有的消息(包括单聊和聊天室),要保证消息id的唯一性,建议使用uuid生成消息的id。客户端发送的消息,由客户端生成这个id;服务器端发送的消息,由服务器端生成这个id。

目前,在发图的时候,存在同时收到两张一样的图片的消息的BUG。为了处理这个BUG,客户端在收到图片消息时,如果消息的图片ID一样,不重复显示图片。但是图片id以faq开头的图片消息不需要判断图片id重复的问题。



\subsection{聊天室的特定格式消息}
\begin{enumerate}
\item 聊天室中,消息id以ckn开头的消息代表是用户首次进入聊天室发送的通知类消息,客户端根据这个ckn头确定是否有新的人进入了聊天室。
\item 聊天室中,消息id以FAQ开头的是问答消息。
\item 聊天室中,消息id以COMMENT开头的是图片的评论消息。聊天室评论消息带楼数的格式:[img:id]楼数:评论内容。
\item 在聊天室收到的id以“coupon”开头的消息时,点击该消息要能跳转到优惠券文件夹。
\end{enumerate}


\subsection{摇一摇及其消息格式}

\begin{verbatim}
摇一摇的消息格式为:
 [摇一摇:$name]

其它客户端收到摇一摇的消息后,再将其转换为文字描述: “$name摇了摇手机和大家say hello~”
\end{verbatim}



\subsection{个人之间聊天的消息发送状态确认}
XMPP协议的消息发送状态确认参考规范“\href{http://xmpp.org/extensions/xep-0022.html}{XEP-0022: Message Events}”。其定义了四种消息事件:Offline、Delivered、Displayed、Composing。Dface只需要其中的两种。

\begin{verbatim}
当接收端收到消息时,发送delivered确认消息,例如:
<message id="Kk98S-16" to="s6@dface.cn/ylt">
<x xmlns="jabber:x:event"><delivered/><id>purplea5e6669c</id></x>
</message>

当接收端展示消息时,发送displayed确认消息,例如:
<message id="Kk98S-17" to="s6@dface.cn/ylt">
<x xmlns="jabber:x:event"><displayed/><id>purplea5e6669c</id></x>
</message>

发送端根据从接收端获得的状态通知更改单条消息的发送状态。

只有类型是message,且body不为空的消息需要确认状态。照片提醒消息(发送人是"sphoto@dface.cn")不需要确认状态。

当本地无法发送消息时(比如无网络、或者连接不上xmpp服务器),消息的状态为“发送失败”。发送失败的消息让用户手动重发。没有收到回执的消息,则自动重发。

对于服务器端发送的消息,比如图片/语音/名片等,客户端先保存消息到数据库,而服务器端发消息的ID客户端不知道。所以当客户端发送回执时。

\end{verbatim}

\subsection{消息提醒}
\begin{enumerate}
\item 当不在现场的界面收到现场聊天室的消息时,在现场按钮旁边加红点提醒;
\item 当收到发送人是"sphoto@dface.cn"发过来的单聊消息时,表明是有照片提醒消息。这时在我的照片等地方加红点提醒。[2.3以上版本,此功能取消]。
\item 已发/已读类状态确认类消息不需要提醒。
\item 好友动态消息以红点的方式提醒。
\item 好友评论消息以数字计数的方式提醒。
\item 其它个人之间聊天的消息以数字计数的方式提醒。
\end{enumerate}


\subsection{关于照片的类型}
聊天相关的照片本来分为两种:单聊的照片和聊天室的照片。分别通过photos和photo2s接口获取。但是后来增加了问答类的图片消息和单聊时的群聊照片提醒/个人头像推送。所以为了区分这几种照片,规定如下:
\begin{enumerate}
\item 在聊天室收到id以“faq”开头的图片消息,说明是问答类的图片消息 ,查看调用photos/show接口;
\item在单聊时收到id以“U”开头的图片消息,说明是个人单聊的图片消息,查看调用photo2s/show接口;
\item在单聊时收到id以“Logo”开头的图片消息,说明是个人头像消息,查看调用photo2s/show接口;
\item收到的其它图片消息都是聊天室的照片,包括在聊天室发的照片以及将该照片转发给个人。这类消息查看可以调用photos/show接口,也可以调用photo2s/show接口。但是这类照片还可以点评。点评时获取照片详情的接口则必须是photo/detail接口。
\end{enumerate}

不管是在单聊和群聊时,图片消息仅仅根据图片id即可判断出类别。

\section{好友动态}
\subsection{好友动态消息}
所有消息id以"FEED"开头的,都属于好友动态消息。好友动态消息以红点的方式提醒,不计数。
好友动态消息增加地点,IOS通过属性实现,Android通过扩展实现,如下:
\begin{verbatim}
"<x xmlns='dface.shop' SID='#{shop.id}' SNAME='#{shop.name}' ></x>"
\end{verbatim}

如果是多图的动态,那么在图片的描述信息开始是数字加":"开始,在扩展属性中有"dface.thumb2s"字段,内容是第二张图到最后一张图的缩略图的url,以“,”号分割。


\subsection{好友动态中的评论消息}
所有消息id以"COMMENT"开头的,都属于好友评论消息。评论消息以数字计数。
\begin{verbatim}

评论消息提醒的消息id格式:"COMMENT#{图片id},#{时间}"。
\end{verbatim}

当同时有动态消息和评论消息时,优先显示评论消息的数字。

好友动态在会话列表中按最新的动态的时间自然排序。

\subsection{对方加我时的单聊提醒消息}
\begin{verbatim}
所有消息id以"FOLLOW"开头的单聊消息,都属于加关注的提醒消息。
该消息的id格式为“FOLLOW#{uid1},#{uid2}”,其中uid1是加我的人的id,uid2是我的id。
\end{verbatim}


\subsection{谁访问过我的主页的消息}
\begin{verbatim}
所有消息id以"VISIT"开头的单聊消息,都属于访问我的主页的提醒消息。
该消息的id格式为“VISIT#{uid1},#{uid2}”,其中uid1是访问我的人的id,uid2是我的id。
\end{verbatim}



\section{在聊天室发图}
聊天发图主要流程是:客户端选择(拍摄)一张照片,通过http上传到服务器。上传完成后在xmpp里发送一个特定格式的消息。其它客户端收到消息后获取图片显示并发送回执消息。

\subsection{上传图片}

\begin{table}[H]
   \begin{center}
\begin{tabular}{|c|c|c|p{12cm}|}
\hline
POST & \multicolumn{3}{|c|}{/photos/create} \\
\hline\hline
 \  参数  & 类型 & 必填 &  说明  \\
\hline
 photo[img]  & file & 必须 &  头像文件\\
 \hline
 photo[room]  & 字符串 & 必须  &  聊天室的id\\
 \hline
 photo[weibo]  & 0/1 & 必须  &  是否同步发送到微博\\
  \hline
 photo[qq]  & 0/1 & 可选  &  是否同步发送到QQ空间\\
   \hline
 photo[wx]  & 0/1/2/3 & 可选  &  分享到微信: 1个人,2朋友圈, 3都分享了\\
   \hline
 photo[desc]  & 字符串 & 可选  &  图像的说明文字\\
 \hline
  photo[t]  & 整数 & 可选 &  图片类型:1拍照;2选自相册\\
 \hline
 wbtoken  & 字符串 & 可选  &  如果分享到微博,该用户的微博token\\
  \hline
 qqtoken  & 字符串 & 可选  &  如果分享到QQ空间,该用户的QQtoken\\
 \hline
\end{tabular}
   \end{center}
\end{table}

注意:

\begin{enumerate}
\item 该请求必须是POST请求,enctype="multipart/form-data"。
\item 所上传文件的type必须是image/jpeg', 'image/gif' 或者'image/png'。
\item 图片文件要在客户端先压缩到640*640。
\item 图片名为logo;缩略图分logo\_thumb2为200*200的png。
\item 分享到新浪微博和QQ空间由服务器执行,客户端只需要传递标志位即可;而分享到微信则由客户端实现,然后报告服务器分享的结果。
\end{enumerate}

例子:

\begin{figure}[H]
\begin{verbatim}
访问:curl -F "photo[img]=@firefox.png;type=image/png" 
/photos

{"_id":"509a1ffcbe4b1982a4000002",
"weibo":false,"room":"1",
"user_id":"502e61bfbe4b1921da000005",
"img_tmp":"20121107-1646-42114-4251/111.jpg",
"logo":"/uploads/tmp/20121107-1646-42114-4251/111.jpg",
"logo_thumb2":null,"id":"509a1ffcbe4b1982a4000002"}

当返回的响应中包含img_tmp字段,且logo_thumb2为null时,说明此时图片上传到了服务器,但是还未启动后台处理。后台异步处理图片,包括生成缩略图以及上传到阿里云。

\end{verbatim}
\end{figure}


\subsection{发送消息}
当图片发送完成后,服务器端发送一个xmpp消息,其中的body内容格式为:

\begin{verbatim}
[img:$id]$desc

比如上面的图片发送成功后,发送消息"[img:502fd10abe4b19ed48000002]"。其中$desc是可选的图片说明文字。
\end{verbatim}

\subsection{在聊天室收到消息后获取图片}

接收端收到消息后,判断body的内容是否符合图片消息的格式。如果符合就调用下面的接口获取图片。

\begin{table}[H]
   \begin{center}
\begin{tabular}{|c|c|c|p{12cm}|}
\hline
GET & \multicolumn{3}{|c|}{/photos/show} \\
\hline\hline
 \  参数  & 类型 & 必填 &  说明  \\
  \hline
 id  & 字符串 & 必须 & 图片id\\
\hline
 size  & 整数 & 可选 &  目前0代表原图;2代表thumb2缩略图,大小为200*200。默认为2。\\ 
\hline
\end{tabular}
   \end{center}
\end{table}

聊天室里发图不需要发送已读回执消息。
在现场收到图片时,如果图片id以“faq”开头,不添加到照片墙上。18:02:24


\subsection{在聊天室收点击图片获取图片详细信息}

\begin{table}[H]
   \begin{center}
\begin{tabular}{|c|c|c|p{12cm}|}
\hline
GET & \multicolumn{3}{|c|}{/photos/detail} \\
\hline\hline
 \  参数  & 类型 & 必填 &  说明  \\
  \hline
 id  & 字符串 & 必须 & 图片id\\
\hline
\end{tabular}
   \end{center}
\end{table}



\section{照片墙}

\subsection{删除聊天室图片}

\begin{table}[H]
   \begin{center}
\begin{tabular}{|c|c|c|p{12cm}|}
\hline
POST & \multicolumn{3}{|c|}{/photos/delete} \\
\hline\hline
 \  参数  & 类型 & 必填 &  说明  \\
  \hline
 id  & 字符串 & 必须 & 图片id\\
\hline
\end{tabular}
   \end{center}
\end{table}
个人只能删除自己发布的照片。



\subsection{对聊天室图片点赞}

\begin{table}[H]
   \begin{center}
\begin{tabular}{|c|c|c|p{12cm}|}
\hline
POST & \multicolumn{3}{|c|}{/photos/like} \\
\hline\hline
 \  参数  & 类型 & 必填 &  说明  \\
  \hline
 id  & 字符串 & 必须 & 图片id\\
\hline
\end{tabular}
   \end{center}
\end{table}

\subsection{对聊天室图片取消赞}

\begin{table}[H]
   \begin{center}
\begin{tabular}{|c|c|c|p{12cm}|}
\hline
POST & \multicolumn{3}{|c|}{/photos/dislike} \\
\hline\hline
 \  参数  & 类型 & 必填 &  说明  \\
  \hline
 id  & 字符串 & 必须 & 图片id\\
\hline
\end{tabular}
   \end{center}
\end{table}
个人只能取消自己给的赞。


\subsection{对聊天室图片点评}

\begin{table}[H]
   \begin{center}
\begin{tabular}{|c|c|c|p{12cm}|}
\hline
POST & \multicolumn{3}{|c|}{/photos/comment} \\
\hline\hline
 \  参数  & 类型 & 必填 &  说明  \\
  \hline
 id  & 字符串 & 必须 & 图片id\\
   \hline
 text  & 字符串 & 必须 & 评论的内容\\
  \hline
 sid  & 字符串 & 可选 & 当前用户所在现场的id\\ 
\hline
\end{tabular}
   \end{center}
\end{table}

[2013-9-24] 新增sid,表示发表评论的用户所在现场的id。同时,如果评论者的sid和图片的商家id一致,那么会在聊天室中发布一条动态消息,消息id以COMMENT开头。评论消息的格式和普通图片消息一致。

\begin{figure}[H]
\begin{verbatim}
返回的数据结构是评论的数组:
 id: 评论人的id
 name: 评论人的名字
 txt:评论的内容
 t:时间
 rid:被回复者的id
 rname: 被回复者的名字
\end{verbatim}
\end{figure}

\subsection{对聊天室图片点评进行回复}

\begin{table}[H]
   \begin{center}
\begin{tabular}{|c|c|c|p{12cm}|}
\hline
POST & \multicolumn{3}{|c|}{/photos/recomment} \\
\hline\hline
 \  参数  & 类型 & 必填 &  说明  \\
  \hline
 id  & 字符串 & 必须 & 图片id\\
 \hline
 rid  & 字符串 & 必须 & 被回复者的id\\
   \hline
 text  & 字符串 & 必须 & 评论的内容\\
  \hline
 sid  & 字符串 & 可选 & 当前用户所在现场的id\\  
\hline
\end{tabular}
   \end{center}
\end{table}
这里的rid是被回复者的id,而发布回复者的id是从session中自动获取的。

这里的回复其实是针对人的,如果一个人在一张图片下发布了多个评论,那么这里的回复不针对特定的那条评论。

[2013-9-24] 新增sid,表示发表评论的用户所在现场的id。


\subsection{对聊天室图片删除点评}

\begin{table}[H]
   \begin{center}
\begin{tabular}{|c|c|c|p{12cm}|}
\hline
POST & \multicolumn{3}{|c|}{/photos/delcomment} \\
\hline\hline
 \  参数  & 类型 & 必填 &  说明  \\
  \hline
 id  & 字符串 & 必须 & 图片id\\
  \hline
 text  & 字符串 & 必须 & 评论的内容\\
\hline
\end{tabular}
   \end{center}
\end{table}
个人只能删除自己发的点评。

评论的删除是区分两种情况:删除/隐藏,评论的发布人可以删除,图片的发布人可以隐藏。
对第三方来说都是删除。

如果一个评论是我发的,同时图也是我发的,那么此时对评论操作就是删除,不需要隐藏。


\subsection{对聊天室图片隐藏点评}

\begin{table}[H]
   \begin{center}
\begin{tabular}{|c|c|c|p{12cm}|}
\hline
POST & \multicolumn{3}{|c|}{/photos/hidecomment} \\
\hline\hline
 \  参数  & 类型 & 必填 &  说明  \\
  \hline
 id  & 字符串 & 必须 & 图片id\\
  \hline
 uid  & 字符串 & 必须 & 评论的发布者id\\
   \hline
 text  & 字符串 & 必须 & 评论的内容\\
\hline
\end{tabular}
   \end{center}
\end{table}
只有图片的发布者才能隐藏他人给自己图片发的点评。调用本接口的必须是图片的发布人。
点评隐藏后只有点评的发布者一人能够看到。


\subsection{我的照片墙}
用户的照片墙由其本人在各个现场照片墙发布的照片组成。

\begin{table}[H]
   \begin{center}
\begin{tabular}{|c|c|c|p{12cm}|}
\hline
GET & \multicolumn{3}{|c|}{/photos/me} \\
\hline\hline
 \  参数  & 类型 & 必填 &  说明  \\
\hline
 page  & 整数 & 可选 & 分页,缺省值为1\\ 
 \hline
 pcount  & 整数 & 可选 & 每页的数量,缺省为20\\ 
\hline
\end{tabular}
   \end{center}
\end{table}

输出格式参见:\ref{photowall} 。唯一不同点在于:
商家照片墙中包含user\_name,指明该照片是谁发的。
用户照片墙中包含shop\_name,指明该照片是在哪儿拍摄的。


\subsection{我的评论}
获得当前登录用户的评论的照片。

\begin{table}[H]
   \begin{center}
\begin{tabular}{|c|c|c|p{12cm}|}
\hline
GET & \multicolumn{3}{|c|}{/photos/my\_comments} \\
\hline\hline
 \  参数  & 类型 & 必填 &  说明  \\
\hline
 page  & 整数 & 可选 & 分页,缺省值为1\\ 
 \hline
 pcount  & 整数 & 可选 & 每页的数量,缺省为20\\ 
\hline
\end{tabular}
   \end{center}
\end{table}

输出格式同上。


\subsection{查看其它用户的照片墙}
其它用户的照片墙由其该在各个现场照片墙发布并分享到微博/qq的照片组成。没有分享过的图片看成只在该现场才能看到的图片。

\begin{table}[H]
   \begin{center}
\begin{tabular}{|c|c|c|p{12cm}|}
\hline
GET & \multicolumn{3}{|c|}{/photos/users} \\
\hline\hline
 \  参数  & 类型 & 必填 &  说明  \\
 \hline
 uid  & 整数 & 必须 & 被查看用户的id\\ 
\hline
 page  & 整数 & 可选 & 分页,缺省值为1\\ 
 \hline
 pcount  & 整数 & 可选 & 每页的数量,缺省为20\\ 
\hline
\end{tabular}
   \end{center}
\end{table}

输出格式同上。



\subsection{其它用户分享到微信}
看到别人发的照片/问答,分享到自己的微信朋友圈或者微信好友,在客户端完成分享后,调用本接口。

\begin{table}[H]
   \begin{center}
\begin{tabular}{|c|c|c|p{12cm}|}
\hline
GET & \multicolumn{3}{|c|}{/photos/re\_share} \\
\hline\hline
 \  参数  & 类型 & 必填 &  说明  \\
 \hline
 user\_id  & 字符串 & 必须 & 分享者的用户id\\ 
  \hline
 photo\_id  & 字符串 & 必须 & 被分享的图片id\\ 
 \hline
 weibo  & 0/1 & 可选  &  是否同步发送到微博\\
  \hline
 qq  & 0/1 & 可选  &  是否同步发送到QQ空间\\
  \hline
 wx  & 0/1/2/3 & 可选  &  分享到微信: 1个人,2朋友圈, 3都分享了\\
\hline
\end{tabular}
   \end{center}
\end{table}

\begin{verbatim}
本接口主要用于统计用户和照片的分享数量。分享到微信时,采用url类型,url地址为“http://www.dface.cn/web_photo/show?id=图片id&uid=用户id”。
\end{verbatim}



\section{个人聊天时发图}

\subsection{个人聊天时上传图片}

\begin{table}[H]
   \begin{center}
\begin{tabular}{|c|c|c|p{12cm}|}
\hline
POST & \multicolumn{3}{|c|}{/photo2s/create} \\
\hline\hline
 \  参数  & 类型 & 必填 &  说明  \\
\hline
 photo[img]  & file & 必须 &  头像文件\\
 \hline
 photo[to\_uid]  & 字符串 & 必须 &  接收人的id\\
\hline
  photo[t]  & 整数 & 可选 &  图片类型:1拍照;2选自相册\\
 \hline
\end{tabular}
   \end{center}
\end{table}


\subsection{发送消息}
当图片发送完成后,服务器端发送一个xmpp消息,其中的body内容格式为:

\begin{verbatim}
[img:$id]

\end{verbatim}

\subsection{在个人聊天时收到消息后获取图片}
\begin{table}[H]
   \begin{center}
\begin{tabular}{|c|c|c|p{12cm}|}
\hline
GET & \multicolumn{3}{|c|}{/photo2s/show} \\
\hline\hline
 \  参数  & 类型 & 必填 &  说明  \\
  \hline
 id  & 字符串 & 必须 & 图片id\\
\hline
 size  & 整数 & 可选 &  目前0代表原图;2代表thumb2缩略图,大小为200*200。默认为2。\\ 
\hline
\end{tabular}
   \end{center}
\end{table}

注意:聊天时发图,取消的100*100大小的缩略图。[2012-11-7]


\subsection{消息状态}
和普通的文字消息一样,接收者收到图片后发送状态更新消息(收到/已读)给发送者。发送者将状态显示在图片旁边。



\section{个人聊天时发语音}

\subsection{个人聊天时上传语音}

\begin{table}[H]
   \begin{center}
\begin{tabular}{|c|c|c|p{12cm}|}
\hline
POST & \multicolumn{3}{|c|}{/sound2s/create} \\
\hline\hline
 \  参数  & 类型 & 必填 &  说明  \\
\hline
 sound[img]  & file & 必须 &  语音文件\\
 \hline
 sound[to\_uid]  & 字符串 & 必须 &  接收人的id\\
\hline
  sound[sec]  & 整数 & 可选 &  语音长度:秒\\
 \hline
\end{tabular}
   \end{center}
\end{table}


\subsection{发送消息}
当语音发送完成后,服务器端发送一个xmpp消息,其中的body内容格式为:

\begin{verbatim}
[sound:$id]$sec

\end{verbatim}

\subsection{在个人聊天时收到消息后获取语音}
\begin{table}[H]
   \begin{center}
\begin{tabular}{|c|c|c|p{12cm}|}
\hline
GET & \multicolumn{3}{|c|}{/sound2s/show} \\
\hline\hline
 \  参数  & 类型 & 必填 &  说明  \\
  \hline
 id  & 字符串 & 必须 & 语音id\\
\hline
\end{tabular}
   \end{center}
\end{table}





\section{用户推荐}
\subsection{ 将个人推荐给好友}
\begin{table}[H]
   \begin{center}
\begin{tabular}{|c|c|c|p{12cm}|}
\hline
POST & \multicolumn{3}{|c|}{/follows/recommend} \\
\hline\hline
 \  参数  & 类型 & 必填 &  说明  \\
\hline
 fid  & 字符串 & 必填 &  好友的用户id\\
 \hline
 uid  & 字符串 & 必填 &  被推荐的用户id\\
\hline
 mid  & 字符串 & 必填 &  客户端生成的UUID格式的消息ID\\
\hline
\end{tabular}
   \end{center}
\end{table}
当前登录用户给自己的好友fid推荐一个用户uid。
服务器端收到请求后,会发送下一节格式的Xmpp消息给好友。

\begin{figure}[H]
\begin{verbatim}
返回值:{"success" : mid}
\end{verbatim}
\end{figure}


\subsection{接收个人名片消息}
在单聊时,可以接收个人名片消息。该消息的格式如下:
\begin{verbatim}
[img:Logo$id:$uid]
其中id是个人头像的id,uid是个人的id.
\end{verbatim}
该消息格式兼容单聊时的图片消息的格式。


\section{优惠券}

\subsection{商家推送优惠券}
当用户使用脸脸时,比如签到/摇一摇等,可以收到商家推送过来的优惠券。优惠券也是一个xmpp消息,其中的body内容格式为:

\begin{verbatim}
[优惠券:$name:$shop:$id:$time:$hint]

 其中,name是优惠券的名称、shop是商家的名称,id是该优惠券的id,time是该张优惠券的下发时间。
 
 hint是可选的,表示使用优惠券时的提示语言,比如“请输入此次的消费金额”。有hint的优惠券使用时要求用户输入内容才能使用优惠券。内容默认是数字。
 
 [2013-11-13]优惠券消息增加一个seq属性,代表优惠券的编号。
 
优惠券id不重复。如果有重复的优惠券id时代表消息重发了,此时要忽略这条消息。
 
 所有优惠券都统一显示在会话中的“优惠券”文件夹中。优惠券的消息状态和其他消息同样处理。
\end{verbatim}


\subsection{获取优惠券列表}
当用户收到优惠券以后,先根据优惠券的Xmpp消息显示“优惠券生成中...”的效果,然后起http请求拿到优惠券列表。
优惠券Xmpp消息给的是优惠券的基本信息,本接口给的是优惠券的详细信息。

关于优惠券的排序,默认按时间排序,最新的在最前面。
相同时间同一批收到的优惠券,按距离排序。
当前所在商家的优惠券自动置顶。

\begin{table}[H]
   \begin{center}
\begin{tabular}{|c|c|c|p{12cm}|}
\hline
GET & \multicolumn{3}{|c|}{/coupons/list} \\
\hline\hline
 \  参数  & 类型 & 必填 &  说明  \\
\hline
 \  user\_id  & 字符串 & 必填 &  用户的id  \\
\hline
 \  status  & 数字 & 必填 &  优惠券的状态 ,1代表未使用的,2代表已使用的,4代表已过期的,8代表未激活的,0代表未使用和未激活的,默认为0\\
\hline
 lat  & 浮点数 & 可选 & 经度\\
\hline
 lng  &  浮点数 & 可选 & 纬度\\ 
 \hline
 \ sid  &  整数 & 可选 & 当前现场的商家id\\ 
  \hline
 page  & 整数 & 可选 & 分页,缺省值为1\\ 
 \hline
 pcount  & 整数 & 可选 & 每页的数量,缺省为10\\ 
\hline
\end{tabular}
   \end{center}
\end{table}


例子:

\begin{figure}[H]
\begin{verbatim}
访问:/coupons/list?
返回值:

[
{"id":"518de040c90d8be6840000b5","loc":[30.2425077887959,120.15320074395899]},
{"id":"5171d8d7c90d8b3c2900004a","loc":[30.282349,120.117012]}]

id是该优惠券的id, name是优惠券的名称、shop是商家的名称,dat是该张优惠券的下发时间, status是状态, hint是优惠券使用提示, seq是优惠券的流水号。
 
\end{verbatim}
\end{figure}





 

\subsection{优惠券展示}
优惠券以图片的方式展示,由服务器端生成该图片。当用户进入优惠券文件夹查看优惠券时,调用下面的接口获得优惠券的图片。

\begin{table}[H]
   \begin{center}
\begin{tabular}{|c|c|c|p{12cm}|}
\hline
GET & \multicolumn{3}{|c|}{/coupons/img} \\
\hline\hline
 \  参数  & 类型 & 必填 &  说明  \\
\hline
 \  id  & 字符串 & 必填 &  该优惠券的id  \\
\hline
 size  & 整数 & 可选 &  目前0代表原图(580, 224);1代表缩略图,大小为(290,112)。默认为1。\\ 
 \hline
 \  seq  & 整数 & 可选 &  0代码客户端没有seq,1代表客户端支持显示seq,默认为0 \\
\hline
\end{tabular}
   \end{center}
\end{table}
2.4以后的版本,都要传递seq=1,这样服务器端才不会重复生成seq.


\subsection{优惠券使用}
目前,所有优惠券都只支持单次使用,使用后即过期。当用户确认使用优惠券时,调用如下的接口:
\begin{table}[H]
   \begin{center}
\begin{tabular}{|c|c|c|p{12cm}|}
\hline
POST & \multicolumn{3}{|c|}{/coupons/use} \\
\hline\hline
 \  参数  & 类型 & 必填 &  说明  \\
\hline
 \  id  & 字符串 & 必填 &  该优惠券的id  \\
 \hline
 \ sid  &  整数 & 可选 & 当前现场的商家id\\  
 \hline
 \  data  & 字符串 & 可选 &  优惠券使用时填写的信息  \\
\hline
\end{tabular}
   \end{center}
\end{table}

优惠券的使用要保证只能点击一次,使用后就灰掉,不需要等待服务器端的返回结果。如果网络不通或者服务器端失败,要以一定的策略重试。

当优惠券有hint字段时,使用时带上消费者输入的data。

优惠券使用时,增加一个可选的参数sid,代表当前现场的商家id。如果用户没有摇入任何商家,sid不传。

\subsection{优惠券删除}


\begin{table}[H]
   \begin{center}
\begin{tabular}{|c|c|c|p{12cm}|}
\hline
GET & \multicolumn{3}{|c|}{/coupons/delete} \\
\hline\hline
 \  参数  & 类型 & 必填 &  说明  \\
\hline
 \  id  & 字符串 & 必填 &  该优惠券的id  \\
\hline
 \  user\_id  & 字符串 & 必填 &  用户的id  \\ 
\hline
\end{tabular}
   \end{center}
\end{table}


\subsection{优惠券批量使用}
本接口用于数据同步。

\begin{table}[H]
   \begin{center}
\begin{tabular}{|c|c|c|p{12cm}|}
\hline
GET & \multicolumn{3}{|c|}{/coupons/batch\_use} \\
\hline\hline
 \  参数  & 类型 & 必填 &  说明  \\
\hline
 \  infos  & 字符串 & 必填 &  批量使用信息 \\
\hline
\end{tabular}
   \end{center}
\end{table}
不同的优惠券以,分割,同一张优惠券内发送三个参数id-sid-data。



\subsection{优惠券批量删除}

\begin{table}[H]
   \begin{center}
\begin{tabular}{|c|c|c|p{12cm}|}
\hline
GET & \multicolumn{3}{|c|}{/coupons/batch\_delete} \\
\hline\hline
 \  参数  & 类型 & 必填 &  说明  \\
\hline
 \  ids  & 字符串 & 必填 &  该优惠券的id,以“,”分割  \\
\hline
\end{tabular}
   \end{center}
\end{table}




\section{XMPP服务器的Web接口}




\subsection{获得聊天室的消息历史记录}

\begin{table}[H]
   \begin{center}
\begin{tabular}{|c|c|c|p{12cm}|}
\hline
GET & \multicolumn{3}{|c|}{http://xmpp\_ip:5280/api/gchat} \\
\hline\hline
 \  参数  & 类型 & 必填 &  说明  \\
\hline
 \  room  & 字符串 & 必填 &  聊天室的房间id  \\
\hline
\end{tabular}
   \end{center}
\end{table}
返回值的数据结构:
[发送者ID,消息内容,以秒表示的时间,消息ID]


\subsection{获得聊天室的消息历史记录2}

\begin{table}[H]
   \begin{center}
\begin{tabular}{|c|c|c|p{12cm}|}
\hline
GET & \multicolumn{3}{|c|}{http://xmpp\_ip:5280/api/gchat2} \\
\hline\hline
 \  参数  & 类型 & 必填 &  说明  \\
\hline
 \  room  & 字符串 & 必填 &  聊天室的房间id  \\
\hline
 \  skip  & 数字 & 必填 &  跳过多少条  \\
\hline
 \  count  & 数字 & 必填 &  取多少条  \\  
\hline
\end{tabular}
   \end{center}
\end{table}
注意:skip=0时,消息历史用ejabberd的内存中获取;skip大于0时,从数据库取。

返回值的数据结构:
[发送者ID,消息内容,以秒表示的时间,消息ID]




\subsection{获得个人聊天的历史记录}

\begin{table}[H]
   \begin{center}
\begin{tabular}{|c|c|c|p{12cm}|}
\hline
GET & \multicolumn{3}{|c|}{http://xmpp\_ip:5280/api/chat} \\
\hline\hline
 \  参数  & 类型 & 必填 &  说明  \\
\hline
 \  uid  & 字符串 & 必填 &  用户的id  \\
\hline
\end{tabular}
   \end{center}
\end{table}
返回值的数据结构:
[接受者ID,消息内容,以秒表示的时间,消息ID]


\subsection{获得两人之间的聊天历史记录}

\begin{table}[H]
   \begin{center}
\begin{tabular}{|c|c|c|p{12cm}|}
\hline
GET & \multicolumn{3}{|c|}{http://xmpp\_ip:5280/api/chat2} \\
\hline\hline
 \  参数  & 类型 & 必填 &  说明  \\
\hline
 \  uid1  & 字符串 & 必填 &  用户1的id  \\
\hline
 \  uid2  & 字符串 & 必填 &  用户2的id  \\
\hline
\end{tabular}
   \end{center}
\end{table}

返回值的数据结构:
[发送者ID, 接受者ID,消息内容,以秒表示的时间,消息ID]


\subsection{其它内部接口}
仅用于服务器端的后台内部调用。

\subsubsection{以给定用户身份给指定的聊天室发送消息}

\begin{table}[H]
   \begin{center}
\begin{tabular}{|c|c|c|p{12cm}|}
\hline
POST & \multicolumn{3}{|c|}{http://xmpp\_ip:5280/api/room} \\
\hline\hline
 \  参数  & 类型 & 必填 &  说明  \\
\hline
 \  roomid  & 字符串 & 必填 &  聊天室的房间id  \\
\hline
 \  message  & 字符串 & 必填 &  需要发送的消息  \\
\hline
 \  uid  & 字符串 & 必填 &  发送此消息的用户id  \\
\hline
\end{tabular}
   \end{center}
\end{table}
这条消息会发送给聊天室的所有人,比如用户进入聊天室时的打招呼。
如果是给聊天室的某个人发消息,比如优惠券中奖/公告消息,则通过rest接口发送groupchat类型的消息。

\subsubsection{用户屏蔽通讯}

\begin{table}[H]
   \begin{center}
\begin{tabular}{|c|c|c|p{12cm}|}
\hline
POST & \multicolumn{3}{|c|}{http://xmpp\_ip:5280/api/block} \\
\hline\hline
 \  参数  & 类型 & 必填 &  说明  \\
\hline
 \  uid  & 字符串 & 必填 &  发起屏蔽请求的用户的id  \\
\hline
 \  bid  & 字符串 & 必填 &  被屏蔽的用户的id  \\
\hline
\end{tabular}
   \end{center}
\end{table}

\subsubsection{用户解除屏蔽}

\begin{table}[H]
   \begin{center}
\begin{tabular}{|c|c|c|p{12cm}|}
\hline
POST & \multicolumn{3}{|c|}{http://xmpp\_ip:5280/api/unblock} \\
\hline\hline
 \  参数  & 类型 & 必填 &  说明  \\
\hline
 \  uid  & 字符串 & 必填 &  发起解除屏蔽请求的用户的id  \\
\hline
 \  bid  & 字符串 & 必填 &  被屏蔽的用户的id  \\
\hline
\end{tabular}
   \end{center}
\end{table}


\subsubsection{获取用户屏蔽列表}

\begin{table}[H]
   \begin{center}
\begin{tabular}{|c|c|c|p{12cm}|}
\hline
POST & \multicolumn{3}{|c|}{http://xmpp\_ip:5280/api/blocklist} \\
\hline\hline
 \  参数  & 类型 & 必填 &  说明  \\
\hline
 \  uid  & 字符串 & 必填 &  发起解除屏蔽请求的用户的id  \\
\hline
\end{tabular}
   \end{center}
\end{table}


\subsubsection{杀掉用户会话}

\begin{table}[H]
   \begin{center}
\begin{tabular}{|c|c|c|p{12cm}|}
\hline
POST & \multicolumn{3}{|c|}{http://xmpp\_ip:5280/api/kill} \\
\hline\hline
 \  参数  & 类型 & 必填 &  说明  \\
\hline
 \  user  & 字符串 & 必填 &  被封杀的用户的id  \\
\hline
\end{tabular}
   \end{center}
\end{table}


\subsubsection{禁止用户在聊天室发言}

\begin{table}[H]
   \begin{center}
\begin{tabular}{|c|c|c|p{12cm}|}
\hline
POST & \multicolumn{3}{|c|}{http://xmpp\_ip:5280/api/room\_ban} \\
\hline\hline
 \  参数  & 类型 & 必填 &  说明  \\
\hline
 \  roomid  & 字符串 & 必填 & 聊天室id \\
 \hline
 \  uid  & 字符串 & 必填 &  被封杀的用户的id  \\
\hline
\end{tabular}
   \end{center}
\end{table}


\subsubsection{解除禁止用户在聊天室发言}

\begin{table}[H]
   \begin{center}
\begin{tabular}{|c|c|c|p{12cm}|}
\hline
POST & \multicolumn{3}{|c|}{http://xmpp\_ip:5280/api/room\_unban} \\
\hline\hline
 \  参数  & 类型 & 必填 &  说明  \\
\hline
 \  roomid  & 字符串 & 必填 & 聊天室id \\
 \hline
 \  uid  & 字符串 & 必填 &  被封杀的用户的id  \\
\hline
\end{tabular}
   \end{center}
\end{table}



\subsubsection{发送任意XMPP消息}

\begin{table}[H]
   \begin{center}
\begin{tabular}{|c|c|c|p{12cm}|}
\hline
POST & \multicolumn{3}{|c|}{http://xmpp\_ip:5280/rest} \\
\hline\hline
 \  参数  & 类型 & 必填 &  说明  \\
\hline
\end{tabular}
   \end{center}
\end{table}

通过POST提交任意类型的XMPP文本消息。



\section{关于Push消息}

\subsection{登录Xmpp服务器时,报告设备的Push消息token}
登录Xmpp服务器,发送的密码格式变更为:

\begin{verbatim}
“用户密码”+《状态码》+token

状态码为:1代表ios开发设备,2代表实际的ios发布设备, 3代表android个推的clientid, 4代表百度云推送.
对于iOS设备,token是一个64位的字符串。对于android设备,如果token小于64,以0填充尾部到64位。百度云推送的token格式为:channel_id,user_id,补齐0到64位)

比如某用户的密码是“c53b2f16a24c61d9”,
token是“ea6e4f05b48f4a057816956e567c6feacf14ee28cbbbab67993e263c3dfa2c27”,
目前是进行开发测试,那么登录Xmpp服务器时的密码为:
“c53b2f16a24c61d91ea6e4f05b48f4a057816956e567c6feacf14ee28cbbbab67993e263c3dfa2c27”。
\end{verbatim}

服务器端解析出状态码和token并保存。


\subsection{退出时,报告设备的Push消息token}

用户退出的API接口新增pushtoken参数,以取消Push消息token绑定。
如果是个推,退出时要接触和个推的绑定。

\subsection{Push消息的发送}
当有离线消息产生时,xmpp服务器获得用户的状态码以决定给那个Apple Push服务器发消息,以及获得和该用户绑定的token。一个用户只能绑定一个token。




\section{帮助链接}

\begin{table}[H]
   \begin{center}
\begin{tabular}{|c|c|c|p{12cm}|}
\hline
GET & \multicolumn{3}{|c|}{/help/index} \\
\hline\hline
 \  参数  & 类型 & 必填 &  说明  \\
 \hline
 ver  & 字符串 & 必须 & 软件版本\\
\hline
 os  & 浮点数 & 必须 & 操作系统\\
\hline
\end{tabular}
   \end{center}
\end{table}

本接口需要通过浏览器以URL的方式直接调用。



\section{关于程序各页面之间的切换规则}

程序的首页是选择现场页面。现场则是一个聊天室页面。

1、程序干净启动时,要判断用户是否处于登录状态。

1.1如果已经登录,进入定位页面,并从服务器端获取新的商家列表。

1.2如果没有登录,进入登录页面。


2、程序从睡眠状态被激活时,要判断距离上次的运行时间。

2.1 如果不足12个小时,上次被任务切换时停留在那个页面就仍然停留在该页面。

2.2 如果超过12个小时,进入定位页面,并从服务器端获取新的商家列表。



4、从其它tab点击现场时,判断用户是否摇进过现场

4.1 如果摇进过某个现场,直接进入该现场聊天室。

4.2 如果没有摇,那么就是定位页面。




\newpage


\end{document}
