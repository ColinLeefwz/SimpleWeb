\documentclass[cs4size]{ctexartutf8} 
\usepackage[unicode={true}]{hyperref}
\hypersetup{colorlinks,%
                 citecolor=black,%
                 filecolor=black,%
                 linkcolor=black,%
                 urlcolor=blue,%
                 pdftex}

\usepackage{graphicx}
\usepackage{float}

				
\author{YXY}
\title{脸脸网服务器与手机端的开发接口API}

\begin{document} 

\maketitle
\tableofcontents

\newpage

\textbf{注意事项}:
\begin{enumerate}
\item 所有的链接,如果要获得json格式的数据,最好带上".json"的后缀。因为同一个链接以后可能返回多种格式的数据,比如xml/html/json等。
\item 当调用出错时,会在返回的json中包含"error"信息。
\item 只有开发调试时调用http的接口,发布时全部使用https接口。
\end{enumerate}

\newpage

\section{初始化}
手机应用启动后,初次访问时调用此API。
\begin{table}[H]
   \begin{center}
\begin{tabular}{|c|c|c|p{12cm}|}
\hline
\multicolumn{4}{|c|}{https://60.191.119.190/init/init} \\
\hline\hline
 \  参数  & 类型 & 必填 &  说明  \\
 \hline
 model  & 字符串 & 必须 &  硬件型号\\
\hline
 os  & 字符串 & 必须 &  手机操作系统版本\\
 \hline
 mac  & 字符串 & 必须 &  无线网卡MAC地址的MD5\\
 \hline
 hash  & 字符串 & 必须 &  hash验证码\\
  \hline
 ver  & 字符串 & 可选 &  软件版本\\
\hline
\end{tabular}
   \end{center}
\end{table}

注意:本API直接通过IP地址“60.191.119.190”访问,不通过DNS。而后续所有API调用的IP地址根据本调用的返回的IP地址确定。在目前只有单台服务器的情况下,返回的ip地址也是“60.191.119.190”,但是以后会根据离用户的距离返回就近的IP地址。


hash的计算方式为:model+os+mac+"init"进行SHA1算法后取前32位,和登录时的hash算法[\ref{hash_algorithm}]类似。

\begin{verbatim}
返回值:
{
"ip": "60.191.119.190",
"xmpp": "60.191.119.190"
}

后续的http访问使用返回的ip地址,xmpp协议的访问使用xmpp地址。

\end{verbatim}

本API的作用一个是给客户端推荐IP地址,一个是统计应用的开机情况和操作系统分布。




\section{登录和退出}
\subsection{老的登录方式:oauth2}


\begin{table}[H]
   \begin{center}
\begin{tabular}{|c|c|c|p{12cm}|}
\hline
\multicolumn{4}{|c|}{/oauth2/sina\_callback} \\
\hline\hline
 \  参数  & 类型 & 必填 &  说明  \\
\hline
 code  & 字符串 & 必须 &  新浪返回的code\\
\hline
\end{tabular}
   \end{center}
\end{table}


例子:

\begin{figure}[H]
\begin{verbatim}
访问:/oauth2/sina_callback?code=...
该访问由新浪微博通过302转向发起。
返回值:

{
"id":1,
"password":"15c663b8ff620502",
"logo":"",
"name" : "name"
"gender" : 1
"wb_uid":"1884834632",
"expires_in":75693,
"token":"2.00aaZYDCMcnDPCb4dc439e06i2m_GC",
"expires_at":1338431259
}

其中,id和password是该用户在脸脸网的id和密码,在首次新浪授权时创建。该id加上"@dface.cn"就是openfire的jid。(目前还没有和jid关联上。)

后面几个字段都是新浪提供的,其中wb_uid是该用户的新浪微博的uid,token是授权的access_token,其它几个是授权有效期信息。

在首次登录获得用户的id和password以后,可以缓存到本地。以后用户每次登录,返回的id和password是不会改变的,除非服务器端重置密码(比如出现密码泄漏等安全情况)。

logo可以得知用户是否成功上传头像。对于第一次登录的用户,其logo为空,此时需要引导用户上传头像。


\end{verbatim}
\end{figure}

\subsection{新的登录方式:xauth}
\label{hash_algorithm}

\begin{table}[H]
   \begin{center}
\begin{tabular}{|c|c|c|p{12cm}|}
\hline
\multicolumn{4}{|c|}{/oauth2/login} \\
\hline\hline
 \  参数  & 类型 & 必填 &  说明  \\
\hline
 name  & 字符串 & 必须 &  用户名\\
 \hline
 pass  & 字符串 & 必须 &  密码\\
  \hline
 mac  & 字符串 & 必须 &  网卡Mac地址\\
 \hline
 hash  & 字符串 & 必须 &  hash验证码\\
\hline
\end{tabular}
   \end{center}
\end{table}

本接口的输出和oauth2登录接口的输出相同。

\begin{verbatim}
hash的计算方式为:name+pass+mac+"dface"进行SHA1算法后取前32位。
比如用户名为name,密码为pa,mac地址为ss
那么待hash的字符串为"namepassdface",
其hash码为"11f4004a73a65117071bc4a7d3dfdf07"。
\end{verbatim}

新的登录方式不需要调用新浪的Xauth API,直接调用本API即可以登录。客户端应该不需要保存AppSecret。通过预先保存的AppKey和本调用输出中包含的token应该就可以访问新浪微博的API了。



\subsection{退出}

\begin{table}[H]
   \begin{center}
\begin{tabular}{|c|c|c|p{12cm}|}
\hline
\multicolumn{4}{|c|}{/oauth2/logout} \\
\hline\hline
 \  参数  & 类型 & 必填 &  说明  \\
\hline
\end{tabular}
   \end{center}
\end{table}

退出调用不需要参数。
当用户主动退出时,调用本接口并断开XMPP的连接。

\section{根据IP地址获得可能的现场商家}

\begin{table}[H]
   \begin{center}
\begin{tabular}{|c|c|c|p{12cm}|}
\hline
\multicolumn{4}{|c|}{/aroundme/shops\_by\_ip} \\
\hline\hline
 \  参数  & 类型 & 必填 &  说明  \\
\hline
 ip  & 字符串 & 可选 & 默认为访问者的IP地址\\
\hline
\end{tabular}
   \end{center}
\end{table}


最多返回10条数据,同一个IP地址商家应该不多。

定位时不再需要本接口,直接调用根据经纬度获得可能的现场商家的接口。




\section{根据经纬度获得可能的现场商家}

\begin{table}[H]
   \begin{center}
\begin{tabular}{|c|c|c|p{12cm}|}
\hline
\multicolumn{4}{|c|}{/aroundme/shops} \\
\hline\hline
 \  参数  & 类型 & 必填 &  说明  \\
\hline
 lat  & 浮点数 & 必须 & 经度\\
\hline
 lng  &  浮点数 & 必须 & 纬度\\ 
\hline
 accuracy  & 整数 & 必须 & 经纬度的精确度\\ 
\hline
\end{tabular}
   \end{center}
\end{table}

注意:一定要在accuracy小于200m或者调用获取GPS的API超过三次后才能调用本接口。

Gps获取位置是一个逐步精确的过程。很多时候首次获取的误差超过1000米。此时显然无法准确获得商家列表。此时建议等待0.5秒再次获取gps。要保证gps的精确度小于200m,或则定位三次以上确实不能更准确了。宁可等待的时间久一些,也要保证定位的准确性。

最多返回50条数据,也有可能没有数据(比如西部沙漠地区)。

例子:

\begin{figure}[H]
\begin{verbatim}
访问:/aroundme/shops.json
返回值:

[
	{
	"name":"新紫轩花店",
	"address":"文一路266号",
	"lng":120.12763,
	"id":40056,
	"phone":"",
	"lat":30.28691,
	"lo":[30.297,120.1288],
	"t":1,
	"user":0,
	"male",0,
	"female",0
	}
]

除了返回商家的id、名称、电话、经纬度以后,增加了在该商家的用户总数“user”、男“male”、女“female”数量。

[2012-8-7] 新增lo字段。这是商家的实际经纬度,以前的lat/lng是地图上显示的经纬度,两者有几百米的误差。计算商家和自己的距离时,要采用lo来计算。

[2012-8-8] 新增t字段,表示商家的类型。

\end{verbatim}
\end{figure}



\section{根据经纬度获得附近的商家}

\begin{table}[H]
   \begin{center}
\begin{tabular}{|c|c|c|p{12cm}|}
\hline
\multicolumn{4}{|c|}{/shop/nearby} \\
\hline\hline
 \  参数  & 类型 & 必填 &  说明  \\
\hline
 lat  & 浮点数 & 必须 & 经度\\
\hline
 lng  &  浮点数 & 必须 & 纬度\\ 
\hline
 accuracy  & 整数 & 必须 & 经纬度的精确度\\ 
 \hline
 page  & 整数 & 可选 & 分页,缺省值为1\\ 
 \hline
 pcount  & 整数 & 可选 & 每页的数量,缺省为20\\ 
  \hline
 name  & 字符串 & 可选 & 商家的名称\\ 
  \hline
 type  & 整数 & 可选 & 商家类型\\  
\hline
\end{tabular}
   \end{center}
\end{table}

商家类型:
\begin{enumerate}
\item 酒吧• 活动
\item 咖啡• 茶馆   
\item 餐饮• 酒店
\item 休闲• 娱乐
\item 购物• 广场
\item 楼宇• 社区
\end{enumerate}


用户当前所在的现场只有一个,但是由于定位有误差,所以服务器返回最有可能的几个商家让用户选择;而附近的商家有很多,按距离由近到远排序,且支持查询、分类筛选和分页。






\section{根据经纬度获得附近的用户}

\begin{table}[H]
   \begin{center}
\begin{tabular}{|c|c|c|p{12cm}|}
\hline
\multicolumn{4}{|c|}{/aroundme/users} \\
\hline\hline
 \  参数  & 类型 & 必填 &  说明  \\
\hline
 lat  & 浮点数 & 必须 & 经度\\
\hline
 lng  &  浮点数 & 必须 & 纬度\\ 
\hline
 accuracy  & 整数 & 必须 & 经纬度的精确度\\ 
  \hline
 page  & 整数 & 可选 & 分页,缺省值为1\\ 
 \hline
 pcount  & 整数 & 可选 & 每页的数量,缺省为20\\ 
\hline
\end{tabular}
   \end{center}
\end{table}


例子:

\begin{figure}[H]
\begin{verbatim}
访问:/aroundme/users.json?lat=30.2829754&lng=120.1337336&accuracy=100
返回值:

[
{"id":1,
"logo":"/phone2/images/namei2.gif",
"name":"name23",
"wb_uid":1884834632,
"gender":0,
"friend":true,
"follower":false,
"birthday":null}
]

如果用户登录后访问该API,那么返回的属性中包括"friend"和"follower"。

"friend"表示该用户是否是当前用户的朋友;"follower"代表该用户是否关注了当前用户。


\end{verbatim}
\end{figure}


\section{根据商家ID获得商家的基本信息}

\begin{table}[H]
   \begin{center}
\begin{tabular}{|c|c|c|p{12cm}|}
\hline
\multicolumn{4}{|c|}{/shop/info} \\
\hline\hline
 \  参数  & 类型 & 必填 &  说明  \\
\hline
 id  & 整数 & 必须 & 商家的ID\\
\hline
\end{tabular}
   \end{center}
\end{table}


\section{根据商家ID获得当前在该商家的所有用户}

\begin{table}[H]
   \begin{center}
\begin{tabular}{|c|c|c|p{12cm}|}
\hline
\multicolumn{4}{|c|}{/shop/users} \\
\hline\hline
 \  参数  & 类型 & 必填 &  说明  \\
\hline
 id  & 整数 & 必须 & 商家的ID\\
\hline
\end{tabular}
   \end{center}
\end{table}

返回值格式和“根据经纬度获得附近的用户”基本一致,加上time字段。

[2012-8-8] 新增time字段,表示用户在该商家的最后活跃时间。


在商家的用户的统计数据根据签到来统计,而签到则是当用户进入商家聊天室时自动后台发送。当用户进入下一个商家时,自动判定为离开上一个商家。如果用户没有退出,那么默认12小时后用户离开商家。

在商家的用户的集合要保证大于在该商家聊天室的用户的集合。

当前的商家用户数据则完全是基于方便测试的目的构造的。



\section{根据商家ID获得商家的公告}

\begin{table}[H]
   \begin{center}
\begin{tabular}{|c|c|c|p{12cm}|}
\hline
\multicolumn{4}{|c|}{/shop\_notices} \\
\hline\hline
 \  参数  & 类型 & 必填 &  说明  \\
\hline
 id  & 整数 & 必须 & 商家的ID\\
\hline
\end{tabular}
   \end{center}
\end{table}

返回值:
\begin{verbatim}
[
{"id":1,"title":"asdf"},
{"id":2,"title":"a title"}
]
\end{verbatim}



\section{进入现场签到}

\begin{table}[H]
   \begin{center}
\begin{tabular}{|c|c|c|p{12cm}|}
\hline
\multicolumn{4}{|c|}{/checkins} \\
\hline\hline
 \  参数  & 类型 & 必填 &  说明  \\
\hline
 lat  & 浮点数 & 必须 & 经度\\
\hline
 lng  &  浮点数 & 必须 & 纬度\\ 
\hline
 accuracy  & 整数 & 必须 & 经纬度的精确度\\ 
\hline
 shop\_id  & 整数 & 必须 &  现场商家id\\ 
\hline
 user\_id  & 整数 & 必须 &  签到用户id\\ 
\hline
 od  & 整数 & 必须 &  实际签到的商家的排序\\  
\hline
 altitude  &  浮点数 & 可选 & 海拔高度\\ 
\hline
 altacc  & 整数 & 可选 & 海拔高度的精确度\\  
\hline
\end{tabular}
   \end{center}
\end{table}

注意:该请求必须是POST请求。如果是GET请求,获得的是签到列表。

[2012-8-10] 新增od字段,表示用户实际签到的商家在现场商家列表中的顺序位置,从1开始,也就是从/aroundme/shops中获得的商家列表中,那个被用户选中了。



\section{获得用户基本信息}

\begin{table}[H]
   \begin{center}
\begin{tabular}{|c|c|c|p{12cm}|}
\hline
\multicolumn{4}{|c|}{/user\_info/get} \\
\hline\hline
 \  参数  & 类型 & 必填 &  说明  \\
\hline
 id  & 整数 & 必须 &  用户id\\
\hline
\end{tabular}
   \end{center}
\end{table}



例子:

\begin{figure}[H]
\begin{verbatim}
访问:/user_info/get?id=1
返回值:

{
"name":null,
wb_uid:1884834632
"gender":null,
"logo":"/system/imgs/1/original/clojure.png?1339398140",
"logo_thumb":"/system/imgs/1/thumb/clojure.png?1339398140"
"birthday":null,
"friend":true,
"follower":false,
"signature":"","job":null,"jobtype":null,"hobby":"Weiqi, go",
"last": "1天以前 顺旺基(益乐路店)" 
"id":1
}

获得用户基本信息里有两个关系字段:friend和follower。这里的关系都是针对我的,这里的我就是当前登录用户。如果我关注了该用户,那么该用户就是我的friend; 如果该用户关注了我,那么他就是我的follower。

[2012-8-8] 新增签名、职业、爱好等字段。
[2012-8-8] 新增last字段,表示用户最后在脸脸上的时间和地点,一个以空格分割的字符串。

\end{verbatim}
\end{figure}


\section{获得用户的头像}

\begin{table}[H]
   \begin{center}
\begin{tabular}{|c|c|c|p{12cm}|}
\hline
\multicolumn{4}{|c|}{/user\_info/logo} \\
\hline\hline
 \  参数  & 类型 & 必填 &  说明  \\
\hline
 id  & 整数 & 必须 &  用户id\\
\hline
 size  & 整数 & 可选 &  目前0代表原图;1代表thumb缩略图,大小为75*75;2代表thumb2缩略图,大小为150*150。默认为1。\\ 
\hline
\end{tabular}
   \end{center}
\end{table}


输出的Response Header中包含“Img\_url”头,这个头就是查看用户信息时显示的用户头像的路径。

例如:“Img\_url:/system/imgs/1/thumb2/1.png?1339649979”

注意:采用阿里云存储后,会输出302重定向,而不是直接给二进制流。


\section{获得用户的图片列表}

\begin{table}[H]
   \begin{center}
\begin{tabular}{|c|c|c|p{12cm}|}
\hline
\multicolumn{4}{|c|}{/user\_info/photos} \\
\hline\hline
 \  参数  & 类型 & 必填 &  说明  \\
\hline
 id  & 整数 & 必须 &  用户id\\
\hline
\end{tabular}
   \end{center}
\end{table}


例子:

\begin{figure}[H]
\begin{verbatim}
访问:/user_info/photos?id=1
返回值:

[
{"logo_thumb2":"/system/imgs/2/thumb2/2.png.png?1340616951","updated_at":"2012-06-25T09:35:51Z",
"id":2,"logo":"/system/imgs/2/original/2.png.png?1340616951","img_file_size":81881,"user_id":1,"logo_thumb":"/system/imgs/2/thumb/2.png.png?1340616951"},
{"logo_thumb2":"/system/imgs/4/thumb2/2.png.png?1340617039","updated_at":"2012-06-25T09:37:20Z",
"id":4,"logo":"/system/imgs/4/original/2.png.png?1340617039","img_file_size":81881,"user_id":1,"logo_thumb":"/system/imgs/4/thumb/2.png.png?1340617039"}
]

图片列表中的第一张图片就是用户的头像。


\end{verbatim}
\end{figure}



\section{获得当前登录用户基本信息}

\begin{table}[H]
   \begin{center}
\begin{tabular}{|c|c|c|p{12cm}|}
\hline
\multicolumn{4}{|c|}{/user\_info/get\_self} \\
\hline\hline
 \  参数  & 类型 & 必填 &  说明  \\
\hline
\end{tabular}
   \end{center}
\end{table}

例子:

\begin{figure}[H]
\begin{verbatim}
访问:/user_info/get_self
返回值:

{
"name":null,
wb_uid:1884834632
"gender":null,
"birthday":null,
"invisible":0,
"logo":"/system/imgs/1/original/clojure.png?1339398140",
"logo_thumb":"/system/imgs/1/thumb/clojure.png?1339398140"
"password":"c84dad462d5b7282",
"id":1
}

\end{verbatim}
\end{figure}



\section{设置当前登录用户基本信息}

\begin{table}[H]
   \begin{center}
\begin{tabular}{|c|c|c|p{12cm}|}
\hline
\multicolumn{4}{|c|}{/user\_info/set} \\
\hline\hline
 \  参数  & 类型 & 必填 &  说明  \\
\hline
 name  & 字符串 & 选填 &  用户的名字. 长度小于64\\
\hline
 gender  & 数字 & 选填 &  用户的性别,未设置0男1女2\\
\hline
 birthday  & 字符串 & 选填 &  用户的生日,格式如2012-06-01\\
 \hline
 signature  & 字符串 & 选填 &  签名档. 长度小于255\\
 \hline
 job  & 字符串 & 选填 &  职业\\
 \hline
 jobtype  & 整数 & 选填 &  职业类别\\
 \hline
 hobby  & 字符串 & 选填 &  爱好. 长度小于255\\
 \hline
 invisible  & 数字 & 选填 &  0:不隐身,1:对陌生人隐身,2:全部隐身\\
\hline
\end{tabular}
   \end{center}
\end{table}

注意:该请求必须是POST请求。

用户初次登录时获取新浪微博的名称/性别/出生日期等信息并提交到服务端。以后更新时可以只更新其中的某个字段。

隐身的效果是:
1、在商家的用户列表中不在出现该用户
2、不更新自己的位置信息

例子:

\begin{figure}[H]
\begin{verbatim}
访问:curl -b "_session_id=ead9ac4f6291c55bb467ad4138eca2ed" 
        -F "name=newname" /user_info/set

返回值:

{
"name":newname,
"gender":null,
"birthday":null,
"logo":"/system/imgs/1/original/clojure.png?1339398140",
"logo_thumb":"/system/imgs/1/thumb/clojure.png?1339398140"
"id":1
}

\end{verbatim}
\end{figure}


\section{当前登录用户上传图片}

\begin{table}[H]
   \begin{center}
\begin{tabular}{|c|c|c|p{12cm}|}
\hline
\multicolumn{4}{|c|}{/user\_logos/create} \\
\hline\hline
 \  参数  & 类型 & 必填 &  说明  \\
\hline
 user\_logo[img]  & file & 必须 &  头像文件\\
\hline
\end{tabular}
   \end{center}
\end{table}

注意:

\begin{enumerate}
\item 该请求必须是POST请求,enctype="multipart/form-data"。
\item 所上传文件的type必须是image/jpeg', 'image/gif' 或者'image/png'。
\item 图片文件必须小于5M。
\item 图片名为logo;缩略图分为两种大小,logo\_thumb为75*75的png,logo\_thumb2为150*150的png。
\end{enumerate}

例子:

\begin{figure}[H]
\begin{verbatim}
访问:curl -F "user_logo[img]=@firefox.png;type=image/png" 
/user_logos


{"logo_thumb2":"/system/imgs/6/thumb2/csky2.png?1340779488",
"updated_at":"2012-06-27T06:44:48Z"
,"id":6,
"logo":"/system/imgs/6/original/csky2.png?1340779488",
"img_file_size":156604,
"user_id":3,
"logo_thumb":"/system/imgs/6/thumb/csky2.png?1340779488"}

\end{verbatim}
\end{figure}



\section{当前登录用户调整图片位置}

\begin{table}[H]
   \begin{center}
\begin{tabular}{|c|c|c|p{12cm}|}
\hline
\multicolumn{4}{|c|}{/user\_logos/position} \\
\hline\hline
 \  参数  & 类型 & 必填 &  说明  \\
\hline
 id  & 整数 & 必须 &  图片的id\\
 \hline
 order  & 整数 & 必须 &  新的位置,从0开始\\
\hline
\end{tabular}
   \end{center}
\end{table}

例子:

\begin{figure}[H]
\begin{verbatim}
访问:curl -F "id=2;order=0" 
/user_logos/position


\end{verbatim}
\end{figure}


\section{当前登录用户批量调整所有图片的位置}

\begin{table}[H]
   \begin{center}
\begin{tabular}{|c|c|c|p{12cm}|}
\hline
\multicolumn{4}{|c|}{/user\_logos/change\_all\_position} \\
\hline\hline
 \  参数  & 类型 & 必填 &  说明  \\
\hline
 ids  & 逗号分割的整数 & 必须 &  所有图片的id按新的循序排列\\
\hline
\end{tabular}
   \end{center}
\end{table}

说明:
当只有一到两张图片更改位置时,不能调用本接口,需要调用单独修改图片位置的接口。

比如原有图片的循序是2468,新的循序是2846,此时需要调用前一个接口,传递id=8\&order=1。此时服务器只需要更新一条记录。如果调用本接口,会导致服务器端所有图片都被更新。

比如原有图片的循序是2468,新的循序是8642,此时需要调用本接口,传递ids=8,6,4,2。此时服务器会更新所有图片的排序,而客户端只需要发送一个请求。

例子:

\begin{figure}[H]
\begin{verbatim}
访问:curl -F "ids=3,1,4" 
/user_logos/position




\end{verbatim}
\end{figure}



\section{当前登录用户删除图片}

\begin{table}[H]
   \begin{center}
\begin{tabular}{|c|c|c|p{12cm}|}
\hline
\multicolumn{4}{|c|}{/user\_logos/delete} \\
\hline\hline
 \  参数  & 类型 & 必填 &  说明  \\
\hline
 id  & 整数 & 必须 &  图片的id\\
\hline
\end{tabular}
   \end{center}
\end{table}

例子:

\begin{figure}[H]
\begin{verbatim}
访问:curl -F "id=2" 
/user_logos/delete

输出:
{"deleted":"2"}

\end{verbatim}
\end{figure}



\section{添加我关注的人}

\begin{table}[H]
   \begin{center}
\begin{tabular}{|c|c|c|p{12cm}|}
\hline
\multicolumn{4}{|c|}{/follows/create} \\
\hline\hline
 \  参数  & 类型 & 必填 &  说明  \\
\hline
 user\_id  & 整数 & 必须 &  当前登录用户的id\\
\hline
 follow\_id  & 整数 & 必须 &  关注用户的id\\
\hline
\end{tabular}
   \end{center}
\end{table}

注意:该请求必须是POST请求。

例子:

\begin{figure}[H]
\begin{verbatim}
访问:curl -b "_session_id=f03bef9371c119b6fcecbeefdaaac1b2"
       -d "user_id=2&follow_id=1" /follows

返回值:

{
"id":1,
"user_id":2,
"follow_id":1
}

\end{verbatim}
\end{figure}



\section{删除我的关注的人}

\begin{table}[H]
   \begin{center}
\begin{tabular}{|c|c|c|p{12cm}|}
\hline
\multicolumn{4}{|c|}{/follows/delete} \\
\hline\hline
 \  参数  & 类型 & 必填 &  说明  \\
\hline
 user\_id  & 整数 & 必须 &  当前登录用户的id\\
\hline
 follow\_id  & 整数 & 必须 &  关注用户的id\\
\hline
\end{tabular}
   \end{center}
\end{table}

注意:该请求必须是POST请求。

例子:

\begin{figure}[H]
\begin{verbatim}
访问:curl -b "_session_id=f03bef9371c119b6fcecbeefdaaac1b2"
       -d "user_id=2&follow_id=3" /follows/delete

返回值:

{
"deleted":3
}

\end{verbatim}
\end{figure}





\section{粉丝列表}

\begin{table}[H]
   \begin{center}
\begin{tabular}{|c|c|c|p{12cm}|}
\hline
\multicolumn{4}{|c|}{/follow\_info/followers} \\
\hline\hline
 \  参数  & 类型 & 必填 &  说明  \\
\hline
 id  & 整数 & 必须 & 用户的id\\
   \hline
 page  & 整数 & 可选 & 分页,缺省值为1\\ 
 \hline
 pcount  & 整数 & 可选 & 每页的数量,缺省为20\\ 
    \hline
 name  & 字符串 & 可选 & 用户的名字\\ 
    \hline    
 hash  & 整数 & 可选 & 是否返回hash,缺省值为0\\ 
\hline

\end{tabular}
   \end{center}
\end{table}

例子:

\begin{figure}[H]
\begin{verbatim}
访问:/follow_info/followers?id=2

返回值:

[
{"count":1}
{"data":
{"id":1,
"name":"name23",
"wb_uid":1884834632,
"gender":0,
"friend":true,
"follower":true,
"birthday":null}
}
]

其中count是符合条件的粉丝的总数。

获得用户基本信息里有两个关系字段:friend和follower。这里的关系都是针对我的,这里的我就是给定ID的用户。如果我关注了该用户,那么该用户就是我的friend; 如果该用户关注了我,那么他就是我的follower。

这里返回的所有用户,其"follower"值一定为true。


\end{verbatim}
\end{figure}





\section{关注(好友)列表}

\begin{table}[H]
   \begin{center}
\begin{tabular}{|c|c|c|p{12cm}|}
\hline
\multicolumn{4}{|c|}{/follow\_info/friends} \\
\hline\hline
 \  参数  & 类型 & 必填 &  说明  \\
\hline
 id  & 整数 & 必须 &  用户的id\\
   \hline
 page  & 整数 & 可选 & 分页,缺省值为1\\ 
 \hline
 pcount  & 整数 & 可选 & 每页的数量,缺省为20\\ 
     \hline
 name  & 字符串 & 可选 & 用户的名字\\ 
     \hline
 hash  & 整数 & 可选 & 是否返回hash,缺省值为0\\ 
\hline

\end{tabular}
   \end{center}
\end{table}

返回的内容格式同粉丝列表一样

这里返回的所有用户,其"friend"值一定为true。


\section{黑名单列表}

\begin{table}[H]
   \begin{center}
\begin{tabular}{|c|c|c|p{12cm}|}
\hline
\multicolumn{4}{|c|}{/blacklists} \\
\hline\hline
 \  参数  & 类型 & 必填 &  说明  \\
\hline
 id  & 整数 & 必须 &  用户的id\\
   \hline
 page  & 整数 & 可选 & 分页,缺省值为1\\ 
 \hline
 pcount  & 整数 & 可选 & 每页的数量,缺省为20\\ 
     \hline
 name  & 字符串 & 可选 & 用户的名字\\ 
     \hline
 hash  & 整数 & 可选 & 是否返回hash,缺省值为0\\ 
\hline
\end{tabular}
   \end{center}
\end{table}

返回的内容格式和好友列表一样。


\section{添加黑名单}

\begin{table}[H]
   \begin{center}
\begin{tabular}{|c|c|c|p{12cm}|}
\hline
\multicolumn{4}{|c|}{/blacklists/create} \\
\hline\hline
 \  参数  & 类型 & 必填 &  说明  \\
\hline
 user\_id  & 整数 & 必须 &  当前登录用户的id\\
\hline
 block\_id  & 整数 & 必须 &  加黑阻止的用户的id\\
 \hline
 report  & 整数 & 可选 &  是否同时举报被加黑的用户,0不举报1举报\\
\hline
\end{tabular}
   \end{center}
\end{table}

注意:该请求必须是POST请求。

\begin{figure}[H]
\begin{verbatim}

返回值:

{
"id":1,
"user_id":2,
"report":false,
"block_id":1
}

\end{verbatim}
\end{figure}



\section{删除黑名单}

\begin{table}[H]
   \begin{center}
\begin{tabular}{|c|c|c|p{12cm}|}
\hline
\multicolumn{4}{|c|}{/blacklists/delete} \\
\hline\hline
 \  参数  & 类型 & 必填 &  说明  \\
\hline
 user\_id  & 整数 & 必须 &  当前登录用户的id\\
\hline
 block\_id  & 整数 & 必须 &  加黑用户的id\\
\hline
\end{tabular}
   \end{center}
\end{table}

注意:该请求必须是POST请求。

例子:

\begin{figure}[H]
\begin{verbatim}
访问:curl -b "_session_id=f03bef9371c119b6fcecbeefdaaac1b2"
       -d "user_id=2&block_id=3" /blacklists/delete

返回值:

{
"deleted":3
}

\end{verbatim}
\end{figure}




\section{XMPP协议接口}


\subsection{与Openfire聊天服务器接口}
脸脸网的XMPP服务器采用的是Openfire。

\begin{enumerate}
\item 脸脸网的用户id加上"@dface.cn"就是openfire的jid,两个系统的密码一致。
\item 商家的id加上"@c.dface.cn"就是现场聊天室的jid。
\item 脸脸的用户名就是openfire的用户名,且也就是聊天时的用户名。Xmpp协议允许用户在加入聊天室时取一个名字,脸脸的聊天室应该忽略该名字。
\item 脸脸中的关注和粉丝关系是单向的,和xmpp中的好友roster关系无关。
\item 只要用户登录,其xmpp协议中的presence就是在线,没有其它状态。
\item 脸脸中的隐身也和xmpp中的隐身状态无关。
\item xmpp中的任意两个用户可以发送消息,不管对方是否隐身,只要对方未设置黑名单即可。如果对方未登录,那么就是离线消息。
\end{enumerate}


\subsection{脸脸中的聊天用户的三种类型}

\begin{enumerate}
\item管理员,其jid固定为"502e6303421aa918ba000001@dface.cn";
\item商家,其jid为's+商家id@dface.cn';
\item个人,其jid为'个人id@dface.cn'。
\end{enumerate}
如果一个jid以字母s开头,那么它一定是商家。


\subsection{摇一摇及其消息格式}
目前,用户在聊天室摇一摇时,客户端发送“用户名 摇了摇手机和大家say hello~”给服务器。这导致要分析摇一摇数据很困难,所以要为摇一摇定义特殊的消息格式。

\begin{verbatim}
摇一摇的消息格式为:
 [摇一摇:$name]

其它客户端收到摇一摇的消息后,再将其转换为文字描述。
\end{verbatim}



\subsection{个人之间聊天的消息发送状态确认}
XMPP协议的消息发送状态确认参考规范“\href{http://xmpp.org/extensions/xep-0022.html}{XEP-0022: Message Events}”。其定义了四种消息事件:Offline、Delivered、Displayed、Composing。Dface只需要其中的两种。

\begin{verbatim}
当接收端收到消息时,发送delivered确认消息,例如:
<message id="Kk98S-16" to="s6@dface.cn/ylt">
<x xmlns="jabber:x:event"><delivered/><id>purplea5e6669c</id></x>
</message>

当接收端展示消息时,发送displayed确认消息,例如:
<message id="Kk98S-17" to="s6@dface.cn/ylt">
<x xmlns="jabber:x:event"><displayed/><id>purplea5e6669c</id></x>
</message>

发送端根据从接收端获得的状态通知更改单条消息的发送状态。

只有类型是message,且body不为空的消息需要确认状态。

当本地无法发送消息时(比如无网络、或者连接不上xmpp服务器),消息的状态为“发送失败”。发送失败的消息可以再次发送。

\end{verbatim}


\section{聊天时发送图片}
聊天发图主要流程是:客户端选择(拍摄)一张照片,通过http上传到服务器。上传完成后在xmpp里发送一个特定格式的消息。其它客户端收到消息后获取图片显示并发送回执消息。

\subsection{上传图片}

\subsubsection{在聊天室上传图片}

\begin{table}[H]
   \begin{center}
\begin{tabular}{|c|c|c|p{12cm}|}
\hline
\multicolumn{4}{|c|}{/photos/create} \\
\hline\hline
 \  参数  & 类型 & 必填 &  说明  \\
\hline
 photo[img]  & file & 必须 &  头像文件\\
 \hline
 photo[room]  & 字符串 & 必须  &  聊天室的id\\
 \hline
 photo[weibo]  & 0/1 & 必须  &  是否同步发送到微博\\
 \hline
\end{tabular}
   \end{center}
\end{table}

注意:

\begin{enumerate}
\item 该请求必须是POST请求,enctype="multipart/form-data"。
\item 所上传文件的type必须是image/jpeg', 'image/gif' 或者'image/png'。
\item 图片文件必须小于5M。
\item 图片名为logo;缩略图分为两种大小,logo\_thumb为75*75的png,logo\_thumb2为150*150的png。
\end{enumerate}

例子:

\begin{figure}[H]
\begin{verbatim}
访问:curl -F "photo[img]=@firefox.png;type=image/png" 
/photos

{"_id":"502fd10abe4b19ed48000002","img_file_name":"7.png","img_content_type":"image/png","img_file_size":99422,"img_updated_at":"2012-08-18T17:29:46Z","room":"237","user_id":"502e61bfbe4b1921da000001","logo":"/system/imgs/502fd10abe4b19ed48000002/original/7.png?1345310986","logo_thumb":"/system/imgs/502fd10abe4b19ed48000002/thumb/7.jpg?1345310986","logo_thumb2":"/system/imgs/502fd10abe4b19ed48000002/thumb2/7.jpg?1345310986","id":"502fd10abe4b19ed48000002"}

\end{verbatim}
\end{figure}

\subsubsection{个人聊天时上传图片}

\begin{table}[H]
   \begin{center}
\begin{tabular}{|c|c|c|p{12cm}|}
\hline
\multicolumn{4}{|c|}{/photo2s/create} \\
\hline\hline
 \  参数  & 类型 & 必填 &  说明  \\
\hline
 photo[img]  & file & 必须 &  头像文件\\
 \hline
 photo[to\_uid]  & 字符串 & 必须 &  接收人的id\\
\hline
\end{tabular}
   \end{center}
\end{table}


\subsection{发送消息}
当图片发送完成后,客户端发送一个xmpp消息,其中的body内容格式为:

\begin{verbatim}
[img:$id]

比如上面的图片发送成功后,发送消息"[img:502fd10abe4b19ed48000002]"。
\end{verbatim}

\subsection{收到消息后获取图片}
\subsubsection{在聊天室收到消息后获取图片}

接收端收到消息后,判断body的内容是否符合图片消息的格式。如果符合就调用下面的接口获取图片。

\begin{table}[H]
   \begin{center}
\begin{tabular}{|c|c|c|p{12cm}|}
\hline
\multicolumn{4}{|c|}{/photos/show} \\
\hline\hline
 \  参数  & 类型 & 必填 &  说明  \\
  \hline
 id  & 字符串 & 必须 & 图片id\\
\hline
 size  & 整数 & 可选 &  目前0代表原图;1代表thumb缩略图,大小为75*75;2代表thumb2缩略图,大小为150*150。默认为1。\\ 
\hline
\end{tabular}
   \end{center}
\end{table}



\subsubsection{在个人聊天时收到消息后获取图片}
\begin{table}[H]
   \begin{center}
\begin{tabular}{|c|c|c|p{12cm}|}
\hline
\multicolumn{4}{|c|}{/photo2s/show} \\
\hline\hline
 \  参数  & 类型 & 必填 &  说明  \\
  \hline
 id  & 字符串 & 必须 & 图片id\\
\hline
 size  & 整数 & 可选 &  目前0代表原图;1代表thumb缩略图,大小为75*75;2代表thumb2缩略图,大小为150*150。默认为1。\\ 
\hline
\end{tabular}
   \end{center}
\end{table}


\subsection{消息状态}
如果是个人之间的传图,接收者发送状态更新消息给发送者。发送者将状态显示在图片旁边。


\section{优惠券}

\subsection{商家推送优惠券}
当用户使用脸脸时,比如签到/摇一摇等,可以收到商家推送过来的优惠券。优惠券也是一个xmpp消息,其中的body内容格式为:

\begin{verbatim}
[优惠券:$name:$shop:$id]

 其中,name是优惠券的名称、shop是商家的名称,id是该优惠券的id。
 
 所有优惠券都统一显示在会话中的“优惠券”文件夹中。优惠券的消息状态和其他消息同样处理。
\end{verbatim}

\subsection{优惠券展示}
优惠券以图片的方式展示,由服务器端生成该图片。当用户进入优惠券文件夹查看优惠券时,调用下面的接口获得优惠券的图片。

\begin{table}[H]
   \begin{center}
\begin{tabular}{|c|c|c|p{12cm}|}
\hline
\multicolumn{4}{|c|}{/coupons/img} \\
\hline\hline
 \  参数  & 类型 & 必填 &  说明  \\
\hline
 \  id  & 字符串 & 必填 &  该优惠券的id  \\
\hline
 size  & 整数 & 可选 &  目前0代表原图(610, 306);1代表缩略图,大小为(305,153)。默认为1。\\ 
\hline
\end{tabular}
   \end{center}
\end{table}



\subsection{优惠券使用}
目前,所有优惠券都只支持单次使用,使用后即过期。当用户确认使用优惠券时,调用如下的接口:
\begin{table}[H]
   \begin{center}
\begin{tabular}{|c|c|c|p{12cm}|}
\hline
\multicolumn{4}{|c|}{/coupons/use} \\
\hline\hline
 \  参数  & 类型 & 必填 &  说明  \\
\hline
 \  id  & 字符串 & 必填 &  该优惠券的id  \\
\hline
\end{tabular}
   \end{center}
\end{table}


\section{关于程序各页面之间的切换规则}

程序的首页是选择现场页面。现场则是一个聊天室页面。

1、程序干净启动时,要判断用户是否处于登录状态。

1.1如果已经登录,进入选择现场页面,并从服务器端获取新的商家列表。

1.2如果没有登录,进入登录页面。


2、程序从睡眠状态被激活时,要判断距离上次的运行时间。

2.1 如果不足12个小时,上次被任务切换时停留在那个页面就仍然停留在该页面。

2.2 如果超过12个小时,进入选择现场页面,并从服务器端获取新的商家列表。


3、从现场返回到选择现场时,判断当前的经纬度和上次进入现场时的经纬度。

3.1如果超过1000米,提示用户是否刷新现场位置。

3.2 如果不足1000米,不处理。用户要选择其它现场的话,从滚轮中选择。


4、从其它tab点击现场时,判断用户是否摇进过现场

4.1 如果摇进过某个现场,直接进入该现场聊天室。进入时要特殊处理下部的tab,提供动画。只有这种情况需要这么特殊处理。

4.2 如果没有摇,那么就是选择现场页面。


5、从附近里点击商家的时候。不进入现场。显示商家的聊天室。但是该聊天室不是现场,没有聊天工具栏。在附近里进入商家聊天室时不能聊天。



\newpage


\end{document}
