
\section{在聊天室发图}
聊天发图主要流程是:客户端选择(拍摄)一张照片,通过http上传到服务器。上传完成后在xmpp里发送一个特定格式的消息。其它客户端收到消息后获取图片显示并发送回执消息。

\subsection{上传图片}

\begin{table}[H]
   \begin{center}
\begin{tabular}{|c|c|c|p{12cm}|}
\hline
\multicolumn{4}{|c|}{/photos/create} \\
\hline\hline
 \  参数  & 类型 & 必填 &  说明  \\
\hline
 photo[img]  & file & 必须 &  头像文件\\
 \hline
 photo[room]  & 字符串 & 必须  &  聊天室的id\\
 \hline
 photo[weibo]  & 0/1 & 必须  &  是否同步发送到微博\\
   \hline
 photo[desc]  & 字符串 & 可选  &  图像的说明文字\\
 \hline
  photo[t]  & 整数 & 可选 &  图片类型:1拍照;2选自相册\\
 \hline
\end{tabular}
   \end{center}
\end{table}

注意:

\begin{enumerate}
\item 该请求必须是POST请求,enctype="multipart/form-data"。
\item 所上传文件的type必须是image/jpeg', 'image/gif' 或者'image/png'。
\item 图片文件要在客户端先压缩到640*640。
\item 图片名为logo;缩略图分logo\_thumb2为200*200的png。
\end{enumerate}

例子:

\begin{figure}[H]
\begin{verbatim}
访问:curl -F "photo[img]=@firefox.png;type=image/png" 
/photos

{"_id":"509a1ffcbe4b1982a4000002",
"weibo":false,"room":"1",
"user_id":"502e61bfbe4b1921da000005",
"img_tmp":"20121107-1646-42114-4251/111.jpg",
"logo":"/uploads/tmp/20121107-1646-42114-4251/111.jpg",
"logo_thumb2":null,"id":"509a1ffcbe4b1982a4000002"}

当返回的响应中包含img_tmp字段,且logo_thumb2为null时,说明此时图片上传到了服务器,但是还未启动后台处理。后台异步处理图片,包括生成缩略图以及上传到阿里云。

\end{verbatim}
\end{figure}


\subsection{发送消息}
当图片发送完成后,服务器端发送一个xmpp消息,其中的body内容格式为:

\begin{verbatim}
[img:$id]$desc

比如上面的图片发送成功后,发送消息"[img:502fd10abe4b19ed48000002]"。其中$desc是可选的图片说明文字。
\end{verbatim}

\subsection{在聊天室收到消息后获取图片}

接收端收到消息后,判断body的内容是否符合图片消息的格式。如果符合就调用下面的接口获取图片。

\begin{table}[H]
   \begin{center}
\begin{tabular}{|c|c|c|p{12cm}|}
\hline
\multicolumn{4}{|c|}{/photos/show} \\
\hline\hline
 \  参数  & 类型 & 必填 &  说明  \\
  \hline
 id  & 字符串 & 必须 & 图片id\\
\hline
 size  & 整数 & 可选 &  目前0代表原图;2代表thumb2缩略图,大小为200*200。默认为2。\\ 
\hline
\end{tabular}
   \end{center}
\end{table}

聊天室里发图不需要发送已读回执消息。
在现场收到图片时,如果图片id以“faq”开头,不添加到照片墙上。18:02:24

\section{商家照片墙}

\subsection{删除聊天室图片}

\begin{table}[H]
   \begin{center}
\begin{tabular}{|c|c|c|p{12cm}|}
\hline
\multicolumn{4}{|c|}{/photos/delete} \\
\hline\hline
 \  参数  & 类型 & 必填 &  说明  \\
  \hline
 id  & 字符串 & 必须 & 图片id\\
\hline
\end{tabular}
   \end{center}
\end{table}
个人只能删除自己发布的照片。



\subsection{对聊天室图片点赞}

\begin{table}[H]
   \begin{center}
\begin{tabular}{|c|c|c|p{12cm}|}
\hline
\multicolumn{4}{|c|}{/photos/like} \\
\hline\hline
 \  参数  & 类型 & 必填 &  说明  \\
  \hline
 id  & 字符串 & 必须 & 图片id\\
\hline
\end{tabular}
   \end{center}
\end{table}

\subsection{对聊天室图片取消赞}

\begin{table}[H]
   \begin{center}
\begin{tabular}{|c|c|c|p{12cm}|}
\hline
\multicolumn{4}{|c|}{/photos/dislike} \\
\hline\hline
 \  参数  & 类型 & 必填 &  说明  \\
  \hline
 id  & 字符串 & 必须 & 图片id\\
\hline
\end{tabular}
   \end{center}
\end{table}
个人只能取消自己给的赞。


\subsection{对聊天室图片点评}

\begin{table}[H]
   \begin{center}
\begin{tabular}{|c|c|c|p{12cm}|}
\hline
\multicolumn{4}{|c|}{/photos/comment} \\
\hline\hline
 \  参数  & 类型 & 必填 &  说明  \\
  \hline
 id  & 字符串 & 必须 & 图片id\\
   \hline
 text  & 字符串 & 必须 & 评论的内容\\
\hline
\end{tabular}
   \end{center}
\end{table}

返回的数据结构是评论的数组。
 id: 评论人的id
 name: 评论人的名字
 txt:评论的内容
 t:时间
 rid:被回复者的id
 rname: 被回复者的名字


\subsection{对聊天室图片点评进行回复}

\begin{table}[H]
   \begin{center}
\begin{tabular}{|c|c|c|p{12cm}|}
\hline
\multicolumn{4}{|c|}{/photos/recomment} \\
\hline\hline
 \  参数  & 类型 & 必填 &  说明  \\
  \hline
 id  & 字符串 & 必须 & 图片id\\
 \hline
 rid  & 字符串 & 必须 & 被回复者的id\\
   \hline
 text  & 字符串 & 必须 & 评论的内容\\
\hline
\end{tabular}
   \end{center}
\end{table}
这里的rid是被回复者的id,而发布回复者的id是从session中自动获取的。

这里的回复其实是针对人的,如果一个人在一张图片下发布了多个评论,那么这里的回复不针对特定的那条评论。


\subsection{对聊天室图片删除点评}

\begin{table}[H]
   \begin{center}
\begin{tabular}{|c|c|c|p{12cm}|}
\hline
\multicolumn{4}{|c|}{/photos/delcomment} \\
\hline\hline
 \  参数  & 类型 & 必填 &  说明  \\
  \hline
 id  & 字符串 & 必须 & 图片id\\
  \hline
 text  & 字符串 & 必须 & 评论的内容\\
\hline
\end{tabular}
   \end{center}
\end{table}
个人只能删除自己发的点评。

评论的删除是区分两种情况:删除/隐藏,评论的发布人可以删除,图片的发布人可以隐藏。
对第三方来说都是删除。

如果一个评论是我发的,同时图也是我发的,那么此时对评论操作就是删除,不需要隐藏。


\subsection{对聊天室图片隐藏点评}

\begin{table}[H]
   \begin{center}
\begin{tabular}{|c|c|c|p{12cm}|}
\hline
\multicolumn{4}{|c|}{/photos/hidecomment} \\
\hline\hline
 \  参数  & 类型 & 必填 &  说明  \\
  \hline
 id  & 字符串 & 必须 & 图片id\\
  \hline
 uid  & 字符串 & 必须 & 评论的发布者id\\
   \hline
 text  & 字符串 & 必须 & 评论的内容\\
\hline
\end{tabular}
   \end{center}
\end{table}
只有图片的发布者才能隐藏他人给自己图片发的点评。调用本接口的必须是图片的发布人。
点评隐藏后只有点评的发布者一人能够看到。


\subsection{获得当前登录用户的照片墙}
用户的照片墙由其本人在各个现场照片墙发布的照片组成。

\begin{table}[H]
   \begin{center}
\begin{tabular}{|c|c|c|p{12cm}|}
\hline
\multicolumn{4}{|c|}{/photos/users} \\
\hline\hline
 \  参数  & 类型 & 必填 &  说明  \\
\hline
 page  & 整数 & 可选 & 分页,缺省值为1\\ 
 \hline
 pcount  & 整数 & 可选 & 每页的数量,缺省为20\\ 
\hline
\end{tabular}
   \end{center}
\end{table}

输出格式参见:\ref{photowall}



\section{个人聊天时发图}

\subsection{个人聊天时上传图片}

\begin{table}[H]
   \begin{center}
\begin{tabular}{|c|c|c|p{12cm}|}
\hline
\multicolumn{4}{|c|}{/photo2s/create} \\
\hline\hline
 \  参数  & 类型 & 必填 &  说明  \\
\hline
 photo[img]  & file & 必须 &  头像文件\\
 \hline
 photo[to\_uid]  & 字符串 & 必须 &  接收人的id\\
\hline
  photo[t]  & 整数 & 可选 &  图片类型:1拍照;2选自相册\\
 \hline
\end{tabular}
   \end{center}
\end{table}


\subsection{发送消息}
当图片发送完成后,服务器端发送一个xmpp消息,其中的body内容格式为:

\begin{verbatim}
[img:$id]

\end{verbatim}

\subsection{在个人聊天时收到消息后获取图片}
\begin{table}[H]
   \begin{center}
\begin{tabular}{|c|c|c|p{12cm}|}
\hline
\multicolumn{4}{|c|}{/photo2s/show} \\
\hline\hline
 \  参数  & 类型 & 必填 &  说明  \\
  \hline
 id  & 字符串 & 必须 & 图片id\\
\hline
 size  & 整数 & 可选 &  目前0代表原图;2代表thumb2缩略图,大小为200*200。默认为2。\\ 
\hline
\end{tabular}
   \end{center}
\end{table}

注意:聊天时发图,取消的100*100大小的缩略图。[2012-11-7]


\subsection{消息状态}
和普通的文字消息一样,接收者收到图片后发送状态更新消息(收到/已读)给发送者。发送者将状态显示在图片旁边。



