\documentclass[cs4size]{ctexartutf8} 
\usepackage[unicode={true}]{hyperref}
\hypersetup{colorlinks,%
                 citecolor=black,%
                 filecolor=black,%
                 linkcolor=black,%
                 urlcolor=blue,%
                 pdftex}

\usepackage{graphicx}
\usepackage{float}

				
\author{YXY}
\title{脸脸网旅游相关接口API}

\begin{document} 
\maketitle
\tableofcontents

\newpage

\textbf{注意事项}:
\begin{enumerate}
\item 本接口API采用WEB协议,输入输出采用json格式。
\item 只有开发调试时可用http的接口,正式发布时全部使用https接口。
\item 接口统一采用UTF-8编码。
\item 提交数据统一用POST请求,查询数据统一用GET请求。
\item 当调用出错时,会在返回的json中包含"error"信息。
\item 服务提供方采用固定的IP地址:42.121.79.210。
\item 调用方要绑定IP地址,只有绑定的IP地址才能调用本接口。
\end{enumerate}

\newpage

\section{基础数据库}

在建立旅行线路/旅行团之前,先要建立景点/商家的基础数据库。网格保证不同的景点/商家,发给脸脸的时候id不同。脸脸负责保证网格发过来的景点/商家在脸脸库的唯一性,不与已有数据冲突。

\subsection{创建地点}
商家/景点统一建库,都归类为地点。这样在定义旅行线路时,可以混合景点与商家。

\begin{table}[H]
   \begin{center}
\begin{tabular}{|c|c|c|p{12cm}|}
\hline
POST & \multicolumn{3}{|c|}{https://42.121.79.210/api/xshop/create} \\
\hline\hline
 \  参数  & 类型 & 必填 &  说明  \\
  \hline
 id  & 字符串 & 必须 &  地点编号\\
 \hline
 name  & 字符串 & 必须 &  景点名称\\
\hline
 addr  & 字符串 & 可选 &  地址\\
\hline
 type  & 字符串 & 可选 &  如果是商家,商家的类型\\
\hline
 lat  & 浮点数 & 必须 & 经度\\
\hline
 lng  &  浮点数 & 必须 & 纬度\\ 
\hline
\end{tabular}
   \end{center}
\end{table}

商家的类型指餐饮/酒店/酒吧/购物等类型。
经纬度都是实际经纬度,不是国内地图加偏后的经纬度。

\begin{verbatim}
成功返回值:
{"ok":"id"}

若有错误,返回值:
{"error":"错误原因"}
\end{verbatim}



\subsection{查询地点}

\begin{table}[H]
   \begin{center}
\begin{tabular}{|c|c|c|p{12cm}|}
\hline
GET & \multicolumn{3}{|c|}{https://42.121.79.210/api/xshop/find} \\
\hline\hline
 \  参数  & 类型 & 必填 &  说明  \\
 \hline
 id  & 字符串 & 必须 &  编号\\
\hline
\end{tabular}
   \end{center}
\end{table}

\begin{verbatim}
成功返回值:
{
"id" : id,
"dface_id":"该地点在脸脸中的id" ,
"name" : name,
"addr" : 地址
...
}


\end{verbatim}


\section{旅行线路}

\subsection{创建旅行线路}

创建旅行线路涉及到的景点/商家需要提前先建立基本数据库。

\begin{table}[H]
   \begin{center}
\begin{tabular}{|c|c|c|p{12cm}|}
\hline
POST & \multicolumn{3}{|c|}{https://42.121.79.210/api/xline/create} \\
\hline\hline
 \  参数  & 类型 & 必填 &  说明  \\
 \hline
 id  & 字符串 & 必须 &  旅行线路编号\\
\hline
 data  & json字符串 & 必须 &  旅行线路的具体描述\\
\hline
\end{tabular}
   \end{center}
\end{table}

\begin{verbatim}
data的格式如下:
{
"name": "线路名称"
"arr": [
  {shopid: "地点编号1", time:"时间1"},
  ...
  {shopid: "地点编号n", time:"时间n"}
 ]
}

其中的time格式需要讨论,基本按照网格的原有格式提供。

成功返回值:
{"ok":"id"}
\end{verbatim}



\subsection{查询旅行线路}

\begin{table}[H]
   \begin{center}
\begin{tabular}{|c|c|c|p{12cm}|}
\hline
POST & \multicolumn{3}{|c|}{https://42.121.79.210/api/xline/find} \\
\hline\hline
 \  参数  & 类型 & 必填 &  说明  \\
 \hline
 id  & 字符串 & 必须 &  旅行线路编号\\
\hline
\end{tabular}
   \end{center}
\end{table}

返回值和对应创建接口的data结构相同。



\section{旅行团}

\subsection{创建旅行团}

\begin{table}[H]
   \begin{center}
\begin{tabular}{|c|c|c|p{12cm}|}
\hline
POST & \multicolumn{3}{|c|}{https://42.121.79.210/api/xgroup/create} \\
\hline\hline
 \  参数  & 类型 & 必填 &  说明  \\
 \hline
 id  & 字符串 & 必须 &  旅行团编号\\
\hline
 data  & json字符串 & 必须 &  旅行团的具体描述\\
\hline
\end{tabular}
   \end{center}
\end{table}

\begin{verbatim}
data的格式如下:
{
  "name" : "西湖二日游",
  "tid":"团号001"
  "line_id" : "线路编号",
  "fat" : "开始日期",
  "tat" : "结束日期",
   "users": [
        {"name":"张三"
        "phone":"15868870000"
        "sfz":"1234567890xxxxxxxxx"},
        {"name":"李四"
        "phone":"15868870001"
        "sfz":"1234567890xxxxxxxx1"}
  ]
  "user_count": 2
}

其中的fat/tat格式需要讨论,基本按照网格的原有格式提供。

注意:旅行团编号id要求所有旅行团唯一,而团号tid则保证在同一线路里唯一即可。

成功返回值:
{"ok":"id"}
\end{verbatim}



\subsection{查询旅行线路}

\begin{table}[H]
   \begin{center}
\begin{tabular}{|c|c|c|p{12cm}|}
\hline
POST & \multicolumn{3}{|c|}{https://42.121.79.210/api/xgroup/find} \\
\hline\hline
 \  参数  & 类型 & 必填 &  说明  \\
 \hline
 id  & 字符串 & 必须 &  旅行线路编号\\
\hline
\end{tabular}
   \end{center}
\end{table}
返回值和对应创建接口的data结构相同。


\section{旅行线路合作商家}

\subsection{创建旅行线路的合作商家}

\begin{table}[H]
   \begin{center}
\begin{tabular}{|c|c|c|p{12cm}|}
\hline
POST & \multicolumn{3}{|c|}{https://42.121.79.210/api/xpartner/create} \\
\hline\hline
 \  参数  & 类型 & 必填 &  说明  \\
 \hline
 id  & 字符串 & 必须 &  旅行线路编号\\
\hline
 data  & json字符串 & 必须 &  合作商家\\
\hline
\end{tabular}
   \end{center}
\end{table}

\begin{verbatim}
data的格式如下:
{
"线路景点编号1": ["合作商家编号1", "合作商家编号2"},
  ...
"线路景点编号n": ["合作商家编号m",..., "合作商家编号mn"}
}

其中线路景点编号必须是给定id的旅行线路中的一个景点。

成功返回值:
{"ok":"id"}
\end{verbatim}



\subsection{查询旅行线路的合作商家}

\begin{table}[H]
   \begin{center}
\begin{tabular}{|c|c|c|p{12cm}|}
\hline
POST & \multicolumn{3}{|c|}{https://42.121.79.210/api/xpartner/find} \\
\hline\hline
 \  参数  & 类型 & 必填 &  说明  \\
 \hline
 id  & 字符串 & 必须 &  旅行线路编号\\
\hline
\end{tabular}
   \end{center}
\end{table}
返回值和对应创建接口的data结构相同。




\section{脸脸网产生的数据的接口}
在整个系统的使用过程中,会在脸脸上产生大量的数据,包括:优惠券发布/下载/使用情况、商家点评、游客发布的图片等。这些数据脸脸全部开放管理后台,可以管理和查询。



\newpage


\end{document}
